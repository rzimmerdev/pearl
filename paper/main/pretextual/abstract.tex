%% USPSC-Abstract.tex
%\autor{Silva, M. J.}
\begin{resumo}[Abstract]
 \begin{otherlanguage*}{english}
	\begin{flushleft} 
		\setlength{\absparsep}{0pt} % ajusta o espaçamento dos parágrafos do resumo		
 		\SingleSpacing  		\imprimirautorabr~~\textbf{\imprimirtitleabstract}.	\imprimirdata.  \pageref{LastPage} p. 
		%Substitua p. por f. quando utilizar oneside em \documentclass
		%\pageref{LastPage} f.
		\imprimirtipotrabalhoabs~-~\imprimirinstituicao, \imprimirlocal, 	\imprimirdata. 
 	\end{flushleft}
	\OnehalfSpacing
    The increasing complexity and computational demands of high-frequency trading (HFT) environments present significant challenges for
    deploying Reinforcement Learning (RL) agents in distributed training systems.
    Conventional distributed RL architectures replicate environment simulators across workers,
    leading to similar computations being replicated in separate environments.
    The usual approach to multi-agent reinforcement learning has agents utilizing shared environments,
    but rarely parallelize multiple environments in a way that minimizes overhead, reduces communication overhead all while maintaining training stability,
    especially when multiple agents interact with computation intensive non-stationary market dynamics.
    This thesis introduces the usage of parallel environments for asynchronous reinforcement learning in a multi-agent context,
    designed to reduce computational overhead and improve learning stability in the high-frequency trading domain.
    Our approach, which we nicknamed parallel environments for asynchronous reinforcement learning (PEARL) replaces per-worker environment replicas with shared,
    parallelized market simulators that adapts the multi-agent RL paradigm for asynchronous interactions with parallel limit order book simulators.
    The system leverages a brokerless messaging layer to decouple trajectory collection from policy learning,
    reducing synchronization lag and enabling higher throughput.
    Empirical evaluation using a proximal policy optimization-based agent showed that the PEARL architecture achieves improved convergence rates, greater sample efficiency,
    and lower latency compared to a traditional replicated environment setup.
    By explicitly addressing simulation redundancy and communication delays,
    we aimed to increase scalability and training efficiency for RL algorithms in financial market making.

   \vspace{\onelineskip}
 
   \noindent 
   \textbf{Keywords}: Reinforcement Learning, Distributed Computing, Market Making, High-Frequency Trading, Deep Learning.
 \end{otherlanguage*}
\end{resumo}

