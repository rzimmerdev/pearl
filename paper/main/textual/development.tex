\chapter{Development}
\label{ch:development}

Distributed reinforcement learning (RL) has evolved considerably in response to the demands of complex tasks such as game playing and financial trading.
Early methods such as the Asynchronous Advantage Actor-Critic (A3C)~\citep{Mnih2016}
algorithm introduced the concept of multiple parallel workers collecting experiences independently,
which helped to increase training speeds and reduce reward variance.
Building on these ideas, architectures like IMPALA and Gorila~\citep{Espeholt2018,Nair2015} have decoupled data collection from model training by using
a decentralized learner and worker structure, which periodically synchronizes the policy with the updated gradient
from the workers through gradient passing.
The decentralized learner design enables significant scalability across heterogeneous computing resources and facilitates high-throughput training.

As mentioned in \fullref{ch:introduction}, a key challenge present in these systems is the communication overhead incurred when transmitting experiences or
policy gradient updates between the learner and the workers.
The IMPALA algorithm aggregates trajectories from multiple workers into
batches for efficient gradient computation while managing the staleness of policy information.
More recent approaches, such as SEED RL\citep{Espeholt2020}, have further optimized communication by bundling inference with training updates.
This reduces the latency associated with model parameter transmission and minimizes network congestion, thereby improving overall training efficiency.
In parallel, methods like Ape-X\citep{Horgan2018} have introduced distributed prioritized experience replay, where critical experiences are sampled with higher probability,
enhancing data efficiency in scenarios where high-frequency data generation is crucial.

In the context of financial applications, RL environments often integrate realistic market simulators.
These simulators statistically represent stylized facts of limit order book (LOB), which are discussed
further in \fullref{subsec:environment}.
For such simulators, the distributed architecture must handle not only the high throughput of simulation data but also the
precise timing of order execution and asynchronous policy updates.
This thesis builds upon existing distributed algorithms through means of shared parallel environments,
designed specifically to address the nature of high-frequency trading applications.
In the following sections, we present the design and implementation of our proposed architecture,
which we refer to as Parallel Environments for Asynchronous Reinforcement Learning (PEARL).

\section{Environment and Agent Methodology}
\label{sec:methodology}

As mentioned in \fullref{ch:introduction}, the proposed framework enables multiple RL agents
to share a single environment instance rather than running isolated simulations and thus
minimize redundant computations and communication overhead from synchronizing workers interacting with independent environment instances.
The dynamics of the simulator and the implemented agent used for testing the framework are described in the following subsections.

%! Author = rzimmerdev
%! Date = 3/28/25

% Preamble
\documentclass[11pt]{article}

% Packages
\usepackage{amsmath}
\usepackage{graphicx}
\usepackage{amssymb}

% Document
\begin{document}
    \subsection{Simulated Trading Environment and Limit Order Book Dynamics}
    \label{subsec:environment}
    The simulated environment is designed to replicate key aspects of a modern financial market by incorporating a limit order book (LOB) mechanism.
    A LOB is the core structure in many electronic trading systems, where buy and sell orders are organized by price and time priority,
    and can be quoted continuously, thereby reflecting the real-time supply and demand of an asset.
    At any given moment, the orders available in the book are accessible to traders in a sorted fashion, organized per price level,
    as shown in~\cref{fig:lob}.

    Market participants interact with the exchange by submitting orders that represent their intention to buy, called a bid order,
    or their intention to sell, called an ask order.
    Specifically, each action corresponds to placing a limit order—detailing a specified quantity at a predetermined price.
    When an agent submits an order, it is added to the appropriate side of the LOB, where orders are queued according to their price and
    the time at which they were submitted.
    An order is executed if there is a matching order on the opposite side of the book that satisfies the price condition,
    or if it is a market order, in which case no price is specified and the order is matched at the best available price;
    otherwise, it remains in the book until a suitable counter-order appears or if it is cancelled by the trader.

    The simulation employs a discrete-event framework to model these processes accurately.
    At each simulation step, the environment generates an observation that includes critical details such as the best bid and ask prices,
    the order book depth across various price levels, and records of recent trade executions.
    These observations are then provided to the agents, enabling them to update their strategies based on the evolving market state.
    This event-driven process is incorporated into the simulator, as it mimics the intricate dynamics of order matching observed in real markets.

    Order submission and execution in the simulated environment are handled asynchronously.
    Agents send their orders via a messaging layer implemented with the ZeroMQ framework, a high-performance messaging library,
    to ensure that multiple agents can interact with the shared LOB concurrently without incurring in significant latency.
    This asynchronous architecture not only improves simulation throughput but also closely aligns with
    the latency-sensitive nature of high-frequency trading environments, usually measured in the milliseconds.

    By integrating a realistic LOB into the simulation, the environment provides a robust benchmark for testing
    distributed reinforcement learning algorithms under complex stochastic market dynamics.
    The detailed representation of market microstructure, combined with fine-grained arrival of order events,
    allows for statistically representative simulations that ensure order submission strategies
    are adequately trained to adapt to market reaction.

    \begin{figure}[htb]
        \centering
        \includegraphics[width=0.8\textwidth]{images/lob}
        \caption{Visualization of a limit order book (LOB) with buy and sell orders organized by price and time priority.}
        \label{fig:lob}
    \end{figure}

    The LOB is implemented using a Red-Black tree data structure, which allows for efficient insertion, deletion,
    and sorted or reversed traversal per price level.
    Each price level is represented as a node in the tree, with the price as the key and a queue of posted order details
    (e.g., client id, quantity, timestamp) as the value.
    New orders are sampled from a mixture of processes, and at each time step, the environment checks for crossing orders and executes trades accordingly,
    updating the order book and the agent's state, as well as the reward signal.

    The system dynamics are modeled as a continuous-time Markov Decision Process (MDP),
    where each state corresponds to some representation of the current market state.
    It can contain both LOB-observed variables, such as $N$-depth price and quantity quotes,
    as it can contain processed or fabricated features, such as $15$ window price moving averages,
    periodic volatility values, mean return rates, and others.
    The governing equation for the state dynamics is the Kolmogorov forward equation,
    which gives the probability of transitioning from some state $s \in S'$ (where $S'$ is the set of all possible states excluding final states)
    to some other state $s' \in S$ (where $S$ is the set of all possible states)
    at a future time $t \geq 0$ given the agent's action in the state $s$ was $a$:
    \begin{equation}
        \frac{\partial P(s', t|s, a)}{\partial t} = \int_S L(x|s, a, t) P(s'|x, a, t) \, dx
        \label{eq:kolmogorov}
    \end{equation}

    Here, the action $a$ chosen by the control agent was sampled or deterministically chosen according to a policy $\pi(s)$,
    so that $a \in \pi(s)$ for deterministic policies or $a \sim \pi(s)$ for stochastic policies.
    Finally, \( L(x|s, a, t) \) is the generator operator for the environment's dynamics at any given time \( t \),
    and depends on the current state $s$ and the action taken by the agent $a$.

    However, solving for the generator operator $L$ under complex market dynamics is usually extremely complex,
    and oftentimes either unfeasible or completely impossible due to the non-linear and non-gaussian nature of asset-return distributions.
    Thus, solving for the transition probabilities is often approached either by simplifying the market model and
    analytically finding closed-form solutions, as in earlier works~\citep{Avellaneda2008, Gueant2017},
    by numerically approximating the observed state transitions for simpler state spaces,
    or by numerically approximating some reward function $R$
    (e.g., by simulating trajectories or using reinforcement learning methods)~\citep{Gueant2022, Gasperov2022, Guo2023}.
    The market-making problem can be modeled using the reinforcement learning paradigm,
    representing the environment through a simulated limit order book (LOB) that reflects stylized market behaviors,
    and incorporating the Profit-and-Loss scores into a known reward function.

    For our environment, we break up the simulating of order arrivals into four separate steps:
    \begin{itemize}
        \item Sampling the next event times;
        \item Sampling the spread, volatility and mean price drift processes;
        \item Sampling the number of new events, and their order prices and quantities;
        \item Inserting the sampled orders into the LOB.
    \end{itemize}
    This setup was designed to provide a highly realistic representation of known limit-order book and
    microstructure stylized facts, such as clustered order arrivals, normal return distributions and price drifts,
    which in turn guided our choice for adopting the Reinforcement Learning approach to optimize our policy instead of trying
    to find a closed-form solution.

    The event times in our simulator follow a Hawkes process,
    a stochastic process with a self-exciting kernel.
    This property is extremely relevant for the context of limit order books,
    as it can replicate the self-exciting nature of clustered order arrivals.
    The intensity kernel \( \lambda(t) \) for the Hawkes is defined as:

    \begin{equation}
        \begin{aligned}
            \lambda(t) = \mu + \sum_{t_i < t} \phi(t - t_i)\\
            \phi(t - t_i) = \alpha e^{-\beta(t - t_i)}
        \end{aligned}
        \label{eq:hawkes}
    \end{equation}

    where \( \mu > 0 \) is the baseline intensity, and \( \alpha \), \( \beta \) govern the magnitude and decay of past events' influence.
    To generate order arrivals, we can sample from an exponential process with intensity \( \lambda(t) \),
    making the inter-arrival times exponentially distributed.
    For the second step, the bid and ask prices are modeled as separate Geometric Brownian Motion (GBM) processes, with prices evolving according to:

    \begin{gather*}
        dX_{\text{ask}} = (\mu_t + s_t) X_{\text{mid}} \, dt + \sigma_t X_{\text{mid}} dW_t\\
        dX_{\text{bid}} = (\mu_t - s_t) X_{\text{mid}} \, dt + \sigma_t X_{\text{mid}} dW_t\\
    \end{gather*}

    where \( \mu_t \) is the drift, \( s_t \) is the spread, and \(  \sigma_t \) is the volatility.
    To ensure our simulator is capable of replicating changing market dynamics,
    we model the GBM drift $\mu_t$, the spread $s_t$ and the volatility $\sigma$ separately.
    The drift \( \mu_t \) follows a mean-reverting Ornstein-Uhlenbeck process~\citep{Uhlenbeck1930},
    while the spread \( s_t \) is sampled from a Cox-Ingersoll-Ross process~\citep{Cox1985}, so we ensure it remains positive.
    Our price volatility is modeled using a GARCH(1,1) process:

    \begin{equation*}
        \begin{aligned}
            X_{\text{mid}} &= \frac{X_{\text{ask}} + X_{\text{bid}}}{2} \\
            \sigma_t^2 &= \omega + \alpha \epsilon_{t-1}^2 + \beta \sigma_{t-1}^2
        \end{aligned}
    \end{equation*}

    where \( \sigma_t \) is the volatility, \( \epsilon_{t-1} \) is the return shock, and \( \alpha \), \( \beta \) are parameters capturing volatility clustering.
    The values for $\omega$, $\alpha$, and $\beta$, as well as the default simulation hyperparameters are
    available in the annexed code and reproducibility section (\Fullref{ch:code}).
    Finally, the midprice \( X_{\text{mid}} \) used in the GBM processes is calculated as the average of the current best bid and ask prices.
    In the absence of orders, the midprice is based on the last traded price or an initial value.
    Together with the sampled ask and bid prices, order quantities \( q \) are sampled from a Poisson distribution with a fixed rate,
    and pairs of price, side (buy or sell), and quantities are combined into orders and inserted into the limit order book,
    which automatically matches received orders.
    All market variables (e.g., spread, order arrival rate, and price drift) are sampled according to the four steps described above,
    and together with the Red-Black tree structure make up the underlying order book dynamics.

\end{document}
%! Author = rzimmerdev
%! Date = 3/28/25

% Preamble
\documentclass[11pt]{article}

% Packages
\usepackage{amsmath}
\usepackage{amsfonts}

% Document
\begin{document}
    \subsection{Chosen Agent Architecture}
    \label{subsec:agent}

    As discussed in \fullref{subsec:environment}, the reinforcement learning problem is formulated as a MDP with a state space \( S \),
    action space \( A \), transition probabilities \( P(s'|s, a) \), and reward function \( R(s', a, s) \).
    The underlying goal is to learn a policy \( \pi(s) \) that maximizes the expected return over time,
    and depends only on the current state.

    In reinforcement learning, the Bellman equation recursively defines the value of a state, \( V(s) \),
    in terms of the reward and the conditional expected future state values.
    For a given policy \( \pi(s) \), the Bellman equation for the value function \( V^{\pi}(s) \) is:

    \[
        V^{\pi}(s) = \sum_{a \in A} \pi(a|s) \sum_{s' \in S} P(s'|s, a) \left( R_t + \gamma V^{\pi}(s') \right)
    \]

    where \( R_t = R(s', a_t, s)\) is the value for the reward function starting in the state $s$, and reaching state $s'$ through
    action $a_t$ at timestep $t$. \( \gamma \) is the discount factor.
    The goal is to find the optimal policy \( \pi^* \) that maximizes the expected return, which is given by:

    \[
        V^*(s) = \max_a \mathbb{E}[R_t + \gamma V^*(s') | s = s, a = a]
    \]

    Policy iteration and value iteration are methods to approximate this equation numerically,
    but they become computationally expensive for large state spaces.
    In such cases, neural networks are used to approximate the value function or the policy itself, making them suitable for complex environments.
    The Generalized Policy Iteration is a framework that unifies these methods by iteratively evaluating and improving the policy.
    Most modern reinforcement learning algorithms and the current state-of-the-art methods are based on this framework,
    including but not limited to Q-learning, Actor-Critic (and variations), and Policy Gradient methods.

    Policy Gradient methods are particularly well-suited for continuous action spaces and have been widely used in finance and trading applications
    ~\citep{Guo2023, Gasperov2022}
    The approach involves parameterizing the policy \( \pi_\theta(a|s) \) with a neural network and optimizing it directly to maximize the expected return.
    The policy gradient given a state-action pair \( (s, a) \) and policy network \( \pi_\theta \) is:

    \[
        \nabla_\theta J(\theta) = \mathbb{E}_{\tau \sim \pi_\theta} \left[ \sum_{t=0}^{T} \nabla_\theta \log \pi_\theta(a_t | s_t) A^\pi(s_t, a_t) \right]
    \]

    For Proximal Policy Optimization (PPO), the policy update is constrained to prevent excessive changes by using a clipped surrogate objective:

    \[
        L_{\text{actor}}(\theta) = \mathbb{E}_t \left[ \min \left( r_t(\theta) \hat{A}_t, \text{clip}(r_t(\theta), 1-\epsilon, 1+\epsilon) \hat{A}_t \right) \right]
    \]

    where \( r_t(\theta) \) is the probability ratio between the new and old policies, and \( \hat{A}_t \) is the advantage function.
    The critic loss is:

    \[
        L_{\text{critic}} = \mathbb{E}_t \left[ (V(s_t) - R_t)^2 \right]
    \]

    These updates are performed iteratively, using a neural network with backpropagation to minimize the loss and improve the policy.
    PPO ensures stability by limiting the policy update size.

    % diagrams/gpi.pdf
    \begin{figure}[htb]
        \centering
        \includegraphics[width=0.8\textwidth]{diagrams/gpi.drawio.pdf}
        \caption{Generalized Policy Iteration (GPI) diagram for Policy Gradient methods.}
        \label{fig:gpi}
    \end{figure}

    \begin{figure}[htb]
        \centering
        \includegraphics[width=0.8\textwidth]{diagrams/architecture.drawio.pdf}
        \caption{Architecture of the reinforcement learning agent used in the simulation.}
        \label{fig:agent}
    \end{figure}

\end{document}


\section{Technical Details and Framework Description}
\label{sec:technical_details}

%! Author = rzimmerdev
%! Date = 3/28/25

% Packages
% Document
\begin{document}
    \subsection{Parallel Environments for Asynchronous Reinforcement Learning (PEARL)}
    \label{subsec:pearl}
    The proposed architecture builds on the core idea of sharing a single, highly parallel environment among multiple agents while
    still allowing multiple environments to be run in parallel.
    In contrast to conventional distributed MARL systems—where each worker simulates an independent environment instance—
    the proposed shared environment unifies multiple independent simulations concurrently accessed by multiple workers
    (see~\fullimg{fig:pearl_architecture}).
    This design minimizes redundant computations and memory overhead,
    which is especially critical in high-frequency trading simulations where the fidelity and speed of the limit order book (LOB) dynamics are paramount.

    At the heart of the architecture lies a central service mesh that manages discovery of the available environments.
    Upon initialization, a learner can spawn as many workers processes as necessary,
    which in turn connect to discovered environments using a dealer/router pattern.
    In this setup, workers act as both data collectors and order executors:
    they send action vectors containing continuous order quotes to the LOB simulation,
    receive market updates, and record episodic trajectories that capture the sequence of states, actions, and rewards into a replay buffer.
    By sharing the environment, workers from different learners benefit from a consistent and synchronized view of the simulated market,
    which besides reducing the number of CPU time used to generate similar data to send to individual agents,
    also serves to enhance the quality of the observable experience data.

    Communication between the learner and workers is designed to be asynchronous.
    Workers gather their generated trajectories to the learner via the ZeroMQ asynchronous messaging layer,
    enabling workers to operate independently of each other—without waiting for unrelated environment steps—
    thereby reducing idle time and mitigating the straggler issues often encountered in distributed setups.
    The intrinsic details of the messaging pipeline is discussed further in~\fullref{subsec:communication}.

    % graphics for diagrams/pearl_architecture.pdf
    \begin{figure}[htb]
        \centering
        \includegraphics[width=0.8\textwidth]{diagrams/pearl.drawio.pdf}
        \caption{Overview of the Parallel Environments for Asynchronous Reinforcement Learning (PEARL) framework.}
        \label{fig:pearl_architecture}
    \end{figure}

    Meanwhile, the learner continuously processes incoming trajectories from the workers,
    and updates the global policy using the RL algorithm after the environments are all finished.
    This cycle ensures that while each worker may temporarily operate on slightly outdated policy parameters,
    the overall system converges steadily as fresh data are incorporated into the learner's updates.

    By allowing the workers to share a common environment and asynchronously gather trajectories,
    the proposed architecture aims to strike a balance between computational efficiency and scalability.
    The worker pool design also contributes to reducing the overall system latency,
    as workers can operate and unlock individual cores and sockets while waiting for the environment to synchronize other agents and return a new state.
    In traditional architectures, the cost of transferring model parameters and simulation data between isolated environment instances can introduce significant delays.
    By contrast, our approach tries to minimize such overhead and reduce communication delays for HFT applications by centralizing the learner weights.

    \begin{algorithm}
        \begin{algorithmic}[1]
            \Require Environments, Policy $\pi_{\theta}$, Rollout length $T$
            \State Initialize empty trajectory buffers for each environment

            \For{each environment $i$ in Environments}
                \State Reset environment and observe initial state $\mathbf{s}_i$
            \EndFor
            \For {each rollout step $t$ until rollout length $T$}
                \State Select actions $\mathbf{a} \sim \pi_{\theta}(\mathbf{s})$
                \State Send actions $\mathbf{a}$ to each corresponding environment asynchronously
                \State \textbf{async for} {each environment $i$ in Environments}
                \State $\quad$ Receive response $\mathbf{r}_i$ from environment $i$
                \State $\quad$ Store transition $(\mathbf{s}_i, \mathbf{a}_i, \mathbf{r}_i)$ in the trajectory buffer
                \State $\quad$ Set $\mathbf{s}_i \leftarrow \mathbf{s'}_i$
                \State $\quad$ \textbf{if} environment $i$ is done
                \State $\quad$$\quad$Remove environment $i$ from current rollout environments
                \State $\quad$ \textbf{end if}
                \State \textbf{end for}
            \EndFor
            \State $\textbf{Return}$ trajectory buffers
        \end{algorithmic}
        \caption{Asynchronous Trajectory Gathering}
        \label{alg:trajectory}
    \end{algorithm}

    Internally, the learner creates a worker pool before starting the training loop, and is then responsible
    for sending each step request asynchronously to each environment's router.
    The environment then waits for either all actions to be received or for a constant timeout interval to be reached, returning the results to the learner.
    Since all environments are executed in separate processes/distributed cores, the learner can continue to process the
    received trajectories while waiting for each asynchronous environment to respond,
    and thus the rollout speed is determined by the speed of the slowest environment.
    This becomes a significant bottleneck if the environments have different processing speeds,
    but due to the nature of the limit order book simulation, the processing time is, although large,
    still relatively consistent across all environments, and the fixed timeout rate also contributes to minimizing stragglers.
    A diagram of the trajectory the individual observations make before reaching the agents' replay buffers within the PEARL framework
    is shown in \fullimg{fig:pipeline}.

    \begin{figure}
        \centering
        \includegraphics[width=1\textwidth]{diagrams/pipeline.drawio.pdf}
        \caption{Illustration of the pipeline between learners, their workers and a single environment instance.}
        \label{fig:pipeline}
    \end{figure}

    In summary, this integrated system architecture—comprising a shared simulation environment, an asynchronous actor–learner framework, and
    a robust communication backbone—addresses many of the limitations inherent in earlier distributed reinforcement learning approaches.
    By combining the advantages of shared resource utilization with scalable, asynchronous updates, the architecture provides a solid foundation for
    developing reinforcement learning systems capable of operating effectively in complex, dynamic domains.
    We implemented both the independent process workers and shared process for learner/workers,
    and decided to use the shared process due to the increased efficiency of running the forward and backward batched passes
    for all environments being faster than the speed gain of running the rollout in parallel.
    That is, instead of sampling actions for each observation independently per worker,
    we gather the state buffers into the centralized learner and perform a batched forward pass on all state tensors.

    We compared the policy act times for both approaches in~\fullref{sec:results},
    and as expected, this did not affect the environment step times in any way.
    It did affect the policy speed, as sampling policy actions on individual environment states instead of
    centralizing the state variable and using a batched approach resulted in significantly longer update times\footnote{
        We attributed the difference in processing times to the overhead of sending and retrieving data from the GPU.
        Even though performing this step in a distributed fashion could increase processing speeds if more GPU computing power was used,
        discussions regarding distributed learners have already been extensively explored in works such as IMPALA and APE-X,
        and are beyond the scope of this thesis.
    }.
    This comparison was our main motivation behind centralizing the actor act method into a batched space tensor
    before performing the policy network forward pass.

\end{document}

%! Author = rzimmerdev
%! Date = 3/28/25

% Preamble
\documentclass[11pt]{article}

% Packages
\usepackage{amsmath}

% Document
\begin{document}
    \subsection{Communication Protocols and Network Layout}
    \label{subsec:communication}

    \begin{figure}[htb]
        \centering
        \includegraphics[width=0.5\textwidth]{diagrams/service_mesh}
        \caption{Overview of the service mesh used to register environments to be discovered by agents.}
        \label{fig:service_mesh}
    \end{figure}

    The communication infrastructure of the system is designed to support fully asynchronous interactions
    between distributed actors and the centralized learner module.
    The adopted service mesh architecture, shown in~\autoref{fig:service_mesh}, provides logical separation between communication and computation concerns,
    and allows the learner and environment subsystems to scale independently across compute nodes.

    As previously discussed, we employ a brokerless ZeroMQ Router/Dealer pattern for the messaging layer.
    This setup enables direct point-to-point communication between actors and the learner without requiring intermediate brokers or persistent queues,
    minimizing the round-trip time of trajectory and gradient message exchange.
    ZeroMQ's non-blocking I/O capabilities and lock-free shared memory buffers make it suitable for high-frequency,
    low-latency applications such as algorithmic trading environments.

    Each actor maintains an independent outbound socket pool for pushing trajectory segments or parameter update requests to the learner.
    This decouples action inference and experience collection from the learning pipeline,
    allowing learners to continuously receive and process incoming batches even in the presence of straggler workers.
    Moreover, actors are not synchronized globally; the system is designed so that each actor proceeds independently through environment interaction,
    buffering experience locally before asynchronously committing the collected data to the learner.

    \begin{figure}[htb]
        \centering
        \includegraphics[width=0.8\textwidth]{diagrams/communication}
        \caption{
            Communication flow between agents and the central learner using ZeroMQ messaging.\\
            Steps 1 and 2 show the initial setup phase, where newly created agents request existing environment services.\\
            Step 3 shows the agent sending an inital request to register itself in each individual environment's connected agent records.\\
            Finally, steps 4 and 5 show the main trajectory gathering loop, in which agents send their actions and
            record environment responses until the end-of-episode is reached and the environment is reset.
        }
        \label{fig:communication}
    \end{figure}

    \autoref{fig:communication} illustrates the end-to-end communication pipeline.
    Each agent is initialized with a direct messaging channel to the central learner process.
    The learner operates as a multiplexed ZeroMQ Router endpoint capable of concurrently handling multiple inbound streams.
    It tags incoming messages with agent identifiers and manages per-agent buffers for temporal ordering and batching of updates.
    This enables efficient mini-batch construction for training while preserving per-agent data locality when needed.

    The overall communication design maximizes concurrency while controlling the negative effects of different processing speeds per agent.
    Since the communication layer is built on top of a service mesh,
    it also supports horizontal scaling by adding agents without need for centralized coordination,
    as both learners and environments can be shutdown, restarted and added without interfering with training.
    This is a key design goal for large-scale reinforcement learning in latency-sensitive domains such as market making and high-frequency trading,
    where simulation fidelity, throughput, and data freshness all interact to determine the agent’s learning efficiency and performance.

\end{document}
