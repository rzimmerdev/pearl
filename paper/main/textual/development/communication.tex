%! Author = rzimmerdev
%! Date = 3/28/25

% Preamble
\documentclass[11pt]{article}

% Packages
\usepackage{amsmath}

% Document
\begin{document}
    \subsection{Communication Protocols and Network Layout}
    \label{subsec:communication}

    % diagram for diagrams/service_mesh.pdf
    % and diagrams/communication.pdf
    \begin{figure}[htb]
        \centering
        \includegraphics[width=0.5\textwidth]{diagrams/service_mesh}
        \caption{Overview of the service mesh architecture for communication between agents and the central learner.}
        \label{fig:service_mesh}
    \end{figure}

    The communication infrastructure of the system is designed to support the asynchronous interaction between agents and the central learner.

    The service mesh architecture, illustrated in \autoref{fig:service_mesh}, provides a scalable and fault-tolerant communication layer that connects the distributed components of the system.
    The communication between agents and the central learner is facilitated by a high-performance messaging layer implemented with ZeroMQ.

    \begin{figure}[htb]
        \centering
        \includegraphics[width=0.8\textwidth]{diagrams/communication}
        \caption{Communication flow between agents and the central learner using ZeroMQ messaging.}
        \label{fig:communication}
    \end{figure}

\end{document}