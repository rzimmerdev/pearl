%! Author = rzimmerdev
%! Date = 3/28/25

% Preamble
\documentclass[11pt]{article}

% Packages
\usepackage{amsmath}

% Document
\begin{document}
    \subsection{Communication Protocols and Network Layout}
    \label{subsec:communication}

    \begin{figure}[htb]
        \centering
        \includegraphics[width=0.5\textwidth]{diagrams/service_mesh.drawio.pdf}
        \caption{Overview of the service mesh used to register environments to be discovered by agents.}
        \label{fig:service_mesh}
    \end{figure}

    The communication infrastructure of the system is designed to support fully asynchronous interactions
    between distributed actors and the centralized learner module.
    The adopted service mesh architecture, shown in~\fullimg{fig:service_mesh}, 
    provides logical separation between communication and computation concerns,
    and allows the learner and environment subsystems to scale independently across compute nodes.

    As previously discussed, we employ a brokerless ZeroMQ Router/Dealer pattern for the messaging layer.
    This setup enables direct point-to-point communication between actors and the learner without requiring intermediate brokers or persistent queues,
    minimizing the round-trip time of trajectory and gradient message exchange.
    ZeroMQ's non-blocking I/O capabilities and lock-free shared memory buffers make it suitable for high-frequency,
    low-latency applications such as algorithmic trading environments.

    Each actor maintains an independent outbound socket pool for pushing trajectory segments or parameter update requests to the learner.
    This decouples action inference and experience collection from the learning pipeline,
    allowing learners to continuously receive and process incoming batches even in the presence of straggler workers.
    Moreover, actors are not synchronized globally; the system is designed so that each actor proceeds independently through environment interaction,
    buffering experience locally before asynchronously committing the collected data to the learner.

    \begin{figure}[htb]
        \centering
        \includegraphics[width=0.8\textwidth]{diagrams/communication.drawio.pdf}
        \caption{
            Communication flow between agents and the central learner using ZeroMQ messaging.\\
            Steps 1 and 2 show the initial setup phase, where newly created agents request existing environment services.\\
            Step 3 shows the agent sending an inital request to register itself in each individual environment's connected agent records.\\
            Finally, steps 4 and 5 show the main trajectory gathering loop, in which agents send their actions and
            record environment responses until the end-of-episode is reached and the environment is reset.
        }
        \label{fig:communication}
    \end{figure}

    \Fullimg{fig:communication} illustrates the end-to-end communication pipeline.
    Each agent is initialized with a direct messaging channel to the central learner process.
    The learner operates as a multiplexed ZeroMQ Router endpoint capable of concurrently handling multiple inbound streams.
    It tags incoming messages with agent identifiers and manages per-agent buffers for temporal ordering and batching of updates.
    This enables efficient mini-batch construction for training while preserving per-agent data locality when needed.

    The overall communication design maximizes concurrency while controlling the negative effects of different processing speeds per agent.
    Since the communication layer is built on top of a service mesh,
    it also supports horizontal scaling by adding agents without need for centralized coordination,
    as both learners and environments can be shutdown, restarted and added without interfering with training.
    This is a key design goal for large-scale reinforcement learning in latency-sensitive domains such as market making and high-frequency trading,
    where simulation fidelity, throughput, and data freshness all interact to determine the agent’s learning efficiency and performance.

\end{document}
