%% USPSC-TCC-pre-textual-ICMC.tex
%% Camandos para definição do tipo de documento (tese ou dissertação), área de concentração, opção, preâmbulo, titulação 
%% referentes ao Programa de Pós-Graduação o IFSC
\instituicao{Instituto de Ci\^encias Matem\'aticas e de Computa\c{c}\~ao, Universidade de S\~ao Paulo}
\unidade{INSTITUTO DE CI\^ENCIAS MATEM\'ATICAS E DE COMPUTA\c{C}\~AO}
\unidademin{Instituto de Ci\^encias Matem\'aticas e de Computa\c{c}\~ao}
\universidademin{Universidade de S\~ao Paulo}
\setorpos{SERVI\c{C}O DE GRADUA\c{C}\~AO DO ICMC-USP}

%\notafolharosto{Vers\~ao original}
%\notafolharostoadic{Original version}
%Para versão original em inglês, comente os comandos/declarações acima (inclua % antes do comando acima) 
% e tire a % dos comandos/declarações abaixo no idioma do texto
\notafolharosto{Original version}
\notafolharostoadic{Vers\~ao original}

%Para versão revisada, comente os comandos/declarações acima (inclua % antes do comando acima) 
% Se o Idioma do texto for português:
%\notafolharosto{Vers\~ao revisada}
%\notafolharosto{Final version}
% Se o Idioma do texto for Inglês: 
%\notafolharosto{Final version}
%\notafolharosto{Vers\~ao revisada}

% dados complementares para CAPA e FOLHA DE ROSTO
\universidade{UNIVERSIDADE DE S\~AO PAULO}

% Idioma do texto em PORTUGUÊS
%\titulo{Modelo para TCC em \LaTeX\ utilizando o Pacote USPSC para o ICMC}
%\titleabstract{Model for TCC in \LaTeX\ using the USPSC Package to the ICMC}
%\tituloadic{Model for TCC in \LaTeX\ using the USPSC Package to the ICMC}
%\tituloresumo{Modelo para TCC em \LaTeX\ utilizando o Pacote USPSC para o ICMC}

% Idioma do texto em INGLÊS
\titulo{A Distributed Framework for Multi-Agent Reinforcement Learning in High-Frequency Trading}
\titleabstract{A Distributed Framework for Multi-Agent Reinforcement Learning in High-Frequency Trading}
\tituloadic{Uma Arquitetura Distribuída para Aprendizado por Reforço Multiagente em Trading de Alta Frequência}
\tituloresumo{Uma Arquitetura Distribuída para Aprendizado por Reforço Multiagente em Trading de Alta Frequência}

\autor{Rafael Zimmer}
\autorficha{Zimmer, Rafael}
\autorabr{ZIMMER, R.}

\cutter{S856m}
% Para gerar a ficha catalográfica sem o Código Cutter
%\cutter{ }

\local{São Paulo}
\data{2025}

% Para o idioma português:
%\renewcommand{\orientadorname}{Orientadora:}
%\orientador{Profa. Dra. Elisa Gon\c{c}alves Rodrigues}
%\orientadoradic{Advisor: Profa. Dra. Elisa Gon\c{c}alves Rodrigues}
%\orientadorcorpoficha{orientadora Elisa Gon\c{c}alves Rodrigues}
%\orientadorficha{Rodrigues, Elisa Gon\c{c}alves, orient}

%Para incluir o nome do(a) coorientados(a), inclua % nos 2 comandos acima e retire a % dos 2 comandos abaixo
%\orientadorcorpoficha{orientadora Elisa Gon\c{c}alves Rodrigues ;  co-orientador Jo\~ao Alves Serqueira}
%\orientadorficha{Rodrigues, Elisa Gon\c{c}alves, orient. II. Serqueira, Jo\~ao Alves, co-orient}


%Se o idioma for o inglês, inclua % nos comandos acima e exclua dos comandos abaixo
\renewcommand{\orientadorname}{Advisor:}
\orientador{Prof. Dr. Oswaldo Luiz do Valle Costa}
\orientadoradic{Orientadora: Prof. Dr. Oswaldo Luiz do Valle Costa}
\orientadorcorpoficha{orientadora Oswaldo Luiz do Valle Costa}
\orientadorficha{Costa, Oswaldo Luiz do Valle, orient}

% Quando houver Coorientador(a): 
% Para o idioma português:
%\newcommand{\coorientadorname}{Coorientador:}
%\newcommand{\coorientadorname}{Coorientadora:}
% Para o idoma inglês:
%\newcommand{\coorientadorname}{Coorientador:}

% Quando houver Coorientador(a), basta tirar a % utilizar o comando abaixo
%\newcommand{\coorientadorname}{Coadvisor:}
%Se houver co-orientador, inclua % antes das duas linhas (antes dos comandos \orientadorcorpoficha e \orientadorficha) 
%          e tire a % antes dos 3 comandos abaixo
%\coorientador{Prof. Dr. Jo\~ao Alves Serqueira}
%\coorientadoradic{ Co-orientador: Prof. Dr. Jo\~ao Alves Serqueira}
%\orientadorcorpoficha{orientadora Elisa Gon\c{c}alves Rodrigues ;  co-orientador Jo\~ao Alves Serqueira}
%\orientadorficha{Rodrigues, Elisa Gon\c{c}alves, orient. II. Serqueira, Jo\~ao Alves, co-orient}

%Para o idioma Inglês, retire a % antes da linha abaixo
\renewcommand{\areaname}{Concentration area: }

%\notaautorizacao{AUTORIZO A REPRODU\c{C}\~AO E DIVULGA\c{C}\~AO TOTAL OU PARCIAL DESTE TRABALHO, POR QUALQUER MEIO CONVENCIONAL OU ELETR\^ONICO PARA FINS DE ESTUDO E PESQUISA, DESDE QUE CITADA A FONTE.}
% Se o idioma for o inglês, inclua a % antes do campo \notaautorizacao acima e retire a % da linha abaixo
\notaautorizacao{I AUTORIZE THE REPRODUCTION AND DISSEMINATION OF TOTAL OR PARTIAL COPIES OF THIS DOCUMENT, BY CONVENCIONAL OR ELECTRONIC MEDIA FOR STUDY OR RESEARCH PURPOSE, SINCE IT IS REFERENCED.}

\notabib{Catalogue file prepared by the Library Prof. Achille Bassi, ICMC/USP, with data provided by the author}

\newcommand{\programa}[1]{


% BCCe ==========================================================================
\ifthenelse{\equal{#1}{BCCe}}{
	\renewcommand{\areaname}{Concentration area:}
    \tipotrabalho{Monografia (Trabalho de Conclus\~ao de Curso)}
    \tipotrabalhoabs{Monograph (Conclusion Course Paper)}
    \renewcommand{\orientadorname}{Advisor:}
    %Para Orientadora, inclua % antes do comando acima e retire a % antes do comando abaixo
    %\renewcommand{\orientadorname}{Orientadora:}
    % Quando houver Coorientador, basta tirar a % utilizar o comando abaixo
   	%\newcommand{\coorientadorname}{Coorientador:}
    \renewcommand{\areaname}{Concentration area: }
    \area{Computer Science and Computational Mathematics}
    \areaadic{\'Area de concentra\c{c}\~ao: Ci\^encias de Computa\c{c}\~ao e Matem\'atica Computacional}
    %\opcao{Nome da Opção em inglês}
    %\opcaoadic{Nome da Opção em português}
    % O preambulo deve conter o tipo do trabalho, o objetivo, 
    % o nome da instituição, a área de concentração e opção quando houver
    % O preambulo deve conter o tipo do trabalho, o objetivo, 
	% o nome da instituição, a área de concentração e opção quando houver
	\preambulo{Course Conclusion thesis submitted to the Undergraduate Program of the Institute of Mathematics and Computer Sciences, University of S\~ao Paulo - ICMC/USP, in partial fulfillment of the  requirements for the degree of the Bachelor in Computer Science.}
	\preambuloadic{Trabalho de conclus\~ao de curso apresentado ao Programa de Gradua\c{c}\~ao, do Instituto de Ci\^encias Matem\'aticas e de Computa\c{c}\~ao, Universidade de S\~ao Paulo - ICMC/USP, como parte dos requisitos para obten\c{c}\~ao do t\'itulo de Bacharel em Ci\^encias de Computa\c{c}\~ao.}
	\notaficha{Monograph (Undergraduate in Computer Science)}
	\notacapaicmc{Course Conclusion Thesis to the Undergraduate Program Bachelor's in\\ Computer Sciences}
    }{
 % BCCp ==========================================================================
 \ifthenelse{\equal{#1}{BCCp}}{
    \tipotrabalho{Monografia (Trabalho de Conclus\~ao de Curso)}
    \tipotrabalhoabs{Monograph (Conclusion Course Paper)}
    % Quando for Orientador, basta incluir uma % antes do comando abaixo
    \renewcommand{\orientadorname}{Orientadora:}
    % Quando for Coorientadora, basta tirar a % utilizar o comando abaixo
    %\newcommand{\coorientadorname}{Coorientador:}
    \area{Ci\^encias de Computa\c{c}\~ao e Matem\'atica Computacional}
    \areaadic{Concentration area: Computer Science and Computational Mathematics}
    %\opcao{Nome da Opção em português}
    %\opcaoadic{Nome da Opção em inglês}
    % O preambulo deve conter o tipo do trabalho, o objetivo, 
    % o nome da instituição, a área de concentração e opção quando houver
    % O preambulo deve conter o tipo do trabalho, o objetivo, 
    % o nome da instituição, a área de concentração e opção quando houver
    \preambulo{Trabalho de conclus\~ao de curso apresentado ao Programa de Gradua\c{c}\~ao, do Instituto de Ci\^encias Matem\'aticas e de Computa\c{c}\~ao, Universidade de S\~ao Paulo - ICMC/USP, como parte dos requisitos para obten\c{c}\~ao do t\'itulo de Bacharel em Ci\^encias de Computa\c{c}\~ao.}
    \preambuloadic{Conclusion course paper submitted to the Undergraduate Program of the Instituto de Ci\^encias Matem\'aticas e de Computa\c{c}\~ao, Universidade de S\~ao Paulo - ICMC/USP, in partial fulfillment of the  requirements for the degree of the Bachelor in Computer Science.}
    \notaficha{Monografia (Gradua\c{c}\~ao em Ci\^encias de Computa\c{c}\~ao)}
    \notacapaicmc{Trabalho de Conclus\~ao de Curso do Programa de Gradua\c{c}\~ao Bacharelado em\\ Ci\^encias de Computa\c{c}\~ao}
    }{
% BMe ==========================================================================
\ifthenelse{\equal{#1}{BMe}}{
	\renewcommand{\areaname}{Concentration area:}
	\tipotrabalho{Monografia (Trabalho de Conclus\~ao de Curso)}
	\tipotrabalhoabs{Monograph (Conclusion Course Paper)}
	%\renewcommand{\orientadorname}{Advisor:}
	%Para Orientadora, inclua % antes do comando acima e retire a % antes do comando abaixo
	%\renewcommand{\orientadorname}{Orientadora:}
	% Quando houver Coorientador, basta tirar a % utilizar o comando abaixo
	%\newcommand{\coorientadorname}{Coorientador:}
	\renewcommand{\areaname}{Concentration area: }
	\area{Mathematics}
	\areaadic{\'Area de concentra\c{c}\~ao: Matem\'atica}
	%\opcao{Nome da Opção em inglês}
	%\opcaoadic{Nome da Opção em português}
	% O preambulo deve conter o tipo do trabalho, o objetivo, 
	% o nome da instituição, a área de concentração e opção quando houver
	% O preambulo deve conter o tipo do trabalho, o objetivo, 
	% o nome da instituição, a área de concentração e opção quando houver
	\preambulo{Conclusion course  paper presented to the Undergraduate Program of the Instituto de Ci\^encias Matem\'aticas e de Computa\c{c}\~ao, Universidade de S\~ao Paulo - ICMC/USP, in partial fulfillment of the  requirements for the degree of the Bachelor in Mathematics.}
	\preambuloadic{Trabalho de conclus\~ao de curso apresentado ao Programa de Gradua\c{c}\~ao do Instituto de Ci\^encias Matem\'aticas e de Computa\c{c}\~ao, Universidade de S\~ao Paulo - ICMC/USP, como parte dos requisitos para obten\c{c}\~ao do t\'itulo de Bacharel em Matem\'atica.}	
	\notaficha{Monograph (Undergraduate in Mathematics)}
	\notacapaicmc{Conclusion Course Paper to the Undergraduate Program Bachelor's in\\ Mathematics}
    }{
% BMp ==========================================================================
\ifthenelse{\equal{#1}{BMp}}{
    \tipotrabalho{Monografia (Trabalho de Conclus\~ao de Curso)}
    \tipotrabalhoabs{Monograph (Conclusion Course Paper)}
    \area{Matem\'atica}
	\areaadic{Concentration area: Mathematics}
	%\opcao{Nome da Opção em português}
	%\opcaoadic{Nome da Opção em inglês}
	% O preambulo deve conter o tipo do trabalho, o objetivo, 
	% o nome da instituição, a área de concentração e opção quando houver
	% O preambulo deve conter o tipo do trabalho, o objetivo, 
	% o nome da instituição, a área de concentração e opção quando houver
	\preambulo{Trabalho de conclus\~ao de curso apresentado ao Programa de Gradua\c{c}\~ao do Instituto de Ci\^encias Matem\'aticas e de Computa\c{c}\~ao, Universidade de S\~ao Paulo - ICMC/USP, como parte dos requisitos para obten\c{c}\~ao do t\'itulo de Bacharel em Matem\'atica.}	
	\preambuloadic{Conclusion course  paper presented to the Undergraduate Program of the Instituto de Ci\^encias Matem\'aticas e de Computa\c{c}\~ao, Universidade de S\~ao Paulo - ICMC/USP, in partial fulfillment of the  requirements for the degree of the Bachelor in Mathematics.}
	\notaficha{Monografia (Gradua\c{c}\~ao em Matem\'atica)}
	\notacapaicmc{Trabalho de Conclus\~ao de Curso do Programa de Gradua\c{c}\~ao Bacharelado em\\ Matem\'atica}
    }{
% BMAe ==========================================================================
\ifthenelse{\equal{#1}{BMAe}}{
	\renewcommand{\areaname}{Concentration area:}
	\tipotrabalho{Monografia (Trabalho de Conclus\~ao de Curso)}
	\tipotrabalhoabs{Monograph (Conclusion Course Paper)}
	\area{Applied Mathematics and Scientific Computing}
	\areaadic{\'Area de concentra\c{c}\~ao: Matem\'atica Aplicada e Computa\c{c}\~ao Cient\'ifica}
	%\opcao{Nome da Opção em inglês}
	%\opcaoadic{Nome da Opção em português}
	% O preambulo deve conter o tipo do trabalho, o objetivo, 
	% o nome da instituição, a área de concentração e opção quando houver
	% O preambulo deve conter o tipo do trabalho, o objetivo, 
	% o nome da instituição, a área de concentração e opção quando houver
	\preambulo{Conclusion course paper presented to the Undergraduate Program of the Instituto de Ci\^encias Matem\'aticas e de Computa\c{c}\~ao, Universidade de S\~ao Paulo - ICMC/USP, in partial fulfillment of the requirements for the degree of the Bachelor in Applied Mathematics and Scientific Computing.}
	\preambuloadic{Trabalho de conclus\~ao de curso apresentado ao Programa de Gradua\c{c}\~ao, do Instituto de Ci\^encias Matem\'aticas e de Computa\c{c}\~ao, Universidade de S\~ao Paulo - ICMC/USP, como parte dos requisitos para obten\c{c}\~ao do t\'itulo de Bacharel em Matem\'atica Aplicada e Computa\c{c}\~ao Cient\'ifica.}	
	\notaficha{Monograph (Undergraduate in Applied Mathematics and Scientific Computing)}
	\notacapaicmc{Conclusion Course Paper to the Undergraduate Program Bachelor's in\\ Applied Mathematics and Scientific Computing}
    }{
% BMAp ==========================================================================
\ifthenelse{\equal{#1}{BMAp}}{
	\tipotrabalho{Monografia (Trabalho de Conclus\~ao de Curso)}
	\tipotrabalhoabs{Monograph (Conclusion Course Paper)}
	\area{Matem\'atica Aplicada e Computa\c{c}\~ao Cient\'ifica}
	\areaadic{Concentration area: Applied Mathematics and Scientific Computing}
	%\opcao{Nome da Opção em português}
	%\opcaoadic{Nome da Opção em inglês}
	% O preambulo deve conter o tipo do trabalho, o objetivo, 
	% o nome da instituição, a área de concentração e opção quando houver
	% O preambulo deve conter o tipo do trabalho, o objetivo, 
	% o nome da instituição, a área de concentração e opção quando houver
	\preambulo{Trabalho de conclus\~ao de curso apresentado ao Programa de Gradua\c{c}\~ao, do Instituto de Ci\^encias Matem\'aticas e de Computa\c{c}\~ao, Universidade de S\~ao Paulo - ICMC/USP, como parte dos requisitos para obten\c{c}\~ao do t\'itulo de Bacharel em Matem\'atica Aplicada e Computa\c{c}\~ao Cient\'ifica.}	
	\preambuloadic{Conclusion course paper presented to the Undergraduate Program of the Instituto de Ci\^encias Matem\'aticas e de Computa\c{c}\~ao, Universidade de S\~ao Paulo - ICMC/USP, in partial fulfillment of the  requirements for the degree of the Bachelor in Applied Mathematics and Scientific Computing.}
	\notaficha{Monografia (Gradua\c{c}\~ao em Matem\'atica Aplicada e Computa\c{c}\~ao Cient\'ifica)}
	\notacapaicmc{Trabalho de Conclus\~ao de Curso do Programa de Gradua\c{c}\~ao Bacharelado em\\ Matem\'atica Aplicada e Computa\c{c}\~ao Cient\'ifica}
    }{
% LMe ==========================================================================
\ifthenelse{\equal{#1}{LMe}}{
	\renewcommand{\areaname}{Concentration area:}
	\tipotrabalho{Monografia (Trabalho de Conclus\~ao de Curso)}
	\tipotrabalhoabs{Monograph (Conclusion Course Paper)}
	\area{Mathematics}
	\areaadic{\'Area de concentra\c{c}\~ao: Matem\'atica}
	%\opcao{Nome da Opção em inglês}
	%\opcaoadic{Nome da Opção em português}
	% O preambulo deve conter o tipo do trabalho, o objetivo, 
	% o nome da instituição, a área de concentração e opção quando houver
	% O preambulo deve conter o tipo do trabalho, o objetivo, 
	% o nome da instituição, a área de concentração e opção quando houver
	\preambulo{Conclusion course paper presented to the Undergraduate Program of the Instituto de Ci\^encias Matem\'aticas e de Computa\c{c}\~ao, Universidade de S\~ao Paulo - ICMC/USP, in partial fulfillment of the  requirements for the degree of the Licentiate in Mathematics.}
	\preambuloadic{Trabalho de conclus\~ao de curso apresentado ao Programa de Gradua\c{c}\~ao, do Instituto de Ci\^encias Matem\'aticas e de Computa\c{c}\~ao, Universidade de S\~ao Paulo - ICMC/USP, como parte dos requisitos para obten\c{c}\~ao do t\'itulo de Licenciado em Matem\'atica.}	
	\notaficha{Monograph (Degree in Mathematics)}
	\notacapaicmc{Conclusion Course Paper to the Undergraduate Program Licenciate in\\ Mathematics}
    }{
% LMp ==========================================================================
\ifthenelse{\equal{#1}{LMp}}{
	\tipotrabalho{Monografia (Trabalho de Conclus\~ao de Curso)}
	\tipotrabalhoabs{Monograph (Conclusion Course Paper)}
	\area{Matem\'atica}
	\areaadic{Concentration area: Mathematics}
	%\opcao{Nome da Opção em português}
	%\opcaoadic{Nome da Opção em inglês}
	% O preambulo deve conter o tipo do trabalho, o objetivo, 
	% o nome da instituição, a área de concentração e opção quando houver
	% O preambulo deve conter o tipo do trabalho, o objetivo, 
	% o nome da instituição, a área de concentração e opção quando houver
	\preambulo{Trabalho de conclus\~ao de curso apresentado ao Programa de Gradua\c{c}\~ao, do Instituto de Ci\^encias Matem\'aticas e de Computa\c{c}\~ao, Universidade de S\~ao Paulo - ICMC/USP, como parte dos requisitos para obten\c{c}\~ao do t\'itulo de Licenciado em Matem\'atica.}	
	\preambuloadic{Conclusion course paper presented to the Undergraduate Program of the Instituto de Ci\^encias Matem\'aticas e de Computa\c{c}\~ao, Universidade de S\~ao Paulo - ICMC/USP, in partial fulfillment of the  requirements for the degree of the Licentiate in Mathematics.}
	\notaficha{Monografia (Licenciatura em Matem\'atica)}
	\notacapaicmc{Trabalho de Conclus\~ao de Curso do Programa de Gradua\c{c}\~ao Licenciatura em\\ Matem\'atica}
    }{
% BCDe ==========================================================================
\ifthenelse{\equal{#1}{BCDe}}{
	\renewcommand{\areaname}{Concentration area:}
	\tipotrabalho{Monografia (Trabalho de Conclus\~ao de Curso)}
	\tipotrabalhoabs{Monograph (Conclusion Course Paper)}
	\area{Data Science}
	\areaadic{\'Area de concentra\c{c}\~ao: Ci\^encia de Dados}
	%\opcao{Nome da Opção em inglês}
	%\opcaoadic{Nome da Opção em português}
	% O preambulo deve conter o tipo do trabalho, o objetivo, 
	% o nome da instituição, a área de concentração e opção quando houver
	% O preambulo deve conter o tipo do trabalho, o objetivo, 
	% o nome da instituição, a área de concentração e opção quando houver
	\preambulo{Conclusion course paper presented to the Undergraduate Program of the Instituto de Ci\^encias Matem\'aticas e de Computa\c{c}\~ao, Universidade de S\~ao Paulo - ICMC/USP, in partial fulfillment of the  requirements for the degree of the Bachelor in Data Science.}
	\preambuloadic{Trabalho de conclus\~ao de curso apresentado ao Programa de Gradua\c{c}\~ao do Instituto de Ci\^encias Matem\'aticas e de Computa\c{c}\~ao, Universidade de S\~ao Paulo - ICMC/USP, como parte dos requisitos para obten\c{c}\~ao do t\'itulo de Bacharel em Ci\^encia de Dados.}	
	\notaficha{Monograph (Undergraduate in Statistics and Data Science)}
	\notacapaicmc{Conclusion Course Paper to the Undergraduate Program Bachelor's in\\ Data Science}
    }{
% BCDp ==========================================================================
\ifthenelse{\equal{#1}{BCDp}}{
    \tipotrabalho{Monografia (Trabalho de Conclus\~ao de Curso)}
    \tipotrabalhoabs{Monograph (Conclusion Course Paper)}
	\area{Ci\^encia de Dados}
	\areaadic{Concentration area: Data Science}
	%\opcao{Nome da Opção em português}
	%\opcaoadic{Nome da Opção em inglês}
	% O preambulo deve conter o tipo do trabalho, o objetivo, 
	% o nome da instituição, a área de concentração e opção quando houver
	% O preambulo deve conter o tipo do trabalho, o objetivo, 
	% o nome da instituição, a área de concentração e opção quando houver
	\preambulo{Trabalho de conclus\~ao de curso apresentado ao Programa de Gradua\c{c}\~ao do Instituto de Ci\^encias Matem\'aticas e de Computa\c{c}\~ao, Universidade de S\~ao Paulo - ICMC/USP, como parte dos requisitos para obten\c{c}\~ao do t\'itulo de Bacharel em Ci\^encia de Dados.}	
	\preambuloadic{Conclusion course paper presented to the Undergraduate Program of the Instituto de Ci\^encias Matem\'aticas e de Computa\c{c}\~ao, Universidade de S\~ao Paulo - ICMC/USP, in partial fulfillment of the requirements for the degree of the Bachelor in Data Science.}
	\notaficha{Monografia (Gradua\c{c}\~ao em Ci\^encia de Dados)}
	\notacapaicmc{Trabalho de Conclus\~ao de Curso do Programa de Gradua\c{c}\~ao Bacharelado em\\ Ci\^encia de Dados}
    }{
% EBECDe ==========================================================================
\ifthenelse{\equal{#1}{EBECDe}}{
	\renewcommand{\areaname}{Concentration area:}
	\tipotrabalho{Projeto de gradua\c{c}\~ao (Trabalho de Conclus\~ao de Curso)}
	\tipotrabalhoabs{Graduation project (Conclusion Course Paper)}
	\area{Statistics and Data Science}
	\areaadic{\'Area de concentra\c{c}\~ao: Estat\'istica e Ci\^encia de Dados}
	%\opcao{Nome da Opção em inglês}
	%\opcaoadic{Nome da Opção em português}
	% O preambulo deve conter o tipo do trabalho, o objetivo, 
	% o nome da instituição, a área de concentração e opção quando houver
	% O preambulo deve conter o tipo do trabalho, o objetivo, 
	% o nome da instituição, a área de concentração e opção quando houver
	\preambulo{Graduation project presented to the Undergraduate Program of the Instituto de Ci\^encias Matem\'aticas e de Computa\c{c}\~ao, Universidade de S\~ao Paulo - ICMC/USP, in partial fulfillment of the requirements for the degree of the Bachelor in Statistics and Data Science.}
	\preambuloadic{Projeto de gradua\c{c}\~ao apresentado ao Programa de Gradua\c{c}\~ao do Instituto de Ci\^encias Matem\'aticas e de Computa\c{c}\~ao, Universidade de S\~ao Paulo - ICMC/USP, como parte dos requisitos para obten\c{c}\~ao do t\'itulo de Bacharel em Estat\'istica e Ci\^encia de Dados.}
	\notaficha{Graduation Project (Undergraduate in Statistics and Data Science)}
	\notacapaicmc{Graduation Project to the Undergraduate Program Bachelor's in\\ Statistics and Data Science}	
    }{
% EBECDp  ==========================================================================
\ifthenelse{\equal{#1}{EBECDp}}{
	\tipotrabalho{Projeto de gradua\c{c}\~ao (Trabalho de Conclus\~ao de Curso)}
	\tipotrabalhoabs{Graduation project (Conclusion Course Paper)}
	\area{Estat\'istica e Ci\^encia de Dados}
	\areaadic{Concentration area: Statistics and Data Science}
	%\opcao{Nome da Opção em português}
	%\opcaoadic{Nome da Opção em inglês}
	% O preambulo deve conter o tipo do trabalho, o objetivo, 
	% o nome da instituição, a área de concentração e opção quando houver
	% O preambulo deve conter o tipo do trabalho, o objetivo, 
	% o nome da instituição, a área de concentração e opção quando houver
	\preambulo{Projeto de gradua\c{c}\~ao apresentado ao Programa de Gradua\c{c}\~ao do Instituto de Ci\^encias Matem\'aticas e de Computa\c{c}\~ao, Universidade de S\~ao Paulo - ICMC/USP, como parte dos requisitos para obten\c{c}\~ao do t\'itulo de Bacharel em Estat\'istica e Ci\^encia de Dados.}	
	\preambuloadic{Graduation project presented to the Undergraduate Program of the Instituto de Ci\^encias Matem\'aticas e de Computa\c{c}\~ao, Universidade de S\~ao Paulo - ICMC/USP, in partial fulfillment of the requirements for the degree of the Bachelor in Statistics and Data Science.}
	\notaficha{Projeto de gradua\c{c}\~ao (Gradua\c{c}\~ao em Estat\'istica e Ci\^encia de Dados)}
	\notacapaicmc{Projeto de gradua\c{c}\~ao do Programa de Gradua\c{c}\~ao Bacharelado em\\ Estat\'istica e Ci\^encia de Dados}
    }{
% BECDe ==========================================================================
\ifthenelse{\equal{#1}{BECDe}}{
	\renewcommand{\areaname}{Concentration area:}
	\tipotrabalho{Monografia (Trabalho de Conclus\~ao de Curso)}
	\tipotrabalhoabs{Monograph (Conclusion Course Paper)}
	\area{Statistics and Data Science}
	\areaadic{\'Area de concentra\c{c}\~ao: Estat\'istica  e Ci\^encia de Dados}
	%\opcao{Nome da Opção em inglês}
	%\opcaoadic{Nome da Opção em português}
	% O preambulo deve conter o tipo do trabalho, o objetivo, 
	% o nome da instituição, a área de concentração e opção quando houver
	% O preambulo deve conter o tipo do trabalho, o objetivo, 
	% o nome da instituição, a área de concentração e opção quando houver
	\preambulo{Conclusion course paper presented to the Undergraduate Program of the Instituto de Ci\^encias Matem\'aticas e de Computa\c{c}\~ao, Universidade de S\~ao Paulo - ICMC/USP, in partial fulfillment of the requirements for the degree of the Bachelor in Statistics and Data Science.}
	\preambuloadic{Trabalho de conclus\~ao de curso apresentado ao Programa de Gradua\c{c}\~ao do Instituto de Ci\^encias Matem\'aticas e de Computa\c{c}\~ao, Universidade de S\~ao Paulo - ICMC/USP, como parte dos requisitos para obten\c{c}\~ao do t\'itulo de Bacharel em  Estat\'istica  e Ci\^encia de Dados.}	
	\notaficha{Monograph (Undergraduate in Statistics and Data Science)}
	\notacapaicmc{Conclusion Course Paper to the Undergraduate Program Bachelor's in\\ Statistics and Data Science}
    }{
% BECDp ==========================================================================
\ifthenelse{\equal{#1}{BECDp}}{
	\tipotrabalho{Monografia (Trabalho de Conclus\~ao de Curso)}
	\tipotrabalhoabs{Monograph (Conclusion Course Paper)}
	\area{Estat\'istica  e Ci\^encia de Dados}
	\areaadic{Concentration area: Statistics and Data Science}
	%\opcao{Nome da Opção em português}
	%\opcaoadic{Nome da Opção em inglês}
	% O preambulo deve conter o tipo do trabalho, o objetivo, 
	% o nome da instituição, a área de concentração e opção quando houver
	% O preambulo deve conter o tipo do trabalho, o objetivo, 
	% o nome da instituição, a área de concentração e opção quando houver
	\preambulo{Trabalho de conclus\~ao de curso apresentado ao Programa de Gradua\c{c}\~ao do Instituto de Ci\^encias Matem\'aticas e de Computa\c{c}\~ao, Universidade de S\~ao Paulo - ICMC/USP, como parte dos requisitos para obten\c{c}\~ao do t\'itulo de Bacharel em  Estat\'istica  e Ci\^encia de Dados.}	
	\preambuloadic{Conclusion course paper presented to the Undergraduate Program of the Instituto de Ci\^encias Matem\'aticas e de Computa\c{c}\~ao, Universidade de S\~ao Paulo - ICMC/USP, in partial fulfillment of the requirements for the degree of the Bachelor in Statistics and Data Science.}
	\notaficha{Monografia (Gradua\c{c}\~ao em Estat\'istica e Ci\^encia de Dados)}
	\notacapaicmc{Trabalho de Conclus\~ao de Curso do Programa de Gradua\c{c}\~ao Bacharelado em\\ Estat\'istica e Ci\^encia de Dados}
    }{
% BSIe ==========================================================================
\ifthenelse{\equal{#1}{BSIe}}{
	\renewcommand{\areaname}{Concentration area:}
	\tipotrabalho{Monografia (Trabalho de Conclus\~ao de Curso)}
	\tipotrabalhoabs{Monograph (Conclusion Course Paper)}
	\area{Information Systems}
	\areaadic{\'Area de concentra\c{c}\~ao: Sistemas de Informa\c{c}\~ao}
	%\opcao{Nome da Opção em inglês}
	%\opcaoadic{Nome da Opção em português}
	% O preambulo deve conter o tipo do trabalho, o objetivo, 
	% o nome da instituição, a área de concentração e opção quando houver
	% O preambulo deve conter o tipo do trabalho, o objetivo, 
	% o nome da instituição, a área de concentração e opção quando houver
	\preambulo{Conclusion course paper presented to the Undergraduate Program of the Instituto de Ci\^encias Matem\'aticas e de Computa\c{c}\~ao, Universidade de S\~ao Paulo - ICMC/USP, in partial fulfillment of the requirements for the degree of the Bachelor in Information Systems.}
	\preambuloadic{Trabalho de conclus\~ao de curso apresentado ao Programa de Gradua\c{c}\~ao do Instituto de Ci\^encias Matem\'aticas e de Computa\c{c}\~ao, Universidade de S\~ao Paulo - ICMC/USP, como parte dos requisitos para obten\c{c}\~ao do t\'itulo de Bacharel em Sistemas de Informa\c{c}\~ao.}	
	\notaficha{Monograph (Degree in Information Systems)}
	\notacapaicmc{Conclusion Course Paper to the Undergraduate Program Bachelor's in\\ Information Systems}	
    }{
% BSIp ==========================================================================
\ifthenelse{\equal{#1}{BSIp}}{
	\tipotrabalho{Monografia (Trabalho de Conclus\~ao de Curso)}
	\tipotrabalhoabs{Monograph (Conclusion Course Paper)}
	\area{Sistemas de Informa\c{c}\~ao}
	\areaadic{Concentration area: Information Systems}
	%\opcao{Nome da Opção em português}
	%\opcaoadic{Nome da Opção em inglês}
	% O preambulo deve conter o tipo do trabalho, o objetivo, 
	% o nome da instituição, a área de concentração e opção quando houver
	% O preambulo deve conter o tipo do trabalho, o objetivo, 
	% o nome da instituição, a área de concentração e opção quando houver
	\preambulo{Trabalho de conclus\~ao de curso apresentado ao Programa de Gradua\c{c}\~ao do Instituto de Ci\^encias Matem\'aticas e de Computa\c{c}\~ao, Universidade de S\~ao Paulo - ICMC/USP, como parte dos requisitos para obten\c{c}\~ao do t\'itulo de Bacharel em Sistemas de Informa\c{c}\~ao.}	
	\preambuloadic{Conclusion course paper presented to the Undergraduate Program of the Instituto de Ci\^encias Matem\'aticas e de Computa\c{c}\~ao, Universidade de S\~ao Paulo - ICMC/USP, in partial fulfillment of the requirements for the degree of the Bachelor in Information Systems.}
	\notaficha{Monografia (Gradua\c{c}\~ao em Sistemas de Informa\c{c}\~ao}
	\notacapaicmc{Trabalho de Conclus\~ao de Curso do Programa de Gradua\c{c}\~ao Bacharelado em\\ Sistemas de Informa\c{c}\~ao}
    }{
% ECe ==========================================================================
\ifthenelse{\equal{#1}{ECe}}{
	\renewcommand{\areaname}{Concentration area:}
	\tipotrabalho{Monografia (Trabalho de Conclus\~ao de Curso)}
	\tipotrabalhoabs{Monograph (Conclusion Course Paper)}
	\area{Computer Engineering}
	\areaadic{\'Area de concentra\c{c}\~ao: Engenharia de Computa\c{c}\~ao}
	\instituicao{Instituto de Ci\^encias Matem\'aticas e de Computa\c{c}\~ao, Universidade de S\~ao Paulo; Escola de Engenharia de S\~ao Carlos, Universidade de S\~ao Paulo}	
	%\opcao{Nome da Opção em inglês}
	%\opcaoadic{Nome da Opção em português}
	% O preambulo deve conter o tipo do trabalho, o objetivo, 
	% o nome da instituição, a área de concentração e opção quando houver
	% O preambulo deve conter o tipo do trabalho, o objetivo, 
	% o nome da instituição, a área de concentração e opção quando houver
	\preambulo{Conclusion course paper presented to the Undergraduate Program of the Instituto de Ci\^encias Matem\'aticas e de Computa\c{c}\~ao, and the Escola de Engenharia de S\~ao Carlos, Universidade de S\~ao Paulo, in partial fulfillment of the requirements for the degree of the Computer Engineer.}
	\preambuloadic{Trabalho de conclus\~ao de curso apresentado ao Programa de Gradua\c{c}\~ao do Instituto de Ci\^encias Matem\'aticas e de Computa\c{c}\~ao e Escola de Engenharia de S\~ao Carlos, Universidade de S\~ao Paulo - ICMC - EESC/USP, como parte dos requisitos para obten\c{c}\~ao do t\'itulo de Engenheiro de Computa\c{c}\~ao.}
	\notaficha{Monograph (Degree in Computer Engineering)}
	\notacapaicmc{Conclusion Course Paper to the Undergraduate Program Bachelor's in\\ Computer Engineering}	
    }{
% ECp ==========================================================================
\ifthenelse{\equal{#1}{ECp}}{
	\tipotrabalho{Monografia (Trabalho de Conclus\~ao de Curso)}
	\tipotrabalhoabs{Monograph (Conclusion Course Paper)}
	\area{Engenharia de Computa\c{c}\~ao}
	\areaadic{Concentration area: Computer Engineering}
	\instituicao{Instituto de Ci\^encias Matem\'aticas e de Computa\c{c}\~ao, Universidade de S\~ao Paulo; Escola de Engenharia de S\~ao Carlos, Universidade de S\~ao Paulo}
	%\opcao{Nome da Opção em português}
	%\opcaoadic{Nome da Opção em inglês}
	% O preambulo deve conter o tipo do trabalho, o objetivo, 
	% o nome da instituição, a área de concentração e opção quando houver
	% O preambulo deve conter o tipo do trabalho, o objetivo, 
	% o nome da instituição, a área de concentração e opção quando houver
	\preambulo{Trabalho de conclus\~ao de curso apresentado ao Programa de Gradua\c{c}\~ao do Instituto de Ci\^encias Matem\'aticas e de Computa\c{c}\~ao e Escola de Engenharia de S\~ao Carlos, Universidade de S\~ao Paulo - ICMC - EESC/USP, como parte dos requisitos para obten\c{c}\~ao do t\'itulo de Engenheiro de Computa\c{c}\~ao.}
	\preambuloadic{Conclusion course paper presented to the Undergraduate Program of the Instituto de Ci\^encias Matem\'aticas e de Computa\c{c}\~ao, and the Escola de Engenharia de S\~ao Carlos, Universidade de S\~ao Paulo, in partial fulfillment of the requirements for the degree of the Computer Engineer.}
	\notaficha{Monografia (Gradua\c{c}\~ao em Engenharia de Computa\c{c}\~ao)}
	\notacapaicmc{Trabalho de Conclus\~ao de Curso do Programa de Gradua\c{c}\~ao Bacharelado em\\ Engenharia de Computa\c{c}\~ao}
	}{
% Outros
	\tipotrabalho{Monografia (Trabalho de Conclus\~ao de Curso)}
	\tipotrabalhoabs{Monograph (Conclusion Course Paper)}
	\area{Nome da \'Area}
	\opcao{Nome da Op\c{c}\~ao}
    % O preambulo deve conter o tipo do trabalho, o objetivo, 
	% o nome da instituição, a área de concentração e opção quando houver				
	\preambulo{Trabalho de conclus\~ao de curso apresentado ao Instituto de Ci\^encias Matem\'aticas e de Computa\c{c}\~ao, Universidade de S\~ao Paulo - ICMC/USP, como parte dos requisitos para obten\c{c}\~ao do t\'itulo de ...}
	\preambuloadic{Term paper submitted to the Instituto de Ci\^encias Matem\'aticas e de Computa\c{c}\~ao, Universidade de S\~ao Paulo - ICMC/USP, in partial fulfillment of the requirements for the degree of the ...}
	\notaficha{Monografia (Gradua\c{c}\~ao em ...)}
	\notacapaicmc{Trabalho de Conclus\~ao de Curso do Programa de Gradua\c{c}\~ao em ...}
    }}}}}}}}}}}}}}}}}}}			







