\documentclass[
12pt,		% tamanho da fonte
openright,	% capítulos começam em pág ímpar (insere página vazia caso preciso)
twoside,  % para impressão em anverso (frente) e verso. Oposto a oneside - Nota: utilizar \imprimirfolhaderosto*
%oneside, % para impressão em páginas separadas (somente anverso) -  Nota: utilizar \imprimirfolhaderosto
% inclua uma % antes do comando twoside e exclua a % antes do oneside 
a4paper,			% tamanho do papel. 
% -- opções da classe abntex2 --
chapter=TITLE,		% títulos de capítulos convertidos em letras maiúsculas
% -- opções do pacote babel --
english,			% idioma adicional para hifenização
french,				% idioma adicional para hifenização
spanish,			% idioma adicional para hifenização
brazil				% o último idioma é o principal do documento
% {USPSC-classe/USPSC} configura o cabeçalho contendo apenas o número da página
]{USPSC-classe/USPSC}
%]{USPSC-classe/USPSC1}
% Inclua % antes de ]{USPSC-classe/USPSC} e retire a % antes de %]{USPSC-classe/USPSC1} para utilizar o 
% cabeçalho diferenciado para as páginas pares e ímpares:
%- páginas ímpares: com seções ou subseções e o número da página
%- páginas pares: com o número da página e o título do capítulo 
% Pacotes básicos - Fundamentais
\usepackage[T1]{fontenc}		% Seleção de códigos de fonte.
\usepackage[utf8]{inputenc}		% Codificação do documento (conversão automática dos acentos)
\usepackage{lmodern}			% Usa a fonte Latin Modern
% Para utilizar a fonte Times New Roman, inclua uma % no início do comando acima  "\usepackage{lmodern}"
% Abaixo, tire a % antes do comando  \usepackage{times}
%\usepackage{times}		    	% Usa a fonte Times New Roman	
% Para usar a fonte , lembre-se de tirar a % do comando %\renewcommand{\ABNTEXchapterfont}{\rmfamily}, localizado mais abaixo, logo após "Outras opções para nota de rodapé no Sistema Numérico" 					
\usepackage{lastpage}			% Usado pela Ficha catalográfica
\usepackage{indentfirst}		% Indenta o primeiro parágrafo de cada seção.
\usepackage{color}				% Controle das cores
\usepackage{graphicx}			% Inclusão de gráficos
\usepackage{float} 				% Fixa tabelas e figuras no local exato
\usepackage{chemfig}            % Para escrever reações químicas
\usepackage{chemmacros}         % Para escrever reações químicas
\usepackage{tikz}				% Para escrever reações químicas e outros
\usetikzlibrary{positioning}
\usepackage{microtype} 			% para melhorias de justificação
\usepackage{pdfpages}
\usepackage{makeidx}            % para gerar índice remissivo
\usepackage{hyphenat}          % Pacote para retirar a hifenizacao DO TEXTO
\usepackage[absolute]{textpos} % Pacote permite o posicionamento do texto
\usepackage{eso-pic}           % Pacote para incluir imagem de fundo
\usepackage{makebox}           % Pacote para criar caixa de texto
% ---

\usepackage[alf, abnt-emphasize=bf, abnt-thesis-year=both, abnt-repeated-author-omit=no, abnt-last-names=abnt, abnt-etal-cite=3, abnt-etal-list=3, abnt-etal-text=it, abnt-and-type=e, abnt-doi=doi, abnt-url-package=none, abnt-verbatim-entry=no]{abntex2cite}
\bibliographystyle{USPSC-classe/abntex2-alf-USPSC}

\renewcommand{\footnotesize}{\small}
\usepackage{lipsum}

\usepackage{multicol}	% Suporte a mesclagens em colunas
\usepackage{multirow}	% Suporte a mesclagens em linhas
\usepackage{longtable}	% Tabelas com várias páginas
\usepackage{threeparttablex}    % notas no longtable
\usepackage{array}

% Compatibilização com a ABNT NBR 6023:2018 e 10520:2023
\usepackage{USPSC-classe/ABNT6023-10520}

% ---
% DADOS INICIAIS - Define sigla com título, área de concentração e opção do Programa 
% Consulte a tabela referente aos Programas, áreas e opções de sua unidade contante do
% arquivo USPSC-Siglas estabelecidas para os Programas de Pós-Graduação nos APÊNDICES B-J
\siglaunidade{ICMC-TCC}
\programa{BCCp}

\definecolor{blue}{RGB}{41,5,195}

\makeatletter
\hypersetup{
	%pagebackref=true,
	pdftitle={\@title}, 
	pdfauthor={\@author},
	pdfsubject={\imprimirpreambulo},
	pdfcreator={LaTeX with abnTeX2},
	pdfkeywords={abnt}{latex}{abntex}{USPSC}{trabalho acadêmico}, 
	colorlinks=true,       		% false: boxed links; true: colored links
	linkcolor=black,          	% color of internal links
	citecolor=black,        		% color of links to bibliography
	filecolor=black,      		% color of file links
	urlcolor=black,
	%Para habilitar as cores dos links, retire a % antes dos comandos abaixo e inclua a % antes das 4 linhas de comando acima 
	%linkcolor=blue,            	% color of internal links
	%citecolor=blue,        		% color of links to bibliography
	%filecolor=magenta,      		% color of file links
	%urlcolor=blue,
	bookmarksdepth=4	
}
\makeatother

% O tamanho do parágrafo é dado por:
\setlength{\parindent}{1.3cm}

% Controle do espaçamento entre um parágrafo e outro:
\setlength{\parskip}{0.2cm}  % tente também \onelineskip

\makeindex


\begin{document}
	
\selectlanguage{brazil}
%\selectlanguage{english}

\frenchspacing 

\renewcommand{\ABNTEXchapterfontsize}{\fontsize{12}{12}\bfseries}
\renewcommand{\ABNTEXsectionfontsize}{\fontsize{12}{12}\bfseries}
\renewcommand{\ABNTEXsubsectionfontsize}{\fontsize{12}{12}\normalfont}
\renewcommand{\ABNTEXsubsubsectionfontsize}{\fontsize{12}{12}\normalfont}
\renewcommand{\ABNTEXsubsubsubsectionfontsize}{\fontsize{12}{12}\normalfont}

%imprimircapa
\include{USPSC-TA-PreTextual/USPSC-CapaICMC}
\AddToShipoutPicture{\BackgroundBranco}
\includepdf{USPSC-TA-PreTextual/USPSC-PaginaEmBranco.pdf}

\imprimirfolhaderosto*

\includepdf{USPSC-TA-PreTextual/USPSC-fichacatalografica.pdf}

\imprimirfolhaderostoadic*


%\include{USPSC-TA-PreTextual/USPSC-Errata}

%\includepdf{USPSC-TA-PreTextual/USPSC-folhadeaprovacao.pdf}

%\includepdf{USPSC-TA-PreTextual/USPSC-PaginaEmBranco.pdf}

%\include{USPSC-TA-PreTextual/USPSC-Dedicatoria}

%\include{USPSC-TA-PreTextual/USPSC-Agradecimentos}

%\include{USPSC-TA-PreTextual/USPSC-Epigrafe}

%%% USPSC-Resumo.tex
\setlength{\absparsep}{18pt} % ajusta o espaçamento dos parágrafos do resumo		
\begin{resumo}
	\begin{flushleft} 
			\setlength{\absparsep}{0pt} % ajusta o espaçamento da referência	
			\SingleSpacing 
			\imprimirautorabr~~\textbf{\imprimirtituloresumo}.	\imprimirdata. \pageref{LastPage} p. 
			%Substitua p. por f. quando utilizar oneside em \documentclass
			%\pageref{LastPage} f.
			\imprimirtipotrabalho~-~\imprimirinstituicao, \imprimirlocal, \imprimirdata. 
 	\end{flushleft}
\OnehalfSpacing 			
% O resumo deve ressaltar o  objetivo, o método, os resultados e as conclusões do documento.
% A ordem e a extensão  destes itens dependem do tipo de resumo (informativo ou indicativo) e do tratamento que cada item recebe no documento original.

 \textbf{Palavras-chave}: LaTeX. Classe USPSC. Tese. Dissertação. Trabalho de conclusão de curso (TCC). Relatório.
\end{resumo}

%\include{USPSC-TA-PreTextual/USPSC-Abstract}

\pdfbookmark[0]{\listfigurename}{lof}
\listoffigures*
\cleardoublepage

\pdfbookmark[0]{\listtablename}{lot}
\listoftables*
\cleardoublepage

\pdfbookmark[0]{\listofquadroname}{loq}
\listofquadro*
\cleardoublepage

\include{USPSC-TA-PreTextual/USPSC-AbreviaturasSiglas}

\include{USPSC-TA-PreTextual/USPSC-Simbolos}

\pdfbookmark[0]{\contentsname}{toc}
\tableofcontents*
\cleardoublepage

\textual

\include{USPSC-TA-Textual/USPSC-Cap1-Introducao}

\include{USPSC-TA-Textual/USPSC-Cap2-Desenvolvimento}

\include{USPSC-TA-Textual/USPSC-Cap3-Conclusao}

\postextual

\bibliography{USPSC-bib/USPSC-modelo-references}

%\glossary

\include{USPSC-TA-PosTextual/USPSC-Apendices}

%% USPSC-Anexos.tex
% ---
% Inicia os anexos
% ---
\begin{anexosenv}

% Imprime uma página indicando o início dos anexos
\partanexos

% ---
\chapter{Annex}\label{ch:annex}
% ---
Elemento opcional, que consiste em um texto ou documento não elaborado pelo autor, que serve de fundamentação,
comprovação e ilustração, conforme a ABNT NBR 14724. \cite{nbr14724}.

O \textbf{ANEXO B} exemplifica como incluir um anexo em pdf.

\chapter{Acentuação (modo texto - \LaTeX)}\label{ch:etc}

\end{anexosenv}


\include{USPSC-TA-PosTextual/USPSC-IndicesRemissivos}

\end{document}
