\documentclass{pre-project}

\usepackage{lmodern}
\usepackage[T1]{fontenc}
\usepackage[utf8]{inputenc}
\usepackage{indentfirst}
\usepackage{color}
\usepackage{graphicx}
\usepackage{microtype}
\usepackage{scalefnt}
\usepackage{balance}
\usepackage{float}
\usepackage{placeins}
\usepackage{hyperref}
\usepackage{mathptmx}
\usepackage{times}
\usepackage{textcomp}
\usepackage{gensymb} % degree. ie 360º
\usepackage{amsmath}
\usepackage{comment}
\usepackage{algorithm}
\usepackage{algorithmic}
\usepackage{multirow}
\usepackage{caption}
\usepackage{hyperref}

\usepackage{helvet}
\renewcommand{\familydefault}{\sfdefault}

\usepackage{hyperref} % Required for hyperlinks
\hypersetup{hidelinks,colorlinks,breaklinks=true,urlcolor=color2,citecolor=color1,linkcolor=color1,bookmarksopen=false,pdftitle={Title},pdfauthor={Author}}

\usepackage[brazilian,hyperpageref]{backref}
\usepackage[alf,
versalete,
abnt-emphasize = bf,
abnt-etal-list = 3,
abnt-etal-text = it,
abnt-and-type = &,
abnt-last-names = abnt, 
abnt-repeated-author-omit = yes
]{abntex2cite}

\renewcommand{\backref}{}
\renewcommand*{\backrefalt}[4]{	}

\titulo{Otimizando Simulações de Mercado com Algoritmos Genéticos: Uma Análise de Replicação de Regimes do Livro de Ordens Limite}

\autor{RAFAEL ZIMMER}

\local{São Carlos - SP}

\data{2024}

\tipotrabalho{TCC}

\orientador{OSWALDO LUIZ DO VALLE COSTA}

\preambulo{Pré-Projeto do trabalho de conclusão de curso
	apresentado ao curso de Ciência da
	Computação da Universidade de São Paulo - USP, 
	como requisito para obtenção do grau de Bacharel.
}

\makeatletter
\hypersetup{
		pdftitle={\@title}, 
		pdfauthor={\@author},
    	pdfsubject={\imprimirpreambulo},
	    pdfcreator={LaTeX with abnTeX2},
		pdfkeywords={abnt}{latex}{abntex}{abntex2}{projeto de pesquisa}, 
		colorlinks=false,
		bookmarksdepth=4,
		pdfborder={0 0 0},
}
\makeatother

\setlength{\parindent}{1.3cm}

\setlength{\parskip}{0.2cm}

\makeindex


\begin{document}

\selectlanguage{brazil}
\frenchspacing 
\pretextual

\imprimircapa
\imprimirfolhaderosto

\pdfbookmark[0]{\listfigurename}{lof}
\listoffigures*
\cleardoublepage

\pdfbookmark[0]{\contentsname}{lot}
\listoftables*
\cleardoublepage

\begin{siglas}
	\item [SIGLA] {DESCRIÇÃO DA SIGLA}
	% Incluir as siglas aqui ...
\end{siglas}

\pdfbookmark[0]{\contentsname}{toc}
\tableofcontents*
\cleardoublepage

\textual


\chapter{TEMA}
**Tema**

O livro de ordens limite é uma representação estruturada das intenções de compra e venda de ativos financeiros em um mercado. Nele, os participantes registram suas ordens de compra (com preço máximo que estão dispostos a pagar) e ordens de venda (com preço mínimo que estão dispostos a aceitar). As ordens limite são executadas somente quando o preço de mercado alcança o preço especificado na ordem. Esse mecanismo de negociação é fundamental para entender a dinâmica dos mercados financeiros e como as transações são realizadas.

Os processos de chegada de ordens referem-se à maneira como as ordens são enviadas ao mercado ao longo do tempo. Tradicionalmente, esses processos são modelados por distribuições de Poisson ou exponenciais. No entanto, abordagens mais recentes têm considerado o uso de Processos de Hawkes, que capturam a natureza auto-excitável dos mercados financeiros, onde uma ordem pode desencadear a chegada de outras ordens em um processo de reação em cadeia.

Os mercados financeiros estão sujeitos a diferentes regimes, como períodos de alta volatilidade e períodos de baixa volatilidade. Esses regimes podem influenciar significativamente as dinâmicas do livro de ordens limite e a intensidade das chegadas de ordens. Portanto, é crucial modelar esses regimes para capturar a complexidade do mercado.

A utilização de Hidden Markov Chains (HMC) oferece uma abordagem poderosa para modelar os diferentes estados do livro de ordens limite, assim como as transições entre esses estados. Cada estado pode representar um regime financeiro específico, com diferentes características de comportamento do mercado. Além disso, as intensidades de chegada de ordens podem variar dependendo do estado do mercado, refletindo a volatilidade e a atividade de negociação.

Agora, introduzindo os algoritmos genéticos, podemos explorar sua aplicação para aproximar as possíveis distribuições dos processos de chegada de ordens. Os algoritmos genéticos são uma técnica de otimização inspirada no processo de seleção natural. Eles podem ser usados para encontrar os parâmetros que melhor se ajustam aos dados históricos, replicando as características estatísticas observadas. Isso permite a geração de cenários realistas no simulador, fornecendo uma base sólida para testar estratégias de negociação em diferentes condições de mercado.

Em resumo, o tema deste trabalho abrange a modelagem do livro de ordens limite, incluindo a representação dos processos de chegada de ordens, a consideração dos regimes financeiros e o uso de Hidden Markov Chains para capturar a complexidade do mercado. Além disso, discute-se a aplicação de algoritmos genéticos para aproximar as distribuições dos processos de chegada de ordens, visando melhorar a precisão e a relevância dos simuladores de mercado financeiro.
	
\chapter{DELIMITAÇÃO DO TEMA}	
	DELIMITAÇÃO
	
\chapter{PROBLEMÁTICA}
A principal motivação para esta pesquisa é permitir que algoritmos de aprendizado por reforço possam ser treinados em simuladores de mercado com dados históricos simulados, de forma que seja possível calcular e levar em conta o impacto de mercado dos agentes. Ou seja, o principal problema não é a falta de dados, mas sim que os simuladores atuais apenas usam dados históricos como ponto de partida inicial e inserem distribuições teóricas para atualizar e criar ordens, em vez de usar dados históricos para replicar características estatísticas dos diferentes regimes.

A necessidade de realizar simulações baseadas em dados históricos é crucial devido à natureza não estacionária dos processos de tempo de chegada de ordens e do retorno financeiro de ativos. Essas características variáveis ao longo do tempo são influenciadas por mudanças nos regimes de mercado e nas próprias empresas, afetando diretamente a escolha de estratégias de negociação, o gerenciamento de riscos e os resultados obtidos pelos participantes do mercado financeiro.

A compreensão e replicação dos regimes de mercado em simulações são essenciais para testar estratégias de investimento, desenvolver modelos de precificação de ativos e validar essas estratégias. Isso porque os regimes representam padrões estatísticos nos preços, volume de negociação, volatilidade e outros indicadores ao longo do tempo, influenciando significativamente as decisões dos agentes no mercado.

Os livros de ordens limite desempenham um papel fundamental nos mercados financeiros, representando a oferta e demanda de ativos em um determinado momento. Modelar esses livros é crucial para entender como as ordens são executadas, como a liquidez é formada e como os preços são determinados no mercado. Portanto, a simulação de ambientes que reflitam os padrões reais do mercado é crucial para o desenvolvimento de estratégias de negociação mais eficazes.

A utilização de algoritmos genéticos para modelagem e simulação de regimes de mercado oferece várias vantagens, incluindo a capacidade de lidar com problemas complexos e não-lineares de forma eficiente. Além disso, esses algoritmos podem ser adaptados para incorporar diferentes fontes de informação e considerar múltiplos objetivos, tornando-os adequados para uma ampla gama de aplicações financeiras.

Tratando especificamente do problema de identificar os diversos regimes de mercado e suas probabilidades, existem diversas técnicas, como o modelo de regressão dinâmica de comutação de Markov, que podem ser empregadas. Essas técnicas ajudam na delimitação e agrupamento de períodos de mercado com base em cadeias de Markov, contribuindo para uma modelagem mais precisa dos regimes de mercado.

Em suma, esta pesquisa visa explorar como os algoritmos genéticos podem ser utilizados para modelar e simular os regimes de mercado, com foco nos livros de ordens limite. Pretende-se investigar como esses algoritmos podem ser adaptados e usados para aproximar as distribuições dos processos com base em dados históricos, visando uma representação mais fiel dos regimes de mercado e como as simulações resultantes podem ser aplicadas na prática.

Os resultados esperados desta pesquisa incluem a identificação de padrões e regularidades nos dados do mercado, a avaliação da eficácia das simulações geradas e a análise do impacto das estratégias de negociação sob diferentes condições de mercado. Dessa forma, esta pesquisa contribuirá para o avanço do conhecimento sobre a aplicação de algoritmos genéticos em finanças e para uma melhor compreensão de como esses métodos podem ser utilizados na modelagem e simulação dos regimes de mercado.
	
\chapter{HIPÓTESE}
O PORQUE DISSO ACONTECER
	
\chapter{OBJETIVOS}
	\section{Objetivo geral}
	OBJETIVO DO PROJETO
	
	\section{Objetivos específicos}
		\begin{enumerate}
			\item MOTIVO 1;
			\item MOTIVO 2;
			\item MOTIVO 4; ...
			\end{enumerate}

\chapter{JUSTIFICATIVA}
JUSTIFICAR O DO PORQUE ESTA FAZENDO ESTE PROJETO \cite{mota2012descobrindo}.

		\begin{citacao}
			CITACAO DIRETA
		\end{citacao}


\chapter{FUNDAMENTAÇÃO TEÓRICA}
	\section{\textit{Linux}}
	
	FUNDAMENTAÇÃO TEÓRICA \cite{nomenacitacao}.
	
	
	
		\begin{citacao}
		CITAÇÃO DIRETA
		\end{citacao}
	
	
		\subsection{SUBSEÇÃO}
		
			% \begin{figure*}[h]
			% \caption{\label{Debian}\textit{QRcode Github}}
			% \centering
			% {\includegraphics[width=9cm]{assets/Karan_GitHub.png}}
			% \legend{Fonte: MODELO PARA ADICAO DE IMAGENS}
			% \end{figure*}
	

	
\chapter{Procedimentos metodológicos}
	
	

\chapter{Cronograma}
\begin{table}[htb]
%\ABNTEXfontereduzida
\caption[Cronograma]{Cronograma.}
\label{tab-nivinv}		Simulando  com Algoritmos Genéticos: Uma Análise de Replicação de Regimes de Livro de Ordens Limite"
\scalefont{0.7}
	\begin{tabular}{|c|c|c|c|c|c|c|c|c|c|c|c|}
		\hline
		\textbf{Ano} & \multicolumn{11}{c|}{\textbf{-------------------- 2019/1 --------------------------|------------------- 2019/2 -------------------}} \\ \hline
		Atividades | MÊS & Fev. & Mar. & Abr. & Mai. & Jun. & Jul. & Ago. & Set. & Out. & Nov. & Dez \\ \hline
		\begin{tabular}[c]{@{}c@{}}Desenvolvimento do Tema e \\ Objetivos\end{tabular} &  &  &  &  &  &  &  &  &  &  &  \\ \hline
		\begin{tabular}[c]{@{}c@{}}Desenvolvimento do Problema,\\ Hipótese e Justificativa\end{tabular} &  &  &  &  &  &  &  &  &  &  &  \\ \hline
		\begin{tabular}[c]{@{}c@{}}Desenvolvimento da\\ Metodologia e Fundamentação Teórica\end{tabular} &  &  &  &  &  &  &  &  &  &  &  \\ \hline
		\begin{tabular}[c]{@{}c@{}}Desenvolvimento do\\ Cronograma e Orçamento\end{tabular} &  &  &  &  &  &  &  &  &  &  &  \\ \hline
		\begin{tabular}[c]{@{}c@{}}Encontros\\ de Orientação\end{tabular} & X & X & X & X & X & X & X & X & X & X & X \\ \hline
		\begin{tabular}[c]{@{}c@{}}Defesa do\\ Projeto de Pesquisa\end{tabular} &  &  &  &  &  &  &  &  &  &  &  \\ \hline
		\begin{tabular}[c]{@{}c@{}}Desenvolvimento\\ da Introdução\end{tabular} &  &  &  &  &  &  &  &  &  &  &  \\ \hline
		\begin{tabular}[c]{@{}c@{}}Desenvolvimento\\ do TCC\end{tabular} &  &  &  &  &  &  &  &  &  &  &  \\ \hline
		\begin{tabular}[c]{@{}c@{}}Correção de\\ Erros\end{tabular} &  &  &  &  &  &  &  &  &  &  &  \\ \hline
		\begin{tabular}[c]{@{}c@{}}Elaboração da\\ apresentação do TCC\end{tabular} &  &  &  &  &  &  &  &  &  &  &  \\ \hline
		\begin{tabular}[c]{@{}c@{}}Apresentação\\ do TCC\end{tabular} &  &  &  &  &  &  &  &  &  &  &  \\ \hline
		\begin{tabular}[c]{@{}c@{}}Entrega Versão\\ final\end{tabular} &  &  &  &  &  &  &  &  &  &  &  \\ \hline
	\end{tabular}
	
\end{table}


\phantompart

\postextual

\bibliography{bibliography}

\end{document}
