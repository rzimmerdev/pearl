\documentclass{pre-project}

\usepackage{lmodern}
\usepackage[T1]{fontenc}
\usepackage[utf8]{inputenc}
\usepackage{indentfirst}
\usepackage{color}
\usepackage{graphicx}
\usepackage{microtype}
\usepackage{scalefnt}
\usepackage{balance}
\usepackage{float}
\usepackage{placeins}
\usepackage{hyperref}
\usepackage{mathptmx}
\usepackage{times}
\usepackage{textcomp}
\usepackage{gensymb} % degree. ie 360º
\usepackage{amsmath}
\usepackage{comment}
\usepackage{algorithm}
\usepackage{algorithmic}
\usepackage{multirow}
\usepackage{caption}
\usepackage{hyperref}

\usepackage{helvet}
\renewcommand{\familydefault}{\sfdefault}

\usepackage{hyperref} % Required for hyperlinks
\hypersetup{hidelinks,colorlinks,breaklinks=true,urlcolor=color2,citecolor=color1,linkcolor=color1,bookmarksopen=false,pdftitle={Title},pdfauthor={Author}}

\usepackage[brazilian,hyperpageref]{backref}
\usepackage[alf,
versalete,
abnt-emphasize = bf,
abnt-etal-list = 3,
abnt-etal-text = it,
abnt-and-type = &,
abnt-last-names = abnt, 
abnt-repeated-author-omit = yes
]{abntex2cite}

\renewcommand{\backref}{}
\renewcommand*{\backrefalt}[4]{	}

\titulo{Otimizando Simulações de Mercado com Algoritmos Genéticos: Uma Análise de Replicação de Regimes do Livro de Ordens Limite}

\autor{RAFAEL ZIMMER}

\local{São Carlos - SP}

\data{2024}

\tipotrabalho{TCC}

\orientador{OSWALDO LUIZ DO VALLE COSTA}

\preambulo{Pré-Projeto do trabalho de conclusão de curso
	apresentado ao curso de Ciência da
	Computação da Universidade de São Paulo - USP, 
	como requisito para obtenção do grau de Bacharel.
}

\makeatletter
\hypersetup{
		pdftitle={\@title}, 
		pdfauthor={\@author},
    	pdfsubject={\imprimirpreambulo},
	    pdfcreator={LaTeX with abnTeX2},
		pdfkeywords={abnt}{latex}{abntex}{abntex2}{projeto de pesquisa}, 
		colorlinks=false,
		bookmarksdepth=4,
		pdfborder={0 0 0},
}
\makeatother

\setlength{\parindent}{1.3cm}

\setlength{\parskip}{0.2cm}

\makeindex


\begin{document}

\selectlanguage{brazil}
\frenchspacing 
\pretextual

\imprimircapa
\imprimirfolhaderosto

\pdfbookmark[0]{\listfigurename}{lof}
\listoffigures*
\cleardoublepage

\pdfbookmark[0]{\contentsname}{lot}
\listoftables*
\cleardoublepage

\begin{siglas}
	\item [SIGLA] {DESCRIÇÃO DA SIGLA}
	% Incluir as siglas aqui ...
\end{siglas}

\pdfbookmark[0]{\contentsname}{toc}
\tableofcontents*
\cleardoublepage

\textual


\chapter{TEMA}
TEMA DO PROJETO
	
\chapter{DELIMITAÇÃO DO TEMA}	
	DELIMITAÇÃO
	
\chapter{PROBLEMA}
DIZER QUAIS AS PROBLEMÁTICAS
	
\chapter{HIPÓTESE}
O PORQUE DISSO ACONTECER
	
\chapter{OBJETIVOS}
	\section{Objetivo geral}
	OBJETIVO DO PROJETO
	
	\section{Objetivos específicos}
		\begin{enumerate}
			\item MOTIVO 1;
			\item MOTIVO 2;
			\item MOTIVO 4; ...
			\end{enumerate}

\chapter{JUSTIFICATIVA}
JUSTIFICAR O DO PORQUE ESTA FAZENDO ESTE PROJETO \cite{mota2012descobrindo}.

		\begin{citacao}
			CITACAO DIRETA
		\end{citacao}


\chapter{FUNDAMENTAÇÃO TEÓRICA}
	\section{\textit{Linux}}
	
	FUNDAMENTAÇÃO TEÓRICA \cite{nomenacitacao}.
	
	
	
		\begin{citacao}
		CITAÇÃO DIRETA
		\end{citacao}
	
	
		\subsection{SUBSEÇÃO}
		
			\begin{figure*}[h]
			\caption{\label{Debian}\textit{QRcode Github}}
			\centering
			{\includegraphics[width=9cm]{Imagens/Karan_GitHub.png}}
			\legend{Fonte: MODELO PARA ADICAO DE IMAGENS}
			\end{figure*}
	

	
\chapter{Procedimentos metodológicos}
	
	

\chapter{Cronograma}
\begin{table}[htb]
%\ABNTEXfontereduzida
\caption[Cronograma]{Cronograma.}
\label{tab-nivinv}		
\scalefont{0.7}
	\begin{tabular}{|c|c|c|c|c|c|c|c|c|c|c|c|}
		\hline
		\textbf{Ano} & \multicolumn{11}{c|}{\textbf{-------------------- 2019/1 --------------------------|------------------- 2019/2 -------------------}} \\ \hline
		Atividades | MÊS & Fev. & Mar. & Abr. & Mai. & Jun. & Jul. & Ago. & Set. & Out. & Nov. & Dez \\ \hline
		\begin{tabular}[c]{@{}c@{}}Desenvolvimento do Tema e \\ Objetivos\end{tabular} &  &  &  &  &  &  &  &  &  &  &  \\ \hline
		\begin{tabular}[c]{@{}c@{}}Desenvolvimento do Problema,\\ Hipótese e Justificativa\end{tabular} &  &  &  &  &  &  &  &  &  &  &  \\ \hline
		\begin{tabular}[c]{@{}c@{}}Desenvolvimento da\\ Metodologia e Fundamentação Teórica\end{tabular} &  &  &  &  &  &  &  &  &  &  &  \\ \hline
		\begin{tabular}[c]{@{}c@{}}Desenvolvimento do\\ Cronograma e Orçamento\end{tabular} &  &  &  &  &  &  &  &  &  &  &  \\ \hline
		\begin{tabular}[c]{@{}c@{}}Encontros\\ de Orientação\end{tabular} & X & X & X & X & X & X & X & X & X & X & X \\ \hline
		\begin{tabular}[c]{@{}c@{}}Defesa do\\ Projeto de Pesquisa\end{tabular} &  &  &  &  &  &  &  &  &  &  &  \\ \hline
		\begin{tabular}[c]{@{}c@{}}Desenvolvimento\\ da Introdução\end{tabular} &  &  &  &  &  &  &  &  &  &  &  \\ \hline
		\begin{tabular}[c]{@{}c@{}}Desenvolvimento\\ do TCC\end{tabular} &  &  &  &  &  &  &  &  &  &  &  \\ \hline
		\begin{tabular}[c]{@{}c@{}}Correção de\\ Erros\end{tabular} &  &  &  &  &  &  &  &  &  &  &  \\ \hline
		\begin{tabular}[c]{@{}c@{}}Elaboração da\\ apresentação do TCC\end{tabular} &  &  &  &  &  &  &  &  &  &  &  \\ \hline
		\begin{tabular}[c]{@{}c@{}}Apresentação\\ do TCC\end{tabular} &  &  &  &  &  &  &  &  &  &  &  \\ \hline
		\begin{tabular}[c]{@{}c@{}}Entrega Versão\\ final\end{tabular} &  &  &  &  &  &  &  &  &  &  &  \\ \hline
	\end{tabular}
	
\end{table}


\phantompart

\postextual

\bibliography{bibliografia}

\end{document}
