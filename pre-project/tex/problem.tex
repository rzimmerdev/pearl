A principal motivação para esta pesquisa é permitir que algoritmos de aprendizado por reforço possam ser treinados em simuladores de mercado com dados históricos simulados, de forma que seja possível calcular e levar em conta o impacto de mercado dos agentes. Ou seja, o principal problema não é a falta de dados, mas sim que os simuladores atuais apenas usam dados históricos como ponto de partida inicial e inserem distribuições teóricas para atualizar e criar ordens, em vez de usar dados históricos para replicar características estatísticas dos diferentes regimes.

A necessidade de realizar simulações baseadas em dados históricos é crucial devido à natureza não estacionária dos processos de tempo de chegada de ordens e do retorno financeiro de ativos. Essas características variáveis ao longo do tempo são influenciadas por mudanças nos regimes de mercado e nas próprias empresas, afetando diretamente a escolha de estratégias de negociação, o gerenciamento de riscos e os resultados obtidos pelos participantes do mercado financeiro.

A compreensão e replicação dos regimes de mercado em simulações são essenciais para testar estratégias de investimento, desenvolver modelos de precificação de ativos e validar essas estratégias. Isso porque os regimes representam padrões estatísticos nos preços, volume de negociação, volatilidade e outros indicadores ao longo do tempo, influenciando significativamente as decisões dos agentes no mercado.

Os livros de ordens limite desempenham um papel fundamental nos mercados financeiros, representando a oferta e demanda de ativos em um determinado momento. Modelar esses livros é crucial para entender como as ordens são executadas, como a liquidez é formada e como os preços são determinados no mercado. Portanto, a simulação de ambientes que reflitam os padrões reais do mercado é crucial para o desenvolvimento de estratégias de negociação mais eficazes.

A utilização de algoritmos genéticos para modelagem e simulação de regimes de mercado oferece várias vantagens, incluindo a capacidade de lidar com problemas complexos e não-lineares de forma eficiente. Além disso, esses algoritmos podem ser adaptados para incorporar diferentes fontes de informação e considerar múltiplos objetivos, tornando-os adequados para uma ampla gama de aplicações financeiras.

Tratando especificamente do problema de identificar os diversos regimes de mercado e suas probabilidades, existem diversas técnicas, como o modelo de regressão dinâmica de comutação de Markov, que podem ser empregadas. Essas técnicas ajudam na delimitação e agrupamento de períodos de mercado com base em cadeias de Markov, contribuindo para uma modelagem mais precisa dos regimes de mercado.

Em suma, esta pesquisa visa explorar como os algoritmos genéticos podem ser utilizados para modelar e simular os regimes de mercado, com foco nos livros de ordens limite. Pretende-se investigar como esses algoritmos podem ser adaptados e usados para aproximar as distribuições dos processos com base em dados históricos, visando uma representação mais fiel dos regimes de mercado e como as simulações resultantes podem ser aplicadas na prática.

Os resultados esperados desta pesquisa incluem a identificação de padrões e regularidades nos dados do mercado, a avaliação da eficácia das simulações geradas e a análise do impacto das estratégias de negociação sob diferentes condições de mercado. Dessa forma, esta pesquisa contribuirá para o avanço do conhecimento sobre a aplicação de algoritmos genéticos em finanças e para uma melhor compreensão de como esses métodos podem ser utilizados na modelagem e simulação dos regimes de mercado.