O tema deste pré-projeto de pesquisa é a aplicação de algoritmos genéticos na modelagem e simulação de regimes de mercado, com foco específico nos livros de ordens limite. Algoritmos genéticos são métodos de otimização baseados em princípios evolutivos, que buscam encontrar soluções para problemas complexos através da seleção, recombinação e mutação de indivíduos em uma população. Neste contexto, os algoritmos genéticos são utilizados para aproximar as distribuições dos processos estocásticos que descrevem a chegada de ordens limite em um mercado financeiro.

Entender os diferentes regimes de mercado é fundamental para os participantes do mercado financeiro, pois eles influenciam diretamente as estratégias de negociação, o gerenciamento de risco e os resultados financeiros. Um regime de mercado refere-se aos padrões observados nos preços, volume de negociação, volatilidade e outros indicadores ao longo do tempo. Compreender e replicar esses regimes em simulações é essencial para testar estratégias de investimento, desenvolver modelos de precificação de ativos e entender o comportamento do mercado em diferentes condições.

Os livros de ordens limite desempenham um papel fundamental nos mercados financeiros, pois representam a oferta e a demanda de ativos em um determinado momento. Eles exibem as ordens de compra e venda, juntamente com seus respectivos preços e quantidades, permitindo que os participantes do mercado visualizem a liquidez disponível e as intenções de negociação dos outros participantes. Modelar e simular os livros de ordens limite é essencial para entender como as ordens são executadas, como a liquidez é formada e como os preços são determinados no mercado.

A utilização de algoritmos genéticos para modelagem e simulação de regimes de mercado oferece várias vantagens. Esses algoritmos são capazes de lidar com problemas complexos e não-lineares, e podem encontrar soluções aproximadas eficientemente. Além disso, eles podem ser adaptados para incorporar diferentes fontes de informação e considerar múltiplos objetivos, tornando-os adequados para uma ampla gama de aplicações financeiras.

A pesquisa proposta visa explorar como os algoritmos genéticos podem ser utilizados para modelar e simular os regimes de mercado, com foco particular nos livros de ordens limite. Pretende-se investigar como esses algoritmos podem ser adaptados e calibrados para melhor representar os processos estocásticos subjacentes, e como as simulações resultantes podem ser usadas para entender o comportamento do mercado, testar estratégias de negociação e avaliar o risco.

Esta pesquisa contribuirá para o avanço do conhecimento sobre a aplicação de algoritmos genéticos em finanças, fornecendo insights sobre como esses métodos podem ser utilizados para modelar e simular os regimes de mercado. Os resultados esperados incluem a identificação de padrões e regularidades nos dados do mercado, a avaliação da eficácia das simulações geradas e a análise do impacto das estratégias de negociação sob diferentes condições de mercado.

Em última análise, espera-se que esta pesquisa ajude os participantes do mercado financeiro a tomar decisões mais informadas e a desenvolver estratégias de investimento mais robustas e eficazes. Ao entender melhor os regimes de mercado e as dinâmicas do livro de ordens limite, os investidores estarão melhor preparados para enfrentar os desafios e aproveitar as oportunidades nos mercados financeiros.
