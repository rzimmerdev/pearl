O tema deste pré-projeto de pesquisa é a aplicação de algoritmos genéticos na modelagem e simulação de regimes de mercado, com foco específico nos livros de ordens limite. Algoritmos genéticos são métodos de otimização baseados em princípios evolutivos, que buscam encontrar soluções para problemas complexos através da seleção, recombinação e mutação de indivíduos em uma população \cite{goldberg1989genetic, holland1975adaptation}. Durante a pesquisa, os algoritmos genéticos serão utilizados como aproximadores sas distribuições dos processos estocásticos que descrevem o tempo de chegada de ordens limite em um determinado mercado.

Contudo, é conhecido que os processos de tempo de chegada de ordens, assim como o do retorno financeiro de ativos, entre outros, não são estacionários, e tem suas características como média, variância e em alguns casos a classificação da distribuição adjacente alteradas ao longo do tempo \cite{cont2001empirical}. Essas alterações ocorrem tanto devido à mudanças nos regimes de mercado como no regime de subáreas ou das próprias empresas. Portanto é fundamental para os participantes do mercado financeiro conseguirem analisar e identificar tais regimes, pois afetam diretamente na escolha de estratégias de negociação, gerenciamento de risco e resultados obtidos \cite{gatheral2010no}. No contexto de matemática financeira um regime refere-se aos padrões estatísticos observados nos preços, volume de negociação, volatilidade e outros indicadores ao longo do tempo. Compreender e replicar esses regimes em simulações é essencial para testar estratégias de investimento, desenvolver modelos de precificação de ativos e validar as mesmas \cite{back1997handbook, mitchell1996introduction}.

Os livros de ordens limite desempenham um papel fundamental nos mercados financeiros, pois representam a oferta e a demanda de ativos em um determinado momento \cite{hasbrouck2007empirical}. Eles exibem as ordens de compra e venda, juntamente com seus respectivos preços e quantidades, permitindo que os participantes do mercado visualizem a liquidez disponível e as intenções de negociação dos outros participantes. Modelar os livros de ordens limite é essencial para entender como as ordens são executadas, como a liquidez é formada e como os preços são determinados no mercado, e a simulação de ambientes que repliquem os padrões do mercado permite o desenvolvimento de estratégias de negociação mais complexas.

A utilização de algoritmos genéticos para modelagem e simulação de regimes de mercado oferece várias vantagens \cite{goldberg1989genetic, mitchell1996introduction}. Esses algoritmos são capazes de lidar com problemas complexos e não-lineares, e podem encontrar aproximações para os processos de chegada eficientemente. Além disso, eles podem ser adaptados para incorporar diferentes fontes de informação e considerar múltiplos objetivos, tornando-os adequados para uma ampla gama de aplicações financeiras. Tratando do problema de identificar os diversos regimes e suas probabilidades, existem diversas técnicas para delimitação e agrupamento de períodos do mercado com base em cadeias de Markov, como o modelo de regressão dinâmica de comutação de Markov que será utilizado na pesquisa\cite{cont2004financial}.

Como um todo, a proposta visa explorar como os algoritmos genéticos podem ser utilizados para modelar e simular os regimes de mercado, com foco particular nos livros de ordens limite. Pretende-se investigar como esses algoritmos podem ser adaptados e usados para aproximar as distribuições dos processos com base em dados históricos, para melhor representar, e como as simulações resultantes podem ser usadas.

Esta pesquisa contribuirá para o avanço do conhecimento sobre a aplicação de algoritmos genéticos em finanças e sobre como esses métodos podem ser utilizados para modelar e simular os regimes de mercado. Os resultados esperados incluem a identificação de padrões e regularidades nos dados do mercado, a avaliação da eficácia das simulações geradas e a análise do impacto das estratégias de negociação sob diferentes condições de mercado.