%% USPSC-Errata.tex
\begin{errata}
	%\OnehalfSpacing 			
	A errata é um elemento opcional, que consiste de uma lista de erros da obra, precedidos pelas folhas e linhas onde eles ocorrem e seguidos pelas correções correspondentes. Deve ser inserida logo após a folha de rosto e conter a referência do trabalho para facilitar sua identificação, conforme a ABNT NBR 14724 \cite{nbr14724}.
	
	Modelo de Errata:
		
	\begin{flushleft} 
			\setlength{\absparsep}{0pt} % ajusta o espaçamento da referência	
			\SingleSpacing 
			\imprimirautorabr~ ~\textbf{\imprimirtituloresumo}.	\imprimirdata. \pageref{LastPage} p. 
			%Substitua p. por f. quando utilizar oneside em \documentclass
			%\pageref{LastPage} f.
			\imprimirtipotrabalho~-~\imprimirinstituicao, \imprimirlocal, \imprimirdata. 
 	\end{flushleft}
\vspace{\onelineskip}
\OnehalfSpacing 
\center
\textbf{ERRATA}
\vspace{\onelineskip}
\OnehalfSpacing 
\begin{table}[htb]
	\center
	\footnotesize
	\begin{tabular}{p{2cm} p{2cm} p{4cm} p{4cm} }
		\hline
		\textbf{Folha} & \textbf{Linha}  & \textbf{Onde se lê}  & \textbf{Leia-se}  \\
			\hline
			1 & 10 & auto-conclavo & autoconclavo\\
		\hline
	\end{tabular}
\end{table}
\end{errata}
% ---