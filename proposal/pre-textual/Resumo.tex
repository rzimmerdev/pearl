% Não altere nada nesta parte
\chapter*{\hspace{7.55cm}{RESUMO}}
\addcontentsline{toc}{chapter}{Resumo}

\begin{center}
  {\bfseries \sffamily \fontsize{12}{12} \TítuloDoTrabalho}
\end{center}


%%%%
% Insira aqui o seu resumo em português
\begin{abstract}
  O presente trabalho propõe uma técnica de modelagem para simuladores de mercado financeiro, especificamente das dinâmicas de livros de ordens limite. A utilização de simuladores se revela de extrema importância para testar estratégias de negociação sem depender de grandes volumes de dados históricos, especialmente para estratégias cujo impacto no estado do livro de ordens não é desconsiderável e também não é refletido ao se utilizar ordens históricas e estáticas.
  
  O simulador a ser desenvolvido é fundamentado na teoria de modelos ocultos de Markov (\textit{Hidden Markov Chains}, ou \textit{HMC}, em inglês) para modelar o comportamento dinâmico do livro de ordens limite. Além disso, serão comparadas diferentes distribuições para os processos de chegada de ordens e suas intensidades, especificamente os Processos de Hawkes, Cadeias de Markov Autoregressivas e Modelos de Hawkes Dependentes do Estado. Por fim, a principal contribuição do trabalho introduzir o uso de algoritmos genéticos como técnica para aproximar as distribuições dos processos de chegada de ordens dependentes dos estados, utilizando-se estimativas por máxima verossimilhança para comparar com outras técnicas utilizadas na literatura. 
  
  A abordagem proposta visa melhorar a precisão das simulações e fornecer insights mais robustos sobre o comportamento do mercado financeiro em diferentes cenários.
  O uso de algoritmos evolutivos tem como objetivo contribuir no desenvolvimento e na melhor compreensão de modelos simuladores do mercado financeiro, assim como o desenvolvimento de estratégias de negociação mais eficientes e resilientes a diferentes regimes de mercado.
\end{abstract}

% Não altere a seguir. As palavras-chave estão no arquivo
% TodasInformações.tex e são usadas em mais de uma parte do texto.
% Abra tal arquivo para eventuais alterações.
\begin{abstract}
\vspace{-.75cm}
\noindent\textbf{Palavras-chave:} \PalavrasChave
\noindent\textbf{JEL:} \JELCodes
\end{abstract}
