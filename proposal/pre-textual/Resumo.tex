% Não altere nada nesta parte
\section*{\hspace{7.55cm}{RESUMO}}
\addcontentsline{toc}{chapter}{Resumo}

\begin{center}
  {\bfseries \sffamily \fontsize{12}{12} \TítuloDoTrabalho}
\end{center}


%%%%
% Insira aqui o seu resumo em português
\begin{abstract}
  O presente trabalho propõe a implementação e comparação métricas computacionais de técnicas de modelagem para simuladores de mercado financeiro, especificamente das dinâmicas de livros de ordens limite. A utilização de simuladores se revela de extrema importância para testar estratégias de negociação sem depender de grandes volumes de dados históricos, especialmente para estratégias cujo impacto no estado do livro de ordens não é desconsiderável e também não é refletido ao se utilizar ordens históricas e estáticas.
  
  As técnicas desenvolvidas na literatura de simuladores são fundamentadas na teoria de modelos ocultos de Markov (\textit{Hidden Markov Chains}, ou \textit{HMC}, em inglês) e modelam o comportamento do livro de ordens limite utilizando estados ocultos e estados observáveis para as dinâmicas de preços, quantidades e tempo entre eventos. Esta tese terá como contribuição principal a implementação dessas técnicas em formato concorrente, vetorizado e por fim paralelizado, apresentando e comparando métricas de tempo de execução, características assintóticas dos algoritmos, escalabilidade e eficiência quando comparados entre si. Para tal, será desenvolvido um \textit{framework} em comum e replicável para execução das instâncias de simulação. 
  
  Dentre as técnicas a serem implementadas, a tese terá como foco as variações de simuladores mais comumente utilizadas na literatura, tendo como inspiração análises bibliográficas realizadas na área. Em suma, serão implementadas técnicas de amostragem por métodos de Monte Carlo para Cadeias de Markov (MCMC), processos de Hawkes Multidimensionais, modelos de Markov Escondidos (HMM) e por fim redes neurais com memória e atenção (LSTM e Transformers). A principal contribuição do trabalho será introduzir o \textit{framework} para comparação e execução dos simuladores de acordo com os paradigmas mencionados, utilizando-se estimativas por máxima verossimilhança para aproximar os modelos e comparar os resultados com àqueles apresentados na literatura. 
  
  A abordagem proposta visa comparar as técnicas das simulações sob a perspectiva de diferentes paradigmas de programação, fornecendo insights mais robustos sobre a aplicabilidade das mesmas em cenários práticos do mercado financeiro.
  O uso de um \textit{framework} para comparação computacional de modelos simuladores de dinâmicas do mercado financeiro possibilitam o desenvolvimento e teste de estratégias de negociação mais eficientes e resilientes a diferentes regimes de mercado.
\end{abstract}

% Não altere a seguir. As palavras-chave estão no arquivo
% TodasInformações.tex e são usadas em mais de uma parte do texto.
% Abra tal arquivo para eventuais alterações.
\begin{abstract}
\vspace{-.75cm}
\noindent\textbf{Palavras-chave:} \PalavrasChave
\noindent\textbf{JEL:} \JELCodes
\end{abstract}
