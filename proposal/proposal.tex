%% USPSC-TCC-modelo-ICMCp.tex
% ---------------------------------------------------------------
% USPSC: Modelo de Trabalho Acadêmico (tese de doutorado, dissertacao de
% mestrado e trabalhos monograficos em geral) em conformidade com 
% ABNT NBR 14724:2011: Informação e documentação - Trabalhos acadêmicos -
% Apresentação
%----------------------------------------------------------------
%% Esta é uma customização do abntex2-modelo-trabalho-academico.tex de v-1.9.5 laurocesar 
%% para as Unidades do Campus USP de São Carlos:
%% EESC - Escola de Engenharia de São Carlos
%% IAU - Instituto de Arquitetura e Urbanismo
%% ICMC - Instituto de Ciências Matemáticas e de Computação
%% IFSC - Instituto de Física de São Carlos
%% IQSC - Instituto de Química de São Carlos
%%
%% Este trabalho utiliza a classe USPSC (USPSC.cls e USPSC1.cls) que é mantida 
%% pela seguinte equipe:
%% 
%% Coordenação e Programação:
%%   - Marilza Aparecida Rodrigues Tognetti - marilza@sc.usp.br (PUSP-SC)
%%   - Ana Paula Aparecida Calabrez - aninha@sc.usp.br (PUSP-SC)
%% Normalização:
%%   - Brianda de Oliveira Ordonho Sigolo - brianda@usp.br (IAU)
%%   - Eduardo Graziosi Silva - edu.gs@sc.usp.br (EESC)
%%   - Eliana de Cássia Aquareli Cordeiro - eliana@iqsc.usp.br (IQSC)
%%   - Flávia Helena Cassin - cassinp@sc.usp.br (EESC)
%%   - Maria Cristina Cavarette Dziabas - mcdziaba@ifsc.usp.br (IFSC)
%%   - Regina Célia Vidal Medeiros - rcvmat@icmc.usp.br (ICMC)
%%
%% Os modelos desenvolvidos utilizam diversos arquivos relacionado em 
%% 2.1 Pacote USPSC: Classe USPSC e modelos de trabalhos acadêmicos	do Tutorial do Pacote 
%%  USPSC para modelos de trabalhos de acadêmicos em LaTeX - versão 3.2

%----------------------------------------------------------------
%% Sobre a classe abntex2.cls:
%% abntex2.cls, v-1.9.5 laurocesar
%% Copyright 2012-2015 by abnTeX2 group at https://www.abntex.net.br/ 
%%
%----------------------------------------------------------------

\documentclass[
% -- opções da classe memoir --
12pt,		% tamanho da fonte
openright,	% capítulos começam em pág ímpar (insere página vazia caso preciso)
twoside,  % para impressão em anverso (frente) e verso. Oposto a oneside - Nota: utilizar \imprimirfolhaderosto*
%oneside, % para impressão em páginas separadas (somente anverso) -  Nota: utilizar \imprimirfolhaderosto
% inclua uma % antes do comando twoside e exclua a % antes do oneside 
a4paper,			% tamanho do papel. 
% -- opções da classe abntex2 --
chapter=TITLE,		% títulos de capítulos convertidos em letras maiúsculas
% -- opções do pacote babel --
english,			% idioma adicional para hifenização
french,				% idioma adicional para hifenização
spanish,			% idioma adicional para hifenização
brazil				% o último idioma é o principal do documento
% {USPSC-classe/USPSC} configura o cabeçalho contendo apenas o número da página
]{USPSC-classe/USPSC}
%]{USPSC-classe/USPSC1}
% Inclua % antes de ]{USPSC-classe/USPSC} e retire a % antes de %]{USPSC-classe/USPSC1} para utilizar o 
% cabeçalho diferenciado para as páginas pares e ímpares:
%- páginas ímpares: com seções ou subseções e o número da página
%- páginas pares: com o número da página e o título do capítulo 
% ---
% ---
% Pacotes básicos - Fundamentais 
% ---
\usepackage[T1]{fontenc}		% Seleção de códigos de fonte.
\usepackage[utf8]{inputenc}		% Codificação do documento (conversão automática dos acentos)
\usepackage{lmodern}			% Usa a fonte Latin Modern
% Para utilizar a fonte Times New Roman, inclua uma % no início do comando acima  "\usepackage{lmodern}"
% Abaixo, tire a % antes do comando  \usepackage{times}
%\usepackage{times}		    	% Usa a fonte Times New Roman	
% Para usar a fonte , lembre-se de tirar a % do comando %\renewcommand{\ABNTEXchapterfont}{\rmfamily}, localizado mais abaixo, logo após "Outras opções para nota de rodapé no Sistema Numérico" 					
\usepackage{lastpage}			% Usado pela Ficha catalográfica
\usepackage{indentfirst}		% Indenta o primeiro parágrafo de cada seção.
\usepackage{color}				% Controle das cores
\usepackage{graphicx}			% Inclusão de gráficos
\usepackage{float} 				% Fixa tabelas e figuras no local exato
\usepackage{chemfig}            % Para escrever reações químicas
\usepackage{chemmacros}         % Para escrever reações químicas
\usepackage{tikz}				% Para escrever reações químicas e outros
\usetikzlibrary{positioning}
\usepackage{microtype} 			% para melhorias de justificação
\usepackage{pdfpages}
\usepackage{makeidx}            % para gerar índice remissivo
\usepackage{hyphenat}          % Pacote para retirar a hifenizacao DO TEXTO
\usepackage[absolute]{textpos} % Pacote permite o posicionamento do texto
\usepackage{eso-pic}           % Pacote para incluir imagem de fundo
\usepackage{makebox}           % Pacote para criar caixa de texto
% ---

% ---
% Pacotes de citações
% Citações padrão ABNT
% ---
% Sistemas de chamada: autor-data ou numérico.
% Sistema autor-data
\usepackage[alf, abnt-emphasize=bf, abnt-thesis-year=both, abnt-repeated-author-omit=no, abnt-last-names=abnt, abnt-etal-cite=3, abnt-etal-list=3, abnt-etal-text=it, abnt-and-type=e, abnt-doi=doi, abnt-url-package=none, abnt-verbatim-entry=no]{abntex2cite}
\bibliographystyle{USPSC-classe/abntex2-alf-USPSC}
% Se o idioma for o inglês, inclua % no comando acima e exclua o % do comando abaixo
%\bibliographystyle{USPSC-classe/abntex2-alfeng-USPSC}

% Para o IQSC, que indica todos os autores nas referências, incluir % no início dos comandos acima e retirar a % dos comandos abaixo 
%\usepackage[alf, abnt-emphasize=bf, abnt-thesis-year=both, abnt-repeated-author-omit=no, abnt-last-names=abnt, abnt-etal-cite=3, abnt-etal-list=0, abnt-etal-text=it, abnt-and-type=e, abnt-doi=doi, abnt-url-package=none, abnt-verbatim-entry=no]{abntex2cite} 
%\bibliographystyle{USPSC-classe/abntex2-alf-USPSC}
% Se o idioma for o inglês, exclua % no comando acima ou do comando abaixo
%\bibliographystyle{USPSC-classe/abntex2-alfeng-USPSC}

% Sistema Numérico
% Para citações numéricas, sistema adotado pelo IFSC, incluir % no início dos comandos acima e retirar a % dos comandos abaixo 
%\usepackage{cite}              % agrupa citações numéricas consecutivas
%\usepackage[num, abnt-emphasize=bf, abnt-thesis-year=both, abnt-repeated-author-omit=no, abnt-last-names=abnt, abnt-etal-cite=3, abnt-etal-list=3, abnt-etal-text=it, abnt-and-type=e, abnt-doi=doi, abnt-url-package=none, abnt-verbatim-entry=no]{abntex2cite} 
%\bibliographystyle{USPSC-classe/abntex2-num-USPSC}
% Se o idioma for o inglês, exclua % no comando acima ou do comando abaixo
%\bibliographystyle{USPSC-classe/abntex2-numeng-USPSC}

% Complementarmente, verifique as instruções abaixo sobre os Pacotes de Nota de rodapé
% ---
% Pacotes de Nota de rodapé
% Configurações de nota de rodapé

% O presente modelo adota o formato numérico para as notas de rodapés quando utiliza o sistema de chamada autor-data para citações e referências. Para utilizar o sistema de chamada numérico para citações e referências, habilitar um dos comandos abaixo.
% Há diversa opções para nota de rodapé no Sistema Numérico.  Para o IFSC, habilitade o comando abaixo.

%\renewcommand{\thefootnote}{\fnsymbol{footnote}}  %Comando para inserção de símbolos em nota de rodapé

% Outras opções para nota de rodapé no Sistema Numérico:
%\renewcommand{\thefootnote}{\alph{footnote}}      %Comando para inserção de letras minúscula em nota de rodapé
%\renewcommand{\thefootnote}{\Alph{footnote}}      %Comando para inserção de letras maiúscula em nota de rodapé
%\renewcommand{\thefootnote}{\roman{footnote}}     %Comando para inserção de números romanos minúsculos  em nota de rodapé
%\renewcommand{\thefootnote}{\Roman{footnote}}     %Comando para inserção de números romanos minúsculos  em nota de rodapé

\renewcommand{\footnotesize}{\small} %Comando para diminuir a fonte das notas de rodapé
%Para utilizar a fonte Times New Roman, inclua retire % do início do comando abaixo 
%\renewcommand{\ABNTEXchapterfont}{\rmfamily}

% ---
% Pacote para agrupar a citação numérica consecutiva
% Quando for adotado o Sistema Numérico, a exemplo do IFSC, habilite 
% o pacote cite abaixo retirando a porcentagem antes do comando abaixo
%\usepackage[superscript]{cite}	

% ---
% Pacotes adicionais, usados apenas no âmbito do Modelo Canônico do abnteX2
% ---
\usepackage{lipsum}				% para geração de dummy text
% ---

% pacotes de tabelas
\usepackage{multicol}	% Suporte a mesclagens em colunas
\usepackage{multirow}	% Suporte a mesclagens em linhas
\usepackage{longtable}	% Tabelas com várias páginas
\usepackage{threeparttablex}    % notas no longtable
\usepackage{array}

% ----
% Compatibilização com a ABNT NBR 6023:2018 e 10520:2023
% Para tirar <> da URL e tornar as expressões latinas em itálico
\usepackage{USPSC-classe/ABNT6023-10520}
% As demais compatibilizações estão nos arquivos abntex2-alf-USPSC.bst,abntex2-alfeng-USPSC.bst, abntex2-num-USPSC.bst e abntex2-numeng-USPSC.bst, dependendo do idioma do textos e se o sistemas de chamada for autor-data ou numérico, conforme explicitado acima.
% ----

% ---
% DADOS INICIAIS - Define sigla com título, área de concentração e opção do Programa 
% Consulte a tabela referente aos Programas, áreas e opções de sua unidade contante do
% arquivo USPSC-Siglas estabelecidas para os Programas de Pós-Graduação nos APÊNDICES B-J
\siglaunidade{ICMC-TCC}
\programa{BCCp}
% Os demais dados deverão ser fornecidos no arquivo USPSC-pre-textual-UUUU ou USPSC-TCC-pre-textual-UUUU, onde UUUU é a sigla da Unidade. 
% Exemplo:USPSC-pre-textual-IFSC.tex
% ---
% Configurações de aparência do PDF final
% alterando o aspecto da cor azul
\definecolor{blue}{RGB}{41,5,195}

% informações do PDF
\makeatletter
\hypersetup{
	%pagebackref=true,
	pdftitle={\@title}, 
	pdfauthor={\@author},
	pdfsubject={\imprimirpreambulo},
	pdfcreator={LaTeX with abnTeX2},
	pdfkeywords={abnt}{latex}{abntex}{USPSC}{trabalho acadêmico}, 
	colorlinks=true,       		% false: boxed links; true: colored links
	linkcolor=black,          	% color of internal links
	citecolor=black,        		% color of links to bibliography
	filecolor=black,      		% color of file links
	urlcolor=black,
	%Para habilitar as cores dos links, retire a % antes dos comandos abaixo e inclua a % antes das 4 linhas de comando acima 
	%linkcolor=blue,            	% color of internal links
	%citecolor=blue,        		% color of links to bibliography
	%filecolor=magenta,      		% color of file links
	%urlcolor=blue,
	bookmarksdepth=4	
}
\makeatother
% --- 
% --- 
% Espaçamentos entre linhas e parágrafos 
% --- 

% O tamanho do parágrafo é dado por:
\setlength{\parindent}{1.3cm}

% Controle do espaçamento entre um parágrafo e outro:
\setlength{\parskip}{0.2cm}  % tente também \onelineskip

% ---
% compila o sumário e índice
\makeindex
% ---

% ----
% Início do documento
% ----
\begin{document}

% Seleciona o idioma do documento (conforme pacotes do babel)
\selectlanguage{brazil}
% Se o idioma do texto for inglês, inclua uma % antes do 
%      comando \selectlanguage{brazil} e 
%      retire a % antes do comando abaixo
%\selectlanguage{english}

% Retira espaço extra obsoleto entre as frases.
\frenchspacing 

% --- Formatação dos Títulos
\renewcommand{\ABNTEXchapterfontsize}{\fontsize{12}{12}\bfseries}
\renewcommand{\ABNTEXsectionfontsize}{\fontsize{12}{12}\bfseries}
\renewcommand{\ABNTEXsubsectionfontsize}{\fontsize{12}{12}\normalfont}
\renewcommand{\ABNTEXsubsubsectionfontsize}{\fontsize{12}{12}\normalfont}
\renewcommand{\ABNTEXsubsubsubsectionfontsize}{\fontsize{12}{12}\normalfont}

% ----------------------------------------------------------
% ELEMENTOS PRÉ-TEXTUAIS
% ----------------------------------------------------------
% ---
% Capa
% ---
%imprimircapa
% 26/03/2021 ----------------
%Capa do ICMC
%% USPSC-CapaICMC.tex
\AddToShipoutPicture{\BackgroundPic}
%-------------
\begin{minipage}[c]{144mm}
   \centering
   \begin{textblock*}{144mm}(61mm,65mm)
   \vspace*{1,2cm}
   \linespread{0.5}
   \ABNTEXchapterfont\bfseries\Large
   \textcolor{capa-azul}{\nohyphens{\imprimirtitulo}}
   \end{textblock*}
\end{minipage}
\vfill
\vspace*{5cm}
\begin{minipage}[t][65mm][t]{125mm}
   \begin{textblock*}{130mm}(81mm,123mm)
      \ABNTEXchapterfont\bfseries\Large
      \textcolor{capa-azul}{\nohyphens{\imprimirautor}} 
      \vfill
      \vspace{9pt}
      \ABNTEXsubsectionfontsize\small
      \renewcommand{\ABNTEXsubsectionfontsize}{\fontsize{10}{6}\normalfont}
      \ABNTEXsubsectionfontsize 
      \textcolor{capa-azul}{\nohyphens{\imprimirnotacapaicmc}}
      \renewcommand{\ABNTEXsubsectionfontsize}{\fontsize{12}{12}\normalfont}
   \end{textblock*}	
\end{minipage} 
% ---

\AddToShipoutPicture{\BackgroundBranco}
% ----------------
% ---
% Folha de rosto do ICMC
% (o * indica impressão em anverso (frente) e verso )
% ---
%\imprimirfolharostocar
%\imprimirfolharostocar*

% Folha de rosto padrão do Pacote USPSC
% (o * indica impressão em anverso (frente) e verso )
% ---
%\imprimirfolhaderosto
% ---
% ---
% Inserir a ficha catalográfica em pdf
% ---
% A biblioteca da sua Unidade lhe fornecerá um PDF com a ficha
% catalográfica definitiva. 
% Quando estiver com o documento, salve-o como PDF no diretório
% do seu projeto como fichacatalografica.pdf e inclua o arquivo
% utilizando o comando abaixo:

% Se você optar por elaborar a ficha catalográfica, deverá 
% incluir uma % antes da linha % antes
% do comando %% USPSC-fichacatalografica.tex
% ---
% Inserir a ficha bibliografica
% ---
% Isto é um exemplo de Ficha Catalográfica, ou ``Dados internacionais de
% catalogação-na-publicação''. Você pode utilizar este modelo como referência. 
% Porém, provavelmente a biblioteca da sua universidade lhe fornecerá um PDF
% com a ficha catalográfica definitiva após a defesa do trabalho. Quando estiver
% com o documento, salve-o como PDF no diretório do seu projeto e substitua todo
% o conteúdo de implementação deste arquivo pelo comando abaixo:
%
\begin{fichacatalografica}
	\hspace{-1.4cm}
	\imprimirnotaautorizacao \\ \\
	%\sffamily
	\vspace*{\fill}					% Posição vertical
\begin{center}					% Minipage Centralizado
  \imprimirnotabib \\
  \begin{table}[htb]
	\scriptsize
	\centering	
	\begin{tabular}{|p{0.9cm} p{8.7cm}|}
		\hline
	      & \\
		  &	  \imprimirautorficha     \\
		
		 \imprimircutter & 
							\hspace{0.4cm}\imprimirtitulo~  / ~\imprimirautor~ ;  ~\imprimirorientadorcorpoficha. -- 	\imprimirlocal, \imprimirdata.   \\
		
		  &  % Para incluir nota referente à versão corrigida no corpo da ficha,
			  % incluir % no início da linha acima e tirar a % do início da linha abaixo
			  %	\hspace{0.4cm} \imprimirtitulo~  / ~\imprimirautor~ ; ~\imprimirorientadorcorpoficha~- ~\imprimirnotafolharosto. -- \imprimirlocal, \imprimirdata.  \\
		
			\hspace{0.4cm}\pageref{LastPage} p. : il. (algumas color.) ; 30 cm.\\ 
		  & \\
		  & 
		    \hspace{0.4cm}\imprimirnotaficha ~--~ 
						  \imprimirunidademin, 
						  \imprimiruniversidademin, 
		                  \imprimirdata. \\ 
		  & \\                 
		   % Para incluir nota referente à versão corrigida em notas,
		    % incluir uma % no início da linha acima e	
		    % tirar a % do início da linha abaixo
		    % & \hspace{0.4cm}\imprimirnotafolharosto \\ 
		  & \\ 
		  & \hspace{0.4cm}1. LaTeX. 2. abnTeX. 3. Classe USPSC. 4. Editoração de texto. 5. Normalização da documentação. 6. Tese. 7. Dissertação. 8. Documentos (elaboração). 9. Documentos eletrônicos. I. \imprimirorientadorficha. 
		   II. Título. \\
	
		     %Se houver co-orientador, inclua % antes da linha (antes de II. Título.) 
		     %          e tire a % antes do comando abaixo 
		     %III. Título. \\   
		  \hline
	\end{tabular}
  \end{table}
\end{center}
\end{fichacatalografica}
% ---

 
% e retirar o % do comando abaixo
%%% USPSC-fichacatalografica.tex
% ---
% Inserir a ficha bibliografica
% ---
% Isto é um exemplo de Ficha Catalográfica, ou ``Dados internacionais de
% catalogação-na-publicação''. Você pode utilizar este modelo como referência. 
% Porém, provavelmente a biblioteca da sua universidade lhe fornecerá um PDF
% com a ficha catalográfica definitiva após a defesa do trabalho. Quando estiver
% com o documento, salve-o como PDF no diretório do seu projeto e substitua todo
% o conteúdo de implementação deste arquivo pelo comando abaixo:
%
\begin{fichacatalografica}
	\hspace{-1.4cm}
	\imprimirnotaautorizacao \\ \\
	%\sffamily
	\vspace*{\fill}					% Posição vertical
\begin{center}					% Minipage Centralizado
  \imprimirnotabib \\
  \begin{table}[htb]
	\scriptsize
	\centering	
	\begin{tabular}{|p{0.9cm} p{8.7cm}|}
		\hline
	      & \\
		  &	  \imprimirautorficha     \\
		
		 \imprimircutter & 
							\hspace{0.4cm}\imprimirtitulo~  / ~\imprimirautor~ ;  ~\imprimirorientadorcorpoficha. -- 	\imprimirlocal, \imprimirdata.   \\
		
		  &  % Para incluir nota referente à versão corrigida no corpo da ficha,
			  % incluir % no início da linha acima e tirar a % do início da linha abaixo
			  %	\hspace{0.4cm} \imprimirtitulo~  / ~\imprimirautor~ ; ~\imprimirorientadorcorpoficha~- ~\imprimirnotafolharosto. -- \imprimirlocal, \imprimirdata.  \\
		
			\hspace{0.4cm}\pageref{LastPage} p. : il. (algumas color.) ; 30 cm.\\ 
		  & \\
		  & 
		    \hspace{0.4cm}\imprimirnotaficha ~--~ 
						  \imprimirunidademin, 
						  \imprimiruniversidademin, 
		                  \imprimirdata. \\ 
		  & \\                 
		   % Para incluir nota referente à versão corrigida em notas,
		    % incluir uma % no início da linha acima e	
		    % tirar a % do início da linha abaixo
		    % & \hspace{0.4cm}\imprimirnotafolharosto \\ 
		  & \\ 
		  & \hspace{0.4cm}1. LaTeX. 2. abnTeX. 3. Classe USPSC. 4. Editoração de texto. 5. Normalização da documentação. 6. Tese. 7. Dissertação. 8. Documentos (elaboração). 9. Documentos eletrônicos. I. \imprimirorientadorficha. 
		   II. Título. \\
	
		     %Se houver co-orientador, inclua % antes da linha (antes de II. Título.) 
		     %          e tire a % antes do comando abaixo 
		     %III. Título. \\   
		  \hline
	\end{tabular}
  \end{table}
\end{center}
\end{fichacatalografica}
% ---


% As informações que compõem a ficha catalográfica estão 
% definidas no arquivo USPSC-pre-textual-UUUU.tex
% ---

% ---
% Folha de rosto adicional
% Para imprimir a folha de rosto adicional, exigida por algumas Unidades, a exemplo do ICMC,
% retire a % antes do comando abaixo
% ---


% ---
% Inserir errata
% ---

% ---
% Inserir folha de aprovação
% ---

% A Folha de aprovação é um elemento obrigatório da NBR 4724/2011 (seção 4.2.1.3). 
% Após a defesa/aprovação do trabalho, gere o arquivo folhadeaprovacao.pdf da página assinada pela banca 
% e iclua o arquivo utilizando o comando abaixo:
% Alternativa para a Folha de Aprovação:
% Se for a sua opção elaborar uma folha de aprovação, insira uma % antes do comando acima que inclui o arquivo folhadeaprovacao.pdf,
% tire o % do comando abaixo e altere o arquivo folhadeaprovacao.tex conforme suas necessidades
%\include{folhadeaprovacao}

% A T E N Ç Ã O
% Se o idioma do texto for em inglês, o abstract deve preceder o resumo
% resumo em português
%
% Resumo
% ---
%% USPSC-Resumo.tex
\setlength{\absparsep}{18pt} % ajusta o espaçamento dos parágrafos do resumo		
\begin{resumo}
	\begin{flushleft} 
			\setlength{\absparsep}{0pt} % ajusta o espaçamento da referência	
			\SingleSpacing 
			\imprimirautorabr~~\textbf{\imprimirtituloresumo}.	\imprimirdata. \pageref{LastPage} p. 
			%Substitua p. por f. quando utilizar oneside em \documentclass
			%\pageref{LastPage} f.
			\imprimirtipotrabalho~-~\imprimirinstituicao, \imprimirlocal, \imprimirdata. 
 	\end{flushleft}
\OnehalfSpacing 			
 O resumo deve ressaltar o  objetivo, o método, os resultados e as conclusões do documento. A ordem e a extensão  destes itens dependem do tipo de resumo (informativo ou indicativo) e do  tratamento que cada item recebe no documento original. O resumo deve ser
 precedido da referência do documento, com exceção do resumo inserido no
 próprio documento. (\ldots)  Salientamos que algumas Unidades exigem o titulo dos trabalhos acadêmicos em inglês, tornando necessário a inclusão das referências nos resumos e abstracts, o que foi adotado no \textbf{Modelo para TCC em \LaTeX\ utilizando a classe USPSC} e no \textbf{Modelo para teses e dissertações em \LaTeX\ utilizando a classe USPSC}. As palavras-chave devem figurar logo abaixo do  resumo, antecedidas da expressão Palavras-chave:, separadas entre si por  ponto e finalizadas também por ponto \cite{nbr6028}.
 

 \textbf{Palavras-chave}: LaTeX. Classe USPSC. Tese. Dissertação. Trabalho de conclusão de curso (TCC). Relatório.
\end{resumo}
% ---

%% USPSC-Introducao.tex

% ----------------------------------------------------------
% Introdução (exemplo de capítulo sem numeração, mas presente no Sumário)
% ----------------------------------------------------------
\chapter[Introdução]{Introdução}
\label{Introdução}


Parte inicial do texto, que contém a delimitação do assunto tratado, objetivos da pesquisa e outros elementos necessários para apresentar o tema do trabalho \cite{aguia2020}.


%% USPSC-Cap2-Desenvolvimento.tex 

% ---
% Este capítulo, utilizado por diferentes exemplos do abnTeX2, ilustra o uso de
% comandos do abnTeX2 e de LaTeX.
% ---

\chapter{Desenvolvimento}\label{cap_exemplos}
Este capítulo é parte principal do trabalho acadêmico e deve conter a exposição ordenada e detalhada do assunto. Divide-se em seções e subseções, em conformidade com a abordagem do tema e do método, abrangendo: revisão bibliográfica, materiais e métodos, técnicas utilizadas, resultados obtidos e discussão.

Abaixo são apresentados minimamente exemplos tabelas, quadros, divisões de documentos e outros itens. Consulte o \textbf{Tutorial do Pacote USPSC para modelos de trabalhos de acad\^emicos em LaTeX - vers\~ao 3.2} para demais informações. 

\section{Resultados de comandos}\label{sec-divisoes}

% ---
\subsection{Tabelas e quadros}

O \textbf{Tutorial do Pacote USPSC para modelos de trabalhos de acad\^emicos em LaTeX - vers\~ao 3.2} apresenta orientações completas e diversas formatações de tabelas, dentre elas a \autoref{tab-ibge}, que é um exemplo de tabela alinhada que pode ser longa ou curta, conforme padrão do Instituto Brasileiro de Geografia e Estatística (IBGE).

%\begin{table}[H]
\begin{table}[htb]
	\IBGEtab{%
		\caption{Frequência anual por categoria de usuários}%
		\label{tab-ibge}
	}{%
		\begin{tabular}{ccc}
			\toprule
			Categoria de Usuários & Frequência de Usuários \\
			\midrule \midrule
			Graduação & 72\% \\
			\midrule 
			Pós-Graduação & 15\% \\
			\midrule 
			Docente & 10\% \\
			\midrule 
			Outras & 3\% \\
			\bottomrule
		\end{tabular}%
	}{%
		\fonte{Elaborada pelos autores.}%
		\nota{Exemplo de uma nota.}%
		\nota[Anotações]{Uma anotação adicional, que pode ser seguida de várias
			outras.}%
		
	}
\end{table}


A formatação do quadro é similar à tabela, mas deve ter suas laterais fechadas e conter as linhas horizontais.
\newpage

% o comando \newpage foi utilizado para forçar a quebra de página

\begin{quadro}[htb]
	\caption{\label{quadro_modelo}Níveis de investigação}
	\begin{tabular}{|p{2.6cm}|p{6.0cm}|p{2.25cm}|p{3.40cm}|}
		\hline
		\textbf{Nível de Investigação} & \textbf{Insumos}  & \textbf{Sistemas de Investigação}  & \textbf{Produtos}  \\
		\hline
		Meta-nível & Filosofia\index{filosofia} da Ciência  & Epistemologia &
		Paradigma  \\
		\hline
		Nível do objeto & Paradigmas do metanível e evidências do nível inferior &
		Ciência  & Teorias e modelos \\
		\hline
		Nível inferior & Modelos e métodos do nível do objeto e problemas do nível inferior & Prática & Solução de problemas \\ 
		\hline
	\end{tabular}
	\begin{flushleft}
		%\fonte{\citeonline{van1986}}
		Fonte: \citeonline{van1986}
	\end{flushleft}
\end{quadro} 


No \textbf{Tutorial do Pacote USPSC para modelos de trabalhos de acad\^emicos em LaTeX - vers\~ao 3.2} são apresentados mais exemplos de quadros.

% ---
\subsection{Figuras}\label{sec_figuras}
% ---
\index{figuras}Figuras podem ser criadas diretamente em \LaTeX,
como o exemplo da \autoref{fig_circulo}.

\begin{figure}[htb]
	\caption{\label{fig_circulo}A delimitação do espaço}
	\begin{center}
		\setlength{\unitlength}{9cm}
		\begin{picture}(1,1)
		\put(0,0){\line(0,1){1}}
		\put(0,0){\line(1,0){1}}
		\put(0,0){\line(1,1){1}}
		\put(0,0){\line(1,2){.5}}
		\put(0,0){\line(1,3){.3333}}
		\put(0,0){\line(1,4){.25}}
		\put(0,0){\line(1,5){.2}}
		\put(0,0){\line(1,6){.1667}}
		\put(0,0){\line(2,1){1}}
		\put(0,0){\line(2,3){.6667}}
		\put(0,0){\line(2,5){.4}}
		\put(0,0){\line(3,1){1}}
		\put(0,0){\line(3,2){1}}
		\put(0,0){\line(3,4){.75}}
		\put(0,0){\line(3,5){.6}}
		\put(0,0){\line(4,1){1}}
		\put(0,0){\line(4,3){1}}
		\put(0,0){\line(4,5){.8}}
		\put(0,0){\line(5,1){1}}
		\put(0,0){\line(5,2){1}}
		\put(0,0){\line(5,3){1}}
		\put(0,0){\line(5,4){1}}
		\put(0,0){\line(5,6){.8333}}
		\put(0,0){\line(6,1){1}}
		\put(0,0){\line(6,5){1}}
		\end{picture}
	\end{center}
	\legend{Fonte: \citeonline{equipeabntex2}}
\end{figure}

Consulte o \textbf{Tutorial do Pacote USPSC para modelos de trabalhos de acad\^emicos em LaTeX - vers\~ao 3.2} para conhecer mais recursos referentes à figuras. 

% ---
\section{Divisões do documento}\label{sec-divisoes-b}
Esta seção exemplifica o uso de divisões de documentos em conformidade com a ABNT NBR 6024  \cite{nbr6024}.
% ---
% ---
\subsection{Divisões do documento: subseção}\label{sec-divisoes-subsection}
% ---

Um exemplo de seção é a \autoref{sec-divisoes-b}. Esta é a \autoref{sec-divisoes-subsection}.

\subsubsection{Divisões do documento: subsubseção}\label{sec-divisoes-subsubsection}

Isto é uma \texttt{subsubsection} do \LaTeX, mas é denominada de ``subseção'' porque no português não temos a palavra ``subsubseção''.

\subsubsection{Divisões do documento: subsubseção}

Isto é outra subsubseção.

\subsection{Divisões do documento: subseção}\label{sec-exemplo-subsec}

Isto é uma subseção.

\subsubsection{Divisões do documento: subsubseção}

Isto é mais uma subsubseção da \autoref{sec-exemplo-subsec}.


\subsubsubsection{Esta é uma subseção de quinto
nível}\label{sec-exemplo-subsubsubsection}

Esta é uma seção de quinto nível. Ela é produzida com o seguinte comando:

\begin{verbatim}
\subsubsubsection{Esta é uma subseção de quinto
nível}\label{sec-exemplo-subsubsubsection}
\end{verbatim}

\subsubsubsection{Esta é outra subseção de quinto nível}\label{sec-exemplo-subsubsubsection-outro}

Esta é outra seção de quinto nível.


\paragraph{Este é um parágrafo numerado}\label{sec-exemplo-paragrafo}

Este é um exemplo de parágrafo numerado. Ele é produzido com o comando de
parágrafo:

\begin{verbatim}
\paragraph{Este é um parágrafo numerado}\label{sec-exemplo-paragrafo}
\end{verbatim}

A numeração entre parágrafos numerados e subsubsubseções são contínuas.

\paragraph{Esta é outro parágrafo numerado}\label{sec-exemplo-paragrafo-outro}

Este é outro parágrafo numerado.

% ---
\subsection{Este é um exemplo de nome de subseção longa que se aplica a seções e demais divisões do documento. Ele deve estar alinhado à esquerda e a segunda e demais linhas devem iniciar logo abaixo da primeira palavra da primeira linha} 

Observe que o alinhamento do título obedece esta regra também no sumário.
	








%% USPSC-Cap3-Conclusao.tex
% ---
% Conclusão
% ---
\chapter{Conclusão}
% ---
% O comando abaixo insere parágrafos aleatórios só para exemplificar
Apresentar as conclusões correspondentes aos objetivos ou hipóteses propostos para o desenvolvimento do trabalho, podendo incluir sugestões para novas pesquisas.



\bibliography{USPSC-bib/USPSC-modelo-references}
\end{document}
