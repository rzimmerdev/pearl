%%%%%%%%%%%%%%%%%%%%%%%%%%%%%%%%%%%%%%%%%%%%%%%%%%%%%%%%%%
% Template para redação de Teses/Dissertações da ESALQ/USP
% Autor: Antonio Augusto Franco Garcia
% Conversão para capítulos: Fernando de Pol Mayer
%%%%%%%%%%%%%%%%%%%%%%%%%%%%%%%%%%%%%%%%%%%%%%%%%%%%%%%%%%

% Este arquivo concatena todos os arquivos individuais e o template

% ESTE É O ARQUIVO QUE DEVE SER COMPILADO!!!!!

% AS CONFIGURAÇÕES NESTE ARQUIVO NÃO DEVEM SER ALTERADAS
% INADVERTIDAMENTE. MUDE APENAS O QUE FOR EXPLICITAMENTE INDICADO.

% Preâmbulo: definição da classe e inclusão do template
% NÃO MUDE NADA AQUI, NEM O TAMANHO DA FONTE
\documentclass[a4paper,10pt,twoside,oldfontcommands]{memoir}
\usepackage{./template/Template_Tese}


% Inclua aqui a lista dos pacotes do LaTeX que você necessita usar.
% Por exemplo, para incluir o pacote amsmath, remova o comentário da
% linha correspondente logo abaixo. Insira os demais pacotes de forma
% análoga.

% ATENÇÃO: alguns pacotes são incompatíveis entre si, ou se sobrepõem
% as configurações de outros; use com cuidado
% amsmath: \usepackage{amsmath}

%% O pacote bibtopic é responsável por fazer as referências em cada
%% capítulo. Ele deve ser chamado aqui no preâmbulo e também no início e
%% final de cada capítulo.
\raggedbottom
\usepackage{bibtopic}


%%%%%%%%%%%%%%%%%%%%%%%%%%%%%%%%%%%%%%%%%
% Início do documento - Parte Pré-textual
\begin{document}

% Macro da classe memoir que remove os cabeçalhos (comuns em
% documentos da classe "book")
\pagestyle{fnsizeheadings}

% Inclusão do arquivo com informações necessárias para a parte
% pré-textual.
% %%%%
% VOCÊ DEVE ABRIR O ARQUIVO TodasInformações.tex E
% ALTERÁ-LO DA FORMA APROPRIADA PARA O SEU TRABALHO.
% %%%%
% Lembre-se de  conferir o número de páginas quando tiver a versão final a ser
% submetida.
\input{./pre-textual/TodasInformacoes.tex}


% Inclusão de arquivos que compõem a parte pré-textual

% Inclusão das Capas
% NÃO HÁ NADA A SER ALTERADO NO ARQUIVO ABAIXO
% A capa é produzida automaticamente

%%%%%%%%%%
% Página de rosto, com orientador
\cleardoublepage
\pagenumbering{arabic}  % reinicia a numeração de página
\thispagestyle{empty} % Não mostra o número na Página de Rosto

\begin{center}
  { \fontsize{14}{14} \sffamily \bfseries
    \Autor\\
  }
\end{center}

\vfill % Joga o cabeçalho no topo da página

% Título
\vspace{10pt}
\begin{center}
  \begin{center}
    { \fontsize{14}{14} \sffamily \bfseries
      \TítuloDoTrabalho
    }\\
    {\small\sffamily{\Revisada}}
  \end{center}

% Orientador
  \vspace{60pt}
  \begin{flushright}
    \begin{minipage}{0.5\textwidth}
      { \fontsize{10}{10} \sffamily
        Orientador:\\
        \DoutorOuDoutora\:
        \textbf{\MakeUppercase{\Orientador}}
      }
    \end{minipage}
  \end{flushright}

% Título obtido
  \vspace{20pt}
  \begin{flushright}
    \begin{minipage}{0.5\textwidth}
      { \fontsize{10}{10} \sffamily
        \TítuloObtido
      }
    \end{minipage}
  \end{flushright}

  \vfill

% Parte de baixo da página
  \begin{center}
    { \fontsize{14}{14} \sffamily \bfseries
      São Paulo\\
      \AnoDepósito
    }
  \end{center}

\end{center} 

\newpage

% Duas configurações importantes que não devem ser removidas:
% Paginação constante a partir da Ficha Catalográfica, fontes roman
\openany
\rmfamily

% Para remover partes opcionais indicadas abaixo, basta comentar as
% respectivas linhas (incluindo o \clearpage). Para editar, abra o
% arquivo correspondente.

% Resumo. Abra o arquivo para alterações.
% Não altere nada nesta parte
\chapter*{\hspace{7.55cm}{RESUMO}}
\addcontentsline{toc}{chapter}{Resumo}

\begin{center}
  {\bfseries \sffamily \fontsize{12}{12} \TítuloDoTrabalho}
\end{center}


%%%%
% Insira aqui o seu resumo em português
\begin{abstract}
  O presente trabalho propõe uma técnica de modelagem para simuladores de mercado financeiro, especificamente das dinâmicas de livros de ordens limite. A utilização de simuladores se revela de extrema importância para testar estratégias de negociação sem depender de grandes volumes de dados históricos, especialmente para estratégias cujo impacto no estado do livro de ordens não é desconsiderável e também não é refletido ao se utilizar ordens históricas e estáticas.
  
  O simulador a ser desenvolvido é fundamentado na teoria de modelos ocultos de Markov (\textit{Hidden Markov Chains}, ou \textit{HMC}, em inglês) para modelar o comportamento dinâmico do livro de ordens limite. Além disso, serão comparadas diferentes distribuições para os processos de chegada de ordens e suas intensidades, especificamente os Processos de Hawkes, Cadeias de Markov Autoregressivas e Modelos de Hawkes Dependentes do Estado. Por fim, a principal contribuição do trabalho introduzir o uso de algoritmos genéticos como técnica para aproximar as distribuições dos processos de chegada de ordens dependentes dos estados, utilizando-se estimativas por máxima verossimilhança para comparar com outras técnicas utilizadas na literatura. 
  
  A abordagem proposta visa melhorar a precisão das simulações e fornecer insights mais robustos sobre o comportamento do mercado financeiro em diferentes cenários.
  O uso de algoritmos evolutivos tem como objetivo contribuir no desenvolvimento e na melhor compreensão de modelos simuladores do mercado financeiro, assim como o desenvolvimento de estratégias de negociação mais eficientes e resilientes a diferentes regimes de mercado.
\end{abstract}

% Não altere a seguir. As palavras-chave estão no arquivo
% TodasInformações.tex e são usadas em mais de uma parte do texto.
% Abra tal arquivo para eventuais alterações.
\begin{abstract}
\vspace{-.75cm}
\noindent\textbf{Palavras-chave:} \PalavrasChave
\noindent\textbf{JEL:} \JELCodes
\end{abstract}

\clearpage

%%%%%%%%%%%%%%%%%%%%%%%%%%%%%%%%%%%%%%%%%%%%%%%%%%%
% Início da Parte Textual. Conteúdo mais importante

% A opção abaixo, que não deve ser removida, inicia cada capítulo
% em página ímpar
\openright

% Os items abaixo dependem do tipo de trabalho: capítulos, texto
% convencional, etc, de acordo com o Regulamento. Abra os arquivos
% abaixo (estão no diretório chamado "textual") para ver como os
% modelos foram criados. Basicamente, é uma edição usando LaTeX, sem
% modificações substanciais. Modifique o conteúdo dos arquivos, insira
% novos .tex, etc.

\begin{btUnit}
\chapter{Introdução}

O livro de ordens limite representa o estado atual de todas ordens aguardando execução. Eseencialmente fornece uma representação dinâmica das intenções de compra e venda de ativos financeiros em um determinado mercado por meio de uma lista de ordens, organizadas por preço e quantidade, onde cada entrada representa o desejo de um participante do mercado de comprar ou vender um certo ativo. Os intervalos de tempo entre a chegada de duas ordens subsequentes podem ser representados por processos de chegada e são fundamentais para modelar e representar formalmente a dinâmica do mercado financeiro. Tradicionalmente, os modelos de chegada de ordens têm se baseado em processos de contagem, onde os incrementos entre eventes são distribuídos de acordo com distribuições de Poisson ou distribuições exponenciais. No entanto, recentemente, houve um interesse crescente em modelar os processos de chegada de ordens usando processos pontuais autoexcitáveis, especificamente processos de Hawkes. 

Além das diferentes características de processos individuais de chegada, o mercado financeiro é caracterizado por diferentes regimes, seja em escala macro como regimes de inflação, como em regimes de curto prazo, que podem ser definidos como períodos de tempo em que as características do mercado, como volatilidade, liquidez e volume de negociação, exibem comportamentos distintos. Esses regimes podem ser influenciados por uma variedade de fatores, incluindo eventos econômicos, políticos e sazonais.

Uma abordagem promissora para modelar o livro de ordens e os processos de chegada de ordens em diferentes regimes é o uso de Hidden Markov Chains (HMC). As HMCs são modelos estatísticos que assumem a existência de estados ocultos, não observáveis diretamente, mas que podem ser inferidos a partir de observações visíveis. Nesse contexto, os diferentes estados do livro de ordens e as diferentes intensidades de chegada de ordens podem ser representados como estados ocultos da cadeia de Markov, permitindo modelar a transição entre diferentes regimes de mercado de forma dinâmica.

Além disso, os algoritmos genéticos têm se destacado como uma ferramenta poderosa para otimização e modelagem em uma variedade de domínios, incluindo finanças. Esses algoritmos podem ser aplicados para aproximar as possíveis distribuições dos processos de chegada de ordens, utilizando estimativas por máxima verossimilhança para ajustar os parâmetros do modelo aos dados históricos observados. Isso permite replicar as características estatísticas dos dados históricos no simulador, proporcionando uma base sólida para a geração de cenários realistas de mercado.

Em suma, este trabalho propõe uma abordagem inovadora para modelar e simular o livro de ordens limite e os processos de chegada de ordens no mercado financeiro, utilizando Hidden Markov Chains e algoritmos genéticos. Essa abordagem tem o potencial de melhorar significativamente a precisão das simulações e fornecer insights mais robustos sobre o comportamento do mercado em diferentes regimes, contribuindo assim para o desenvolvimento de estratégias de negociação mais eficientes e resilientes.


\end{btUnit}

O tema deste pré-projeto de pesquisa é a aplicação de algoritmos genéticos na modelagem e simulação de regimes de mercado, com foco específico nos livros de ordens limite. Algoritmos genéticos são métodos de otimização baseados em princípios evolutivos, que buscam encontrar soluções para problemas complexos através da seleção, recombinação e mutação de indivíduos em uma população. Neste contexto, os algoritmos genéticos são utilizados para aproximar as distribuições dos processos estocásticos que descrevem a chegada de ordens limite em um mercado financeiro.

Entender os diferentes regimes de mercado é fundamental para os participantes do mercado financeiro, pois eles influenciam diretamente as estratégias de negociação, o gerenciamento de risco e os resultados financeiros. Um regime de mercado refere-se aos padrões observados nos preços, volume de negociação, volatilidade e outros indicadores ao longo do tempo. Compreender e replicar esses regimes em simulações é essencial para testar estratégias de investimento, desenvolver modelos de precificação de ativos e entender o comportamento do mercado em diferentes condições.

Os livros de ordens limite desempenham um papel fundamental nos mercados financeiros, pois representam a oferta e a demanda de ativos em um determinado momento. Eles exibem as ordens de compra e venda, juntamente com seus respectivos preços e quantidades, permitindo que os participantes do mercado visualizem a liquidez disponível e as intenções de negociação dos outros participantes. Modelar e simular os livros de ordens limite é essencial para entender como as ordens são executadas, como a liquidez é formada e como os preços são determinados no mercado.

A utilização de algoritmos genéticos para modelagem e simulação de regimes de mercado oferece várias vantagens. Esses algoritmos são capazes de lidar com problemas complexos e não-lineares, e podem encontrar soluções aproximadas eficientemente. Além disso, eles podem ser adaptados para incorporar diferentes fontes de informação e considerar múltiplos objetivos, tornando-os adequados para uma ampla gama de aplicações financeiras.

A pesquisa proposta visa explorar como os algoritmos genéticos podem ser utilizados para modelar e simular os regimes de mercado, com foco particular nos livros de ordens limite. Pretende-se investigar como esses algoritmos podem ser adaptados e calibrados para melhor representar os processos estocásticos subjacentes, e como as simulações resultantes podem ser usadas para entender o comportamento do mercado, testar estratégias de negociação e avaliar o risco.

Esta pesquisa contribuirá para o avanço do conhecimento sobre a aplicação de algoritmos genéticos em finanças, fornecendo insights sobre como esses métodos podem ser utilizados para modelar e simular os regimes de mercado. Os resultados esperados incluem a identificação de padrões e regularidades nos dados do mercado, a avaliação da eficácia das simulações geradas e a análise do impacto das estratégias de negociação sob diferentes condições de mercado.

Em última análise, espera-se que esta pesquisa ajude os participantes do mercado financeiro a tomar decisões mais informadas e a desenvolver estratégias de investimento mais robustas e eficazes. Ao entender melhor os regimes de mercado e as dinâmicas do livro de ordens limite, os investidores estarão melhor preparados para enfrentar os desafios e aproveitar as oportunidades nos mercados financeiros.

\chapter{Objetivos do Trabalho}
Este trabalho visa comparar o desempenho de Algoritmos Genéticos (AGs) e Métodos de Monte Carlo para Cadeias de Markov (MCMC) na otimização da amostragem das distribuições dos processos de chegada de ordens no mercado financeiro, com foco nos resultados computacionais das simulações e de treino, especificamente os tempos de execução para teste, geração de dados e tempo para realização dos testes estatísticos, tanto em CPU, quanto em unidades de processamento gráficas (\textit{GPU}) utilizando bibliotecas de computação vetorial. A primeira abordagem utiliza AGs para ajustar os parâmetros das distribuições aos dados históricos, enquanto a segunda emprega MCMC, como Metropolis-Hastings e Gibbs sampling, para amostrar diretamente das distribuições. O objetivo é avaliar a precisão e eficácia de cada método na replicação das características estatísticas dos processos de chegada em dados históricos, fornecendo insights para estratégias de negociação algorítmicas mais robustas.
\chapter{Metodologia}

Para atingir os objetivos propostos, adotaremos uma metodologia em duas etapas. Primeiramente, será implementado um simulador de mercado com dados históricos estáticos disponibilizados pela bolsa de Nova Yorque (NASDAQ). Em sequência, o simulador será modelado para utilizar um estimador das distribuições de chegada de eventos, especificamente com duas abordagens a serem comparadas: Algoritmos Genéticos (AGs) e Métodos de Monte Carlo para Cadeias de Markov (MCMC) como modelos dos processos de chegada de ordens. Utilizaremos linguagens de programação adequadas para a implementação eficiente desses métodos, em Python com bibliotecas de processamento vetorial, especificamente NumPy, PyMC3 e JAX.

\begin{enumerate}
	\item Na primeira etapa, aplicaremos AGs para otimizar a representação das distribuições dos processos de chegada de ordens. Isso envolverá a definição de uma função de aptidão (fitness) que quantifique a adequação de uma determinada distribuição aos dados históricos. Os parâmetros das distribuições serão ajustados iterativamente através de operadores genéticos, como seleção, recombinação e mutação, para maximizar a função de aptidão.
	
	\item Na segunda etapa, implementaremos e utilizaremos métodos de MCMC, como Metropolis-Hastings e Gibbs sampling, para amostrar diretamente das distribuições dos processos de chegada. Faremos isso construindo cadeias de Markov com distribuição estacionária igual à distribuição alvo. Avaliaremos a convergência das cadeias de Markov e a qualidade das amostras geradas em relação aos dados históricos.
\end{enumerate}

Por fim, conduziremos experimentos computacionais para comparar o desempenho e a eficácia dos dois métodos em termos de precisão na representação dos processos de chegada de ordens. Analisaremos métricas relevantes, como erro médio, tempo de convergência e robustez em diferentes cenários de mercado.
\chapter{Cronograma}

A execução da pesquisa para o trabalho de conclusão do curso será realizada de acordo com o seguinte cronograma, para o Projeto de Graduação I, e será continuada no Projeto de Graduação II no primeiro semestre de 2025.

\begin{table}[h]
	\begin{tcolorbox}[width=\textwidth]
	\centering
	\renewcommand{\arraystretch}{1.5}
	\begin{adjustbox}{width=\textwidth}
	\begin{tabular}{|c|c|c|c|c|c|}
		\hline
		\textbf{Etapa} & 
		\textbf{1. Julho} & 
		\textbf{2. Agosto} & 
		\textbf{3. Setembro} & 
		\textbf{4. Outubro} & 
		\textbf{5. Novembro} \\ \hline
		1. Pesquisa Bibliográfica & X & &  &  &  \\ \hline
		2. Análise Bibliográfica & X & X &  &  &  \\ \hline
		3. Obtenção de Dataset &  & X &  &  &  \\ \hline
		4. Definição dos modelos &  & X & X & X &  \\ \hline
		5. Implementação em código &  &  & X & X & X \\ \hline
		6. Análise das métricas &  &  & & X & X \\ \hline
		7. Escrita do artigo &  & & X & X & X \\ \hline
	\end{tabular}
\end{adjustbox}
\end{tcolorbox}
\end{table}
\begin{btSect}[bibliography/apalikept]{bibliography/bibliography}
	\section*{Referências}
	\addcontentsline{toc}{section}{Referências}
	\btPrintCited
\end{btSect}

\end{document}
