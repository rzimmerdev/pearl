%%%%%%%%%%%%%%%%%%%%%%%%%%%%%%%%%%%%%%%%%%%%%%%%%%%%%%%%%%
% Template para redação de Teses/Dissertações da ESALQ/USP
% Autor: Antonio Augusto Franco Garcia
% Conversão para capítulos: Fernando de Pol Mayer
%%%%%%%%%%%%%%%%%%%%%%%%%%%%%%%%%%%%%%%%%%%%%%%%%%%%%%%%%%

% Este arquivo concatena todos os arquivos individuais e o template

% ESTE É O ARQUIVO QUE DEVE SER COMPILADO!!!!!

% AS CONFIGURAÇÕES NESTE ARQUIVO NÃO DEVEM SER ALTERADAS
% INADVERTIDAMENTE. MUDE APENAS O QUE FOR EXPLICITAMENTE INDICADO.

% Preâmbulo: definição da classe e inclusão do template
% NÃO MUDE NADA AQUI, NEM O TAMANHO DA FONTE
\documentclass[a4paper,10pt,twoside,oldfontcommands]{memoir}
\usepackage{./template/Template_Tese}


% Inclua aqui a lista dos pacotes do LaTeX que você necessita usar.
% Por exemplo, para incluir o pacote amsmath, remova o comentário da
% linha correspondente logo abaixo. Insira os demais pacotes de forma
% análoga.

% ATENÇÃO: alguns pacotes são incompatíveis entre si, ou se sobrepõem
% as configurações de outros; use com cuidado
% amsmath: \usepackage{amsmath}

%% O pacote bibtopic é responsável por fazer as referências em cada
%% capítulo. Ele deve ser chamado aqui no preâmbulo e também no início e
%% final de cada capítulo.
\raggedbottom
\usepackage{bibtopic}


%%%%%%%%%%%%%%%%%%%%%%%%%%%%%%%%%%%%%%%%%
% Início do documento - Parte Pré-textual
\begin{document}

% Macro da classe memoir que remove os cabeçalhos (comuns em
% documentos da classe "book")
\pagestyle{fnsizeheadings}

% Inclusão do arquivo com informações necessárias para a parte
% pré-textual.
% %%%%
% VOCÊ DEVE ABRIR O ARQUIVO TodasInformações.tex E
% ALTERÁ-LO DA FORMA APROPRIADA PARA O SEU TRABALHO.
% %%%%
% Lembre-se de  conferir o número de páginas quando tiver a versão final a ser
% submetida.
\newcommand{\TítuloDoTrabalho}{%
  Abordagens Numéricas para Simuladores de Processos Estocásticos de Chegada de Ordens em Mercados Financeiros
}

%%% Autor (nome completo por extenso)
% OBS: nome exatamente como cadastrado no Sistema Janus
\newcommand{\Autor}{%
  Rafael Zimmer
}

%%% Qual é seu último nome? (Como você usa nas publicações). Se houver
%%% Júnior, Filho, etc, indique apropriadamente. Ex: Silva Filho.
%%% É possível também usar hífen, como p ex Pizzirani-Kleiner.
%%% O Sobrenome de Fulano de Tal é exemplificado abaixo.
%%% Note a exclusão do "de"
\newcommand{\Sobrenome}{%
  Zimmer}

%%% Qual é seu primeiro nome, nome do meio, e demais partes do
%%% sobrenome (não incluídas acima)? A soma deste item e do anterior
%%% devem formar seu nome completo por extenso. Não abrevie.
\newcommand{\Nome}{%
  Rafael
}

%%% Indique o título da graduação que você possui. Ex: Engenheiro Agrônomo
\newcommand{\TituloDaGraduação}{%
  Bacharel
}

%%%% Indique o nome do seu Orientador (nome completo por extenso)
\newcommand{\Orientador}{%
  Oswaldo Luiz do Valle Costa
}

%%% Seu orientador(a) é Prof. Dr. ou Profa. Dra.? Selecione umas das
%%% linhas abaixo. Não remova as barras invertidas, apenas selecione a
%%% linha apropriada, comentando a outra
\newcommand{\DoutorOuDoutora}{%
  Prof.\ Dr.\
  %Profa.\ Dra.\
}

%%%% Que título você pretende obter? Indique abaixo, alterando o texto
% Há várias opções; adapte aquela adequada para o seu caso.
%
% A maioria dos programas da ESALQ e CENA segue o padrão abaixo:
%
% Dissertação apresentada para obtenção do título de Mestre (ou Mestra)
% em Ciências. Área de concentração: Verificar no site da CPG (sem
% ponto final)
%
% Tese apresentada para obtenção do título de Doutor (ou Doutora) em
% Ciências. Área de concentração: Verificar no site da CPG (sem ponto
% final)
%
% No caso do PPG em Recursos Florestais, selecione abaixo (indicando
% mestrado ou doutorado apropriadamente).
%
% Dissertação (ou Tese) apresentada para obtenção do título de Mestre
% (ou Mestra) ou Doutor (ou Doutora) em Ciências, Programa: Recursos
% Florestais. Opção em: Silvicultura e Manejo Florestal
%
% Dissertação (ou Tese) apresentada para obtenção do título de Mestre
% (ou Mestra) ou Doutor (ou Doutora) em Ciências, Programa: Recursos
% Florestais. Opção em: Tecnologia de Produtos Florestais
%
% Dissertação (ou Tese) apresentada para obtenção do título de Mestre
% (ou Mestra) ou Doutor (ou Doutora) em Ciências, Programa: Recursos
% Florestais. Opção em: Conservação de Ecossistemas Florestais
%
% PPG International em Biologia Celular e Molecular Vegetal
%
% Tese apresentada para obtenção do tĩtulo de Doutor (ou Doutora) em
% Ciências. Programa: Internacional Biologia Celular e Molecular Vegetal
%
\newcommand{\TítuloObtido}{%
  Tese para obtenção do título de Bacharel em Ciências da Computação
}


%%%%% Tese revisada
% É possível entregar versão revisada de acordo com a resolução CoPGr
% 6018 de 2011. Caso você esteja fazendo isso, remova o comentário
% (apenas o símbolo %) das linhas abaixo.
\newcommand{\Revis}{%
  %-- -- versão revisada de acordo com a resolução CoPGr 6018 de 2011.
}
\newcommand{\Revisada}{%
  %versão revisada de acordo com a resolução CoPGr 6018 de 2011
}


%%%% Informe o ano de depósito do trabalho
\newcommand{\AnoDepósito}{%
  2024
}

%%%% Indique o número total de páginas do trabalho
\newcommand{\NumPáginas}{%
  150
}

%%%% Indique se é Dissertação (Mestrado) ou Tese (Doutorado),
%%%% comentando a linha apropriada (escolha somente uma opção, obviamente)
\newcommand{\TipoTrabalho}{%
  Tese (Bacharel)
  %Tese (Doutorado)
}


%%%% Pós-graduação do CENA: se for aluno deste programa, remova o
%%%% comentário da primeira linha abaixo. Isto formatará corretamente
%%%% seu trabalho
\newcommand{\CENA}{%
  %Centro de Energia Nuclear na Agricultura

   % DEIXE ESTA LINHA EM BRANCO LOGO ACIMA; ELA É IMPORTANTE PARA FORMATAÇÃO
}


%%%% Indique as palavras-chave para a FICHA CATALOGRÁFICA
% Cada uma delas deve ser precedida por um número com ponto.
% Siga o exemplo abaixo
% Nome científico: usar itálico
% Todas palavras devem iniciar com letras maiúsculas
\newcommand{\PalavrasChaveFicha}{%
  1. Aaaaaaaaa  2. Bbbbbbbbbb bbb 3. Ccccc 4. \textit{Dddddd eeeeeeee}
  L.\
}


%%% Palavras chave para Resumo e Abstract

% Resumo
% Para seguir as normas, separe por vírgulas
% Obviamente, use a mesma lista indicada acima
% Nome científico: usar itálico
% Todas palavras devem iniciar com letras maiúsculas
\newcommand{\PalavrasChave}{%
  Hidden Markov Chains, Limit Order Book Modelling, High-Frequency Data, Markov Chain Monte Carlo methods, Genetic Algorithms\\
}
\newcommand{\JELCodes}{%
C45; C15; G17
}

%%%% Palavras-chave me inglês, separadas por vírgulas
% Siga as regras para as palavras-chave em português
\newcommand{\Keywords}{%
  Eee, Fff, Ggg, \textit{Hhh iii}
}



% Inclusão de arquivos que compõem a parte pré-textual

% Inclusão das Capas
% NÃO HÁ NADA A SER ALTERADO NO ARQUIVO ABAIXO
% A capa é produzida automaticamente

%%%%%%%%%%
% Página de rosto, com orientador
\cleardoublepage
\pagenumbering{arabic}  % reinicia a numeração de página
\thispagestyle{empty} % Não mostra o número na Página de Rosto

\begin{center}
  { \fontsize{14}{14} \sffamily \bfseries
    \Autor\\
  }
\end{center}

\vfill % Joga o cabeçalho no topo da página

% Título
\vspace{10pt}
\begin{center}
  \begin{center}
    { \fontsize{14}{14} \sffamily \bfseries
      \TítuloDoTrabalho
    }\\
    {\small\sffamily{\Revisada}}
  \end{center}

% Orientador
  \vspace{60pt}
  \begin{flushright}
    \begin{minipage}{0.5\textwidth}
      { \fontsize{10}{10} \sffamily
        Orientador:\\
        \DoutorOuDoutora\:
        \textbf{\MakeUppercase{\Orientador}}
      }
    \end{minipage}
  \end{flushright}

% Título obtido
  \vspace{20pt}
  \begin{flushright}
    \begin{minipage}{0.5\textwidth}
      { \fontsize{10}{10} \sffamily
        \TítuloObtido
      }
    \end{minipage}
  \end{flushright}

  \vfill

% Parte de baixo da página
  \begin{center}
    { \fontsize{14}{14} \sffamily \bfseries
      São Paulo\\
      \AnoDepósito
    }
  \end{center}

\end{center} 

\newpage

% Duas configurações importantes que não devem ser removidas:
% Paginação constante a partir da Ficha Catalográfica, fontes roman
\openany
\rmfamily

% Para remover partes opcionais indicadas abaixo, basta comentar as
% respectivas linhas (incluindo o \clearpage). Para editar, abra o
% arquivo correspondente.

% Resumo. Abra o arquivo para alterações.
% Não altere nada nesta parte
\chapter*{\hspace{7.55cm}{RESUMO}}
\addcontentsline{toc}{chapter}{Resumo}

\begin{center}
  {\bfseries \sffamily \fontsize{12}{12} \TítuloDoTrabalho}
\end{center}


%%%%
% Insira aqui o seu resumo em português
\begin{abstract}
  O presente trabalho propõe uma técnica de modelagem para simuladores de mercado financeiro, especificamente das dinâmicas de livros de ordens limite. A utilização de simuladores se revela de extrema importância para testar estratégias de negociação sem depender de grandes volumes de dados históricos, especialmente para estratégias cujo impacto no estado do livro de ordens não é desconsiderável e também não é refletido ao se utilizar ordens históricas e estáticas.
  
  O simulador a ser desenvolvido é fundamentado na teoria de modelos ocultos de Markov (\textit{Hidden Markov Chains}, ou \textit{HMC}, em inglês) para modelar o comportamento dinâmico do livro de ordens limite. Além disso, serão comparadas diferentes distribuições para os processos de chegada de ordens e suas intensidades, especificamente os Processos de Hawkes, Cadeias de Markov Autoregressivas e Modelos de Hawkes Dependentes do Estado. Por fim, a principal contribuição do trabalho introduzir o uso de algoritmos genéticos como técnica para aproximar as distribuições dos processos de chegada de ordens dependentes dos estados, utilizando-se estimativas por máxima verossimilhança para comparar com outras técnicas utilizadas na literatura. 
  
  A abordagem proposta visa melhorar a precisão das simulações e fornecer insights mais robustos sobre o comportamento do mercado financeiro em diferentes cenários.
  O uso de algoritmos evolutivos tem como objetivo contribuir no desenvolvimento e na melhor compreensão de modelos simuladores do mercado financeiro, assim como o desenvolvimento de estratégias de negociação mais eficientes e resilientes a diferentes regimes de mercado.
\end{abstract}

% Não altere a seguir. As palavras-chave estão no arquivo
% TodasInformações.tex e são usadas em mais de uma parte do texto.
% Abra tal arquivo para eventuais alterações.
\begin{abstract}
\vspace{-.75cm}
\noindent\textbf{Palavras-chave:} \PalavrasChave
\noindent\textbf{JEL:} \JELCodes
\end{abstract}

\clearpage

%%%%%%%%%%%%%%%%%%%%%%%%%%%%%%%%%%%%%%%%%%%%%%%%%%%
% Início da Parte Textual. Conteúdo mais importante

% A opção abaixo, que não deve ser removida, inicia cada capítulo
% em página ímpar
\openright

% Os items abaixo dependem do tipo de trabalho: capítulos, texto
% convencional, etc, de acordo com o Regulamento. Abra os arquivos
% abaixo (estão no diretório chamado "textual") para ver como os
% modelos foram criados. Basicamente, é uma edição usando LaTeX, sem
% modificações substanciais. Modifique o conteúdo dos arquivos, insira
% novos .tex, etc.

\chapter{Introdução}

O livro de ordens limite representa o estado atual de todas ordens aguardando execução. Essencialmente fornece uma representação dinâmica das intenções de compra e venda de ativos financeiros em um determinado mercado por meio de uma lista de ordens, ordenadas por preço e tempo de chegada, onde cada entrada representa o desejo de um participante do mercado de comprar ou vender um certo ativo \citep{Abergel2020, Avellaneda2008}. Os intervalos de tempo entre a chegada de duas ordens subsequentes são comumente representados por processos estocásticos e o estudo desses processos é uma abordagem fundamental para a modelagem e representação da dinâmica dos mercados financeiros \citep{Shi2022, Guilbaud2013, Liu2021}. Tradicionalmente, os modelos de chegada de ordens têm se baseado em processos de contagem, onde os incrementos entre eventes são distribuídos de acordo com distribuições de Poisson ou distribuições exponenciais \citep{Cont2022, Ponta2012}. Recentemente surge grande interesse na utilização de processos de Hawkes para modelar os processos de chegada, devido a natureza autoexcitável dos mesmos \citep{Abergel2020, MorariuPatrichi2022, Toke2011}. 

O uso de uma distribuição com parâmetros fixos para estimar os tempos de chegada não considera as diferentes propriedades dentre os diversos regimes financeiros existentes do mercado, tanto em escala macro, como por exemplo regimes de inflação \citep{Krause2022}, como em escalas de curto prazo, como por exemplo regimes de alta ou baixa volatilidade, liquidez e volume de negociação \citep{Guilbaud2013}, que exibem comportamentos distintos entre si. Esses regimes podem ser influenciados por uma variedade de fatores, incluindo eventos econômicos, políticos e sazonais e resultam em alterações nas características estatísticas dos processos de preços, e chegada de ordens \citep{Krause2022}.

Uma abordagem promissora que considere os efeitos dos diferentes regimes é o uso de modelos ocultos de Markov (\textit{Hidden Markov Chains}, em inglês, ou \textit{HMC}) \citep{Cont2010}. Um \textit{HMC} é um modelo estocástico que assume a existência de estados ocultos, não observáveis diretamente, mas que podem ser inferidos a partir de observações de variáveis externas \citep{Baum1966}. Nesse contexto, os diferentes estados ou regimes implícitos do livro de ordens e as diferentes intensidades de chegada de ordens podem ser representados como estados ocultos da cadeia de Markov e as distribuições ou parâmetros associados com os respectivos estados \citep{MorariuPatrichi2022}. Essa abordagem permite modelar a transição entre diferentes regimes de mercado de forma dinâmica, considerando uma matriz de transição de estados.

Na literatura de simulação de livros de ordem foram utilizados alguns modelos para a identificação e simulação dos diferentes estados do livro, como por exemplo modelos de \textit{Markov Switching} \citep{Guilbaud2013} ou algoritmos de \textit{Thinning} para simulação de processos dependentes de estados \citep{Ponta2012}. Além disso, as distribuições utilizadas para modelar o processo de chegada de ordens são mantidas fixas durante a simulação, com apenas seus parâmetros sendo ajustados e obtidos a partir dos estados atuais. Em suma, essas abordagens requerem que sejam feitas assunções sobre o funcionamento intrínsico do mercado, além de limitar as características estatísticas da simulação apenas ao período observado devido ao ajuste dos parâmetros não capturarem todas características do conjunto de dados \citep{Zare2021}.

Considerando tais problemas, o presente trabalho propõe o uso e comparação de duas abordagens numéricas para otimizar a amostragem das distribuições dos processos de chegada, cujo formato é tomado como desconhecido como ponto de partida inicial: 
\begin{itemize}
	\item Algoritmos genéticos (AGs) têm se destacado como uma ferramenta poderosa para otimização e modelagem em uma variedade de domínios, incluindo finanças \citep{Oesch2013, Katoch2021}. Esses algoritmos podem ser aplicados para aproximar as possíveis distribuições dos processos de chegada de ordens, utilizando estimativas por máxima verossimilhança \citep{Boonthiem2023, Colla2010} para ajustar os parâmetros das populações aos dados históricos observados;
	
	\item Métodos de Monte Carlo para Cadeias de Markov (\textit{Markov chain Monte Carlo } em inglês, ou MCMC) também vêm ganhando destaque como uma abordagem eficaz para amostragem de distribuições complexas e multidimensionais \citep{Rasmussen2013, Rousseau2018}. Os métodos de MCMC, como o algoritmo de Metropolis-Hastings e o Gibbs sampling, são particularmente úteis quando se trata de amostrar a partir de distribuições de probabilidade desconhecidas ou difíceis de obter diretamente \citep{Glasserman2004}. Ao construir uma cadeia de Markov com distribuição estacionária igual à distribuição alvo, esses métodos podem possivelmente gerar amostras que representam fielmente a distribuição desejada.
\end{itemize}

Em suma, este trabalho propõe uma abordagem computacional alternativa para modelar e simular o livro de ordens limite e os processos de chegada de ordens no mercado financeiro, utilizando Hidden Markov Chains e otimizadores de amostragens por meio de bibliotecas de computação em unidades de processamento gráficas (\textit{GPUs}). As abordagens propostas tem o potencial de melhorar significativamente a precisão das simulações e fornecer propriedades estatísticas mais robustas sobre o comportamento do mercado em diferentes regimes para teste de estratégias de negociação algorítmicas.

**Tema**

O livro de ordens limite é uma representação estruturada das intenções de compra e venda de ativos financeiros em um mercado. Nele, os participantes registram suas ordens de compra (com preço máximo que estão dispostos a pagar) e ordens de venda (com preço mínimo que estão dispostos a aceitar). As ordens limite são executadas somente quando o preço de mercado alcança o preço especificado na ordem. Esse mecanismo de negociação é fundamental para entender a dinâmica dos mercados financeiros e como as transações são realizadas.

Os processos de chegada de ordens referem-se à maneira como as ordens são enviadas ao mercado ao longo do tempo. Tradicionalmente, esses processos são modelados por distribuições de Poisson ou exponenciais. No entanto, abordagens mais recentes têm considerado o uso de Processos de Hawkes, que capturam a natureza auto-excitável dos mercados financeiros, onde uma ordem pode desencadear a chegada de outras ordens em um processo de reação em cadeia.

Os mercados financeiros estão sujeitos a diferentes regimes, como períodos de alta volatilidade e períodos de baixa volatilidade. Esses regimes podem influenciar significativamente as dinâmicas do livro de ordens limite e a intensidade das chegadas de ordens. Portanto, é crucial modelar esses regimes para capturar a complexidade do mercado.

A utilização de Hidden Markov Chains (HMC) oferece uma abordagem poderosa para modelar os diferentes estados do livro de ordens limite, assim como as transições entre esses estados. Cada estado pode representar um regime financeiro específico, com diferentes características de comportamento do mercado. Além disso, as intensidades de chegada de ordens podem variar dependendo do estado do mercado, refletindo a volatilidade e a atividade de negociação.

Agora, introduzindo os algoritmos genéticos, podemos explorar sua aplicação para aproximar as possíveis distribuições dos processos de chegada de ordens. Os algoritmos genéticos são uma técnica de otimização inspirada no processo de seleção natural. Eles podem ser usados para encontrar os parâmetros que melhor se ajustam aos dados históricos, replicando as características estatísticas observadas. Isso permite a geração de cenários realistas no simulador, fornecendo uma base sólida para testar estratégias de negociação em diferentes condições de mercado.

Em resumo, o tema deste trabalho abrange a modelagem do livro de ordens limite, incluindo a representação dos processos de chegada de ordens, a consideração dos regimes financeiros e o uso de Hidden Markov Chains para capturar a complexidade do mercado. Além disso, discute-se a aplicação de algoritmos genéticos para aproximar as distribuições dos processos de chegada de ordens, visando melhorar a precisão e a relevância dos simuladores de mercado financeiro.
\chapter{Objetivos do Trabalho}
Este trabalho visa comparar o desempenho de Algoritmos Genéticos (AGs) e Métodos de Monte Carlo para Cadeias de Markov (MCMC) na otimização da amostragem das distribuições dos processos de chegada de ordens no mercado financeiro, com foco nos resultados computacionais das simulações e de treino, especificamente os tempos de execução para teste, geração de dados e tempo para realização dos testes estatísticos, tanto em CPU, quanto em unidades de processamento gráficas (\textit{GPU}) utilizando bibliotecas de computação vetorial. A primeira abordagem utiliza AGs para ajustar os parâmetros das distribuições aos dados históricos, enquanto a segunda emprega MCMC, como Metropolis-Hastings e Gibbs sampling, para amostrar diretamente das distribuições. O objetivo é avaliar a precisão e eficácia de cada método na replicação das características estatísticas dos processos de chegada em dados históricos, fornecendo insights para estratégias de negociação algorítmicas mais robustas.
\chapter{Metodologia}

Para atingir os objetivos propostos, adotaremos uma metodologia em duas etapas. Primeiramente, será implementado um simulador de mercado com dados históricos estáticos disponibilizados pela bolsa de Nova Yorque (NASDAQ). Em sequência, o simulador será modelado para utilizar um estimador das distribuições de chegada de eventos, especificamente com duas abordagens a serem comparadas: Algoritmos Genéticos (AGs) e Métodos de Monte Carlo para Cadeias de Markov (MCMC) como modelos dos processos de chegada de ordens. Utilizaremos linguagens de programação adequadas para a implementação eficiente desses métodos, em Python com bibliotecas de processamento vetorial, especificamente NumPy, PyMC3 e JAX.

\begin{enumerate}
	\item Na primeira etapa, aplicaremos AGs para otimizar a representação das distribuições dos processos de chegada de ordens. Isso envolverá a definição de uma função de aptidão (fitness) que quantifique a adequação de uma determinada distribuição aos dados históricos. Os parâmetros das distribuições serão ajustados iterativamente através de operadores genéticos, como seleção, recombinação e mutação, para maximizar a função de aptidão.
	
	\item Na segunda etapa, implementaremos e utilizaremos métodos de MCMC, como Metropolis-Hastings e Gibbs sampling, para amostrar diretamente das distribuições dos processos de chegada. Faremos isso construindo cadeias de Markov com distribuição estacionária igual à distribuição alvo. Avaliaremos a convergência das cadeias de Markov e a qualidade das amostras geradas em relação aos dados históricos.
\end{enumerate}

Por fim, conduziremos experimentos computacionais para comparar o desempenho e a eficácia dos dois métodos em termos de precisão na representação dos processos de chegada de ordens. Analisaremos métricas relevantes, como erro médio, tempo de convergência e robustez em diferentes cenários de mercado.
\chapter{Cronograma}
\begin{table}[h]
	\begin{tcolorbox}[width=\textwidth]
	\centering
	\renewcommand{\arraystretch}{1.5}
	\begin{adjustbox}{width=\textwidth}
	\begin{tabular}{|c|c|c|c|c|c|}
		\hline
		\textbf{Etapa} & 
		\textbf{1. Julho} & 
		\textbf{2. Agosto} & 
		\textbf{3. Setembro} & 
		\textbf{4. Outubro} & 
		\textbf{5. Novembro} \\ \hline
		1. Pesquisa Bibliográfica & X & &  &  &  \\ \hline
		2. Análise Bibliográfica & X & X &  &  &  \\ \hline
		3. Obtenção de Dataset &  & X &  &  &  \\ \hline
		4. Definição dos modelos &  & X & X &  &  \\ \hline
		5. Implementação em código &  &  & X & X &  \\ \hline
		6. Análise das métricas &  &  & X & X & X \\ \hline
		7. Escrita do artigo &  & X & X & X & X \\ \hline
	\end{tabular}
\end{adjustbox}
\end{tcolorbox}
\end{table}
\begin{btSect}[bibliography/apalikept]{bibliography/bibliography}
	\section*{Referências}
	\addcontentsline{toc}{section}{Referências}
	\btPrintCited
\end{btSect}

\end{document}
