\begin{btUnit}
\chapter{Introdução}

\section*{Resumo}
\addcontentsline{toc}{section}{Resumo}

\begin{table}
\centering
\caption{Estimates of libero ac erat vulputate feugiat non
  eu justo. Aenean blandit fermentum efficitur. Fusce faucibus turpis
  ac dolor sollicitudin, quis lobortis felis porttitor. Phasellus
  hendrerit quam sit amet erat fringilla, et bibendum ligula feugiat.
  Phasellus odio tortor, mollis ac cursus ut, fermentum ut orci.
  Vestibulum malesuada, justo vitae tincidunt posuere, nibh ipsum
  pellentesque mauris, et suscipit magna turpis pellentesque magna.
  Mauris sed dui vitae velit facilisis pharetra eu in enim.
  Suspendisse fringilla lobortis dolor eget sollicitudin.}
\begin{tabular}{ c p{5.5pc} c p{5.5pc} }
\hline
\multicolumn{2}{c}{Múltiplas colunas} & Aqui & Acolá \\
\hline
Um & \raggedright\arraybackslash Largura fixa em (5.5pc). & Três &
\raggedright\arraybackslash Coluna 4, largura fixa.\\
\hline
\label{Tab:estim}
\end{tabular}
\end{table}
us fringilla hendrerit.

Citação no meio do texto \citet{Casella-Berger2011} e entre parênteses
\citep{Gelman2007}

\begin{btSect}[referencias/apalikept]{referencias/bibliografia}
\section*{Referências}
\addcontentsline{toc}{section}{Referências}
\btPrintCited
\end{btSect}

\end{btUnit}
