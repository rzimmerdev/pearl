\chapter{Introdução}

O livro de ordens limite representa o estado atual de todas ordens aguardando execução. Essencialmente fornece uma representação dinâmica das intenções de compra e venda de ativos financeiros em um determinado mercado por meio de uma lista de ordens, ordenadas por preço e tempo de chegada, onde cada entrada representa o desejo de um participante do mercado de comprar ou vender um certo ativo \citep{Abergel2020, Avellaneda2008}. Os intervalos de tempo entre a chegada de duas ordens subsequentes são comumente representados por processos estocásticos e o estudo desses processos é uma abordagem fundamental para a modelagem e representação da dinâmica dos mercados financeiros \citep{Shi2022, Guilbaud2013, Liu2021}. Tradicionalmente, os modelos de chegada de ordens têm se baseado em processos de contagem, onde os incrementos entre eventes são distribuídos de acordo com distribuições de Poisson ou distribuições exponenciais \citep{Cont2022, Ponta2012}. Recentemente surge grande interesse na utilização de processos de Hawkes no lugar de processos de Poisson com taxas dinâmicas para modelar os processos de chegada, devido a natureza autoexcitável dos mesmos \citep{Abergel2020, MorariuPatrichi2022, Toke2011}. 

O uso de uma distribuição com parâmetros fixos para estimar os tempos de chegada não considera as diferentes propriedades de múltiplos regimes financeiros existentes do mercado, tanto em escala macro, como por exemplo regimes de inflação \citep{Krause2022}, como em escalas de curto prazo, como por exemplo regimes de alta ou baixa volatilidade, liquidez e volume de negociação \citep{Guilbaud2013}, que exibem comportamentos distintos entre si. Esses regimes podem ser influenciados por uma variedade de fatores, incluindo eventos econômicos, políticos e sazonais e resultam em alterações nas características estatísticas dos processos de preços, e chegada de ordens \citep{Krause2022}.

Uma abordagem promissora para acatar os efeitos dos diferentes regimes é o uso de modelos ocultos de Markov (\textit{Hidden Markov Chains}, em inglês, ou \textit{HMC}) \citep{Cont2010}. Um \textit{HMC} é um modelo estocástico que assume a existência de estados ocultos, não observáveis diretamente, mas que podem ser inferidos a partir de observações de variáveis externas \citep{Baum1966}. Nesse contexto, os diferentes estados ou regimes implícitos do livro de ordens e as diferentes intensidades de chegada de ordens podem ser representados como estados ocultos da cadeia de Markov e as distribuições ou parâmetros associados com os respectivos estados \citep{MorariuPatrichi2022}. Essa abordagem permite modelar a transição entre diferentes regimes de mercado de forma dinâmica, considerando uma matriz de transição de estados, utilizando métodos de Thinning ou \citep{Ponta2012} de maximização de esperança. Em suma, essas abordagens requerem que sejam feitas assunções sobre o funcionamento intrínsico do mercado, além de limitar as características estatísticas da simulação apenas ao período observado devido ao ajuste dos parâmetros não capturarem todas características do conjunto de dados \citep{Zare2021}.

Por fim, a representação dos estados não-observáveis dos livros de ordem foram também representados na literatura utilizando redes neurais, onde as distribuições utilizadas para modelar o processo de chegada de ordens são obtidas a partir das saídas de redes. Resultados promissores surgiram com o uso de redes com nós de \textit{Long-short term memory} (LSTMs), com nós de atenção (\textit{Transformers}) e com arquiteturas generativas adversariais (GANs). Os parâmetros da rede são ajustados sob os dados históricos, onde as camadas internas da rede representam os estados não-observáveis do mercado. 

Considerando a diversidade de técnicas existentes, o presente trabalho propõe a implementação e comparação das técnicas de simulação. Nas seções de Objetivos serão discutidas as métricas e fundamentos por trás do \textit{framework} para comparação. A seguir propomos uma descrição dos algoritmos a serem discutidos e implementados na tese:

\begin{itemize}
	\item \textbf{Processos de Ponto} com taxas fixas são modelos estatísticos simples, frequentemente utilizados como referências base para as dinâmicas de mercado, e geralmente simples de implementar, e tempo de execução extremamente baixo quando comparado à outras técnicas. O uso desses processos como simuladores de chegada de ordens não é recomendado para simulações reais, apenas como ponto de partida para comparações \citep{Bouchaud2002, Cont2011}

	\item \textbf{Processos autoexcitantes de Hawkes} têm sido amplamente explorados na literatura financeira devido à sua capacidade de modelar a autoexcitação nos processos de chegada de ordens em mercados financeiros. Desde sua introdução por Hawkes \citep{Hawkes1971}, a literatura a respeito do uso desses processos para dinâmicas de livros de ordem foca em variações e extensões multidimensionais dos processos de Hawkes para capturar nuances e regimes específicos dos mercados financeiros, como modelos multivariados para a interdependência entre diferentes ativos \citep{Bacry2015} e processos de Hawkes dependentes de estados observáveis \citep{MorariuPatrichi2022}. No entanto, é importante considerar as limitações e pressupostos desses modelos, bem como outras abordagens complementares, para uma análise abrangente dos mercados financeiros.
	
	\item \textbf{Modelos Ocultos de Markov} (HMM) são modelos estatísticos que são usados para modelar sistemas com estados ocultos. Eles consistem em uma cadeia de Markov com uma matriz de transição dos estados ocultos e outra matriz de emissão com as distribuições dos valores observáveis para cada estado oculto \citep{Li2005}. No contexto dos livros de ordens, os HMMs podem ser utilizados para representar os diferentes regimes de mercado como estados ocultos, enquanto as observações devem ser adequadas para representar as características das ordens e seus padrões de chegada \citep{Sandoval2015}.
	
	\item \textbf{Métodos de Monte Carlo para Cadeias de Markov} (\textit{Markov chain Monte Carlo } em inglês, ou MCMC) também vêm ganhando destaque como uma abordagem eficaz para amostragem de distribuições complexas e multidimensionais \citep{Rasmussen2013, Rousseau2018}. Os métodos de MCMC, como o algoritmo de Metropolis-Hastings e o Gibbs sampling, são particularmente úteis quando se trata de amostrar a partir de distribuições de probabilidade desconhecidas ou difíceis de obter diretamente \citep{Glasserman2004}. Ao construir uma cadeia de Markov com distribuição estacionária igual à distribuição alvo, esses métodos podem possivelmente gerar amostras que representam fielmente a distribuição desejada.
	
	\item \textbf{Redes Neurais} consistem em camadas de neurônios interconectados, onde cada neurônio processa informações e passa para os neurônios nas camadas seguintes. No contexto do livro de ordens, as redes neurais podem ser treinadas para replicar os padrões históricos de ordens, incluindo a dinâmica das chegadas, preços e volume. As arquiteturas de redes neurais utilizadas para a tarefa de simulação do livro de ordens incluem Long Short-Term Memory (LSTM) para modelar dependências de longo prazo \citep{Shi2022} e redes generativas adversariais (GANs) para gerar dados realistas de livro de ordens \citep{Coletta2022, Cont2022}
	
\end{itemize}

Em suma, este trabalho propõe uma abordagem computacional comparativa para modelar e simular o livro de ordens limite e os processos de chegada de ordens no mercado financeiro, utilizando Hidden Markov Chains otimizadores dos modelos com dados históricos (treinamento) por meio de bibliotecas de computação em unidades de processamento gráficas (\textit{GPUs}), vetorização e computação paralela. As abordagens propostas tem o potencial de melhorar significativamente a aplicabilidade das simulações e fornecer métricas para auxiliar na decisão e uso desses algoritmos em cenários reais.
