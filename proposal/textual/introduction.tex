\chapter{Introdução}

O livro de ordens limite representa o estado atual de todas ordens aguardando execução. Essencialmente fornece uma representação dinâmica das intenções de compra e venda de ativos financeiros em um determinado mercado por meio de uma lista de ordens, ordenadas por preço e tempo de chegada, onde cada entrada representa o desejo de um participante do mercado de comprar ou vender um certo ativo. Os intervalos de tempo entre a chegada de duas ordens subsequentes são comumente representados por processos estocásticos e o estudo desses processos é uma abordagem fundamental para a modelagem e representação da dinâmica dos mercados financeiros. Tradicionalmente, os modelos de chegada de ordens têm se baseado em processos de contagem, onde os incrementos entre eventes são distribuídos de acordo com distribuições de Poisson ou distribuições exponenciais. Recentemente surge grande interesse na utilização de processos de Hawkes para modelar os processos de chegada, devido a natureza autoexcitável dos mesmos.

O uso de uma distribuição com parâmetros fixos para estimar os tempos de chegada não é uma abordagem adequada ao se considerar as diferenças nos tempos dentre diferentes regimes financeiros do mercado, tanto em escala macro, como por exemplo regimes de inflação, como em escalas de curto prazo, como por exemplo regimes de alta ou baixa volatilidade, liquidez e volume de negociação, que exibem comportamentos distintos entre si. Esses regimes podem ser influenciados por uma variedade de fatores, incluindo eventos econômicos, políticos e sazonais.

Uma abordagem promissora que considere os efeitos dos diferentes regimes é o uso de modelos ocultos de Markov (\textit{Hidden Markov Chains}, em inglês, ou \textit{HMC}). Um \textit{HMC} é um modelo estocástico que assume a existência de estados ocultos, não observáveis diretamente, mas que podem ser inferidos a partir de observações de variáveis externas. Nesse contexto, os diferentes estados ou regimes implícitos do livro de ordens e as diferentes intensidades de chegada de ordens podem ser representados como estados ocultos da cadeia de Markov e as distribuições ou parâmetros associados com os respectivos estados. Essa abordagem permite modelar a transição entre diferentes regimes de mercado de forma dinâmica, considerando uma matriz de transição de estados.

Na literatura de simulação de livros de ordem foram utilizados alguns modelos para a identificação e simulação dos diferentes estados do livro, como por exemplo modelos de \textit{Markov Switching} ou algoritmos de \textit{Thinning} para simulação de processos dependentes de estados. Além disso, as distribuições utilizadas para modelar o processo de chegada de ordens são mantidas fixas durante a simulação, com apenas seus parâmetros sendo ajustados e obtidos a partir dos estados atuais. Em suma, essas abordagens requerem que sejam feitas assunções sobre o funcionamento intrínsico do mercado, além de limitar as características estatísticas da simulação apenas ao período observado devido ao ajuste dos parâmetros não capturarem todas características do conjunto de dados.

Considerando tais problemas, o presente trabalho propõe o uso e comparação de duas abordagens numéricas para otimizar a amostragem das distribuições dos processos de chegada, cujo formato é tomado como desconhecido como ponto de partida inicial: 
\begin{itemize}
	\item Algoritmos genéticos (AGs) têm se destacado como uma ferramenta poderosa para otimização e modelagem em uma variedade de domínios, incluindo finanças. Esses algoritmos podem ser aplicados para aproximar as possíveis distribuições dos processos de chegada de ordens, utilizando estimativas por máxima verossimilhança para ajustar os parâmetros das populações aos dados históricos observados;
	
	\item Métodos de Monte Carlo para Cadeias de Markov (\textit{Markov chain Monte Carlo } em inglês, ou MCMC) também vêm ganhando destaque como uma abordagem eficaz para amostragem de distribuições complexas e multidimensionais. Os métodos de MCMC, como o algoritmo de Metropolis-Hastings e o Gibbs sampling, são particularmente úteis quando se trata de amostrar a partir de distribuições de probabilidade desconhecidas ou difíceis de obter diretamente. Ao construir uma cadeia de Markov com distribuição estacionária igual à distribuição alvo, esses métodos permitem gerar amostras que representam fielmente a distribuição desejada.
\end{itemize}

Em suma, este trabalho propõe uma abordagem computacional alternativa para modelar e simular o livro de ordens limite e os processos de chegada de ordens no mercado financeiro, utilizando Hidden Markov Chains e otimizadores de amostragens por meio de bibliotecas de computação em unidades de processamento gráficas (\textit{GPUs}). As abordagens propostas tem o potencial de melhorar significativamente a precisão das simulações e fornecer propriedades estatísticas mais robustas sobre o comportamento do mercado em diferentes regimes para teste de estratégias de negociação algorítmicas.
