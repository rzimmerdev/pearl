**Tema**

O livro de ordens limite é uma representação estruturada das intenções de compra e venda de ativos financeiros em um mercado. Nele, os participantes registram suas ordens de compra (com preço máximo que estão dispostos a pagar) e ordens de venda (com preço mínimo que estão dispostos a aceitar). As ordens limite são executadas somente quando o preço de mercado alcança o preço especificado na ordem. Esse mecanismo de negociação é fundamental para entender a dinâmica dos mercados financeiros e como as transações são realizadas.

Os processos de chegada de ordens referem-se à maneira como as ordens são enviadas ao mercado ao longo do tempo. Tradicionalmente, esses processos são modelados por distribuições de Poisson ou exponenciais. No entanto, abordagens mais recentes têm considerado o uso de Processos de Hawkes, que capturam a natureza auto-excitável dos mercados financeiros, onde uma ordem pode desencadear a chegada de outras ordens em um processo de reação em cadeia.

Os mercados financeiros estão sujeitos a diferentes regimes, como períodos de alta volatilidade e períodos de baixa volatilidade. Esses regimes podem influenciar significativamente as dinâmicas do livro de ordens limite e a intensidade das chegadas de ordens. Portanto, é crucial modelar esses regimes para capturar a complexidade do mercado.

A utilização de Hidden Markov Chains (HMC) oferece uma abordagem poderosa para modelar os diferentes estados do livro de ordens limite, assim como as transições entre esses estados. Cada estado pode representar um regime financeiro específico, com diferentes características de comportamento do mercado. Além disso, as intensidades de chegada de ordens podem variar dependendo do estado do mercado, refletindo a volatilidade e a atividade de negociação.

Agora, introduzindo os algoritmos genéticos, podemos explorar sua aplicação para aproximar as possíveis distribuições dos processos de chegada de ordens. Os algoritmos genéticos são uma técnica de otimização inspirada no processo de seleção natural. Eles podem ser usados para encontrar os parâmetros que melhor se ajustam aos dados históricos, replicando as características estatísticas observadas. Isso permite a geração de cenários realistas no simulador, fornecendo uma base sólida para testar estratégias de negociação em diferentes condições de mercado.

Em resumo, o tema deste trabalho abrange a modelagem do livro de ordens limite, incluindo a representação dos processos de chegada de ordens, a consideração dos regimes financeiros e o uso de Hidden Markov Chains para capturar a complexidade do mercado. Além disso, discute-se a aplicação de algoritmos genéticos para aproximar as distribuições dos processos de chegada de ordens, visando melhorar a precisão e a relevância dos simuladores de mercado financeiro.