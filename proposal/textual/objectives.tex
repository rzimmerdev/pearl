\chapter{Objetivos do Trabalho}
Este trabalho visa comparar o desempenho dos modelos de simulação, com foco nos resultados computacionais das simulações e de treino, especificamente as métricas resultantes da implementação sequencial, paralela e distribuida dos algoritmos, assim como do processo de carregamento dos dados, tanto em CPU, quanto em unidades de processamento gráficas (\textit{GPU}) utilizando bibliotecas de computação vetorial. A arquitetura do \textit{framework} para simulação seguirá um formato replicável, utilizando arquivos para containerização dos códigos, e um orquestrador de conjunto de nós de simulação para processar a geração e agregação das ordens de forma independente, distribuindo a carga de trabalho e reduzindo o tempo de execução. Cada nó gera métricas como taxa de execução, eficiência de processamento e latência de transação, que são agregadas por um componente centralizado para análise. A comunicação entre os nós é realizada por meio de trocas de mensagens assíncronas, minimizando o overhead de comunicação. A arquitetura é dimensionada horizontalmente para lidar com aumentos na carga de trabalho, adicionando nós conforme necessário, e utiliza técnicas de tolerância a falhas para garantir a disponibilidade contínua do sistema. Por fim, as métricas geradas pelo sistema serão apresentadas como resultados da tese, e o funcionamento do \textit{framework} e da usabilidade dos paradigmas serão discutidos na conclusão.
