\chapter{Objetivos do Trabalho}
Este trabalho visa comparar o desempenho de Algoritmos Genéticos (AGs) e Métodos de Monte Carlo para Cadeias de Markov (MCMC) na otimização da amostragem das distribuições dos processos de chegada de ordens no mercado financeiro, com foco nos resultados computacionais das simulações e de treino, especificamente os tempos de execução para teste, geração de dados e tempo para realização dos testes estatísticos, tanto em CPU, quanto em unidades de processamento gráficas (\textit{GPU}) utilizando bibliotecas de computação vetorial. A primeira abordagem utiliza AGs para ajustar os parâmetros das distribuições aos dados históricos, enquanto a segunda emprega MCMC, como Metropolis-Hastings e Gibbs sampling, para amostrar diretamente das distribuições. O objetivo é avaliar a precisão e eficácia de cada método na replicação das características estatísticas dos processos de chegada em dados históricos, fornecendo insights para estratégias de negociação algorítmicas mais robustas.