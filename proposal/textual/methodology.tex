\chapter{Metodologia}

Para atingir os objetivos propostos, adotaremos uma metodologia em duas etapas. Primeiramente, será implementado um simulador de mercado com dados históricos estáticos disponibilizados pela bolsa de Nova Yorque (NASDAQ) \citep{Nagy2023}. Em sequência, o simulador será modelado para utilizar um estimador das distribuições de chegada de eventos, especificamente com duas abordagens a serem comparadas: Algoritmos Genéticos (AGs) e Métodos de Monte Carlo para Cadeias de Markov (MCMC) como modelos dos processos de chegada de ordens. Utilizaremos linguagens de programação adequadas para a implementação eficiente desses métodos, em Python com bibliotecas de processamento vetorial, especificamente NumPy, PyMC3 e JAX \citep{Bradbury2018, Oriol2023}.

\begin{enumerate}
	\item Na primeira etapa, aplicaremos AGs para otimizar a representação das distribuições dos processos de chegada de ordens. Isso envolverá a definição de uma função de aptidão (fitness) que quantifique a adequação de uma determinada distribuição aos dados históricos. Os parâmetros das distribuições serão ajustados iterativamente através de operadores genéticos, como seleção, recombinação e mutação, para maximizar a função de aptidão.
	
	\item Na segunda etapa, implementaremos e utilizaremos métodos de MCMC, como Metropolis-Hastings e Gibbs sampling, para amostrar diretamente das distribuições dos processos de chegada. Faremos isso construindo cadeias de Markov com distribuição estacionária igual à distribuição alvo. Avaliaremos a convergência das cadeias de Markov e a qualidade das amostras geradas em relação aos dados históricos.
\end{enumerate}

Por fim, conduziremos experimentos computacionais para comparar o desempenho e a eficácia dos dois métodos em termos de precisão na representação dos processos de chegada de ordens. Analisaremos métricas relevantes, como erro médio, tempo de convergência e robustez em diferentes cenários de mercado.