\chapter{Metodologia}

Para atingir os objetivos propostos, adotaremos uma metodologia em duas etapas. Primeiramente, será implementado um \textit{dataloader} de mercado com dados históricos estáticos disponibilizados pela bolsa de Nova Yorque (NASDAQ) \citep{Nagy2023}. Em sequência, o simulador será modelado para utilizar um estimador das distribuições de chegada de eventos, especificamente com as quatro técnicas discutidas na seção de introdução. Utilizaremos linguagens de programação que possibilitem a implementação desses métodos, em Python com bibliotecas de processamento vetorial, especificamente NumPy, PyMC3 e JAX \citep{Bradbury2018, Oriol2023}.

\begin{enumerate}
	\item Na primeira etapa, os algoritmos serão colocados em contâiners \textit{Docker} para permitir a replicabilidade e distribuição das cargas de processamento. Isso envolverá a definição de um orquestrador e agregador de ordens que centralize as ordens no livro de ordens limite. Os parâmetros dos modelos serão ajustados na etapa de treinamento, e salvos localmente para serem carregados nos contâiners.
	
	\item Na segunda etapa, implementaremos e registraremos as métricas de eficiência da computação paralela e distribuida, assim como de overhead, de utilização de hardware, escalabilidade e speedup, e overhead de comunicação com o sistema agregador. Para os treinamentos dos simuladores avaliaremos o tempo para convergência e de treino com diferentes intervalos de dados históricos.
\end{enumerate}

Por fim, as análises de desempenho e eficiências dos métodos serão comparados para diferentes cenários e regimes de mercado.